\section{The \Acstitle{LAT} Instrument Response Functions}
\seclabel{lat_irfs}

The performance of the \ac{LAT} is quantified by its effective area and
its dispersion. The effective area represents the collection area of the
\ac{LAT} and the dispersion represents the probability of misreconstructing
the true parameters of the incident $\gamma$-ray.
The effective area $\effectivearea(\energy,\time,\solidangle)$ is a
function of energy, time, and \ac{SA} and is measured in units of $\cm^2$.

The dispersion is the probability of a photon with true energy
\energy and incoming direction $\solidangle$ at time \time being
reconstructed to have an energy $\energy'$, an incoming direction
$\solidangle'$ at a time $\time'$.  The dispersion is written as
$\dispersion(\energy',\time',\solidangle'|\energy,\time,\solidangle)$.
It represents a probability and is therefore normalized such that
\begin{equation}
  \int \int \int \denergy \dsolidangle \dtime 
  \dispersion(\energy',\time',\solidangle'|\energy,\time,\solidangle) = 1
\end{equation}
Therefore,
$\dispersion(\energy',\time',\solidangle'|\energy,\time,\solidangle)$
has units of 1/energy/\acs{SA}/time

The convolution of the model a source with the \acp{IRF} produces the
expected differential counts (counts per unit energy/time/\acs{SA})
that are reconstructed to have an energy $\energy'$ at a position
$\solidangle'$ and at a time $\time'$:
\begin{equation}
  \eqnlabel{eventrate}
  \eventrate(\energy',\solidangle',\time'|\modelparams)
  = \int \int \int \denergy \, \dsolidangle \, \dtime \,
  \fluxdensity(\energy,\time,\vec\Omega|\modelparams) 
  \effectivearea(\energy,\time,\solidangle) \dispersion(\energy',\time',\solidangle'|\energy,\time,\solidangle)
\end{equation}
Here, this integral is performed over all energies, \acp{SA}, and times.

For \ac{LAT} analysis, we conventionally make the simplifying assumption that
the energy, spatial, and temporal dispersion decouple:
\begin{equation}
  \dispersion(\energy',\time',\solidangle'|\energy,\time,\solidangle) = 
  \psf(\solidangle'|E,\solidangle) \edisp(\energy'|\energy) \tdisp(\time'|\time)
\end{equation}
\edisp represents the energy dispersion of the \ac{LAT}. The energy
dispersion is a function of both the incident energy and angle of
the photon. It varies from $\sim$ 5\% to 20\%, degrading at lower
energies due to energy losses in the tracker and at higher energy due
to electromagnetic shower losses outside the calorimeter. Similarly,
it improves for photons with higher incident angles which are allowed a
longer path through the calorimeter \citep{ackermann_2012a_fermi-large}.
\subsecref{performance_lat} includes a plot of the \edisp of the \ac{LAT}.

$\psf(\solidangle'|E,\solidangle)$ is the probability of reconstructing
a $\gamma$-ray to have a position $\solidangle'$ if the true position of
the $\gamma$-ray has a position $\solidangle$.  For the \ac{LAT}, the
\ac{PSF} is a strong function of energy.  \subsecref{performance_lat}
plots the \ac{PSF} of the \ac{LAT}.

Finally, we note that in principle, there is a finite timing resolution
of $\gamma$-rays measured by the \ac{LAT}. But the timing accuracy is
$<10\unitspace\microsecond$ \citep{atwood_2009a_large-telescope}. Since
this is much less than the smallest timing signal which is expected to
be observed by the \ac{LAT} (millisecond pulsars), issues with timing
accuracy are typically ignored.

For a typical analysis of \ac{LAT} data, we also ignore the inherent
energy dispersion of the \ac{LAT}.  \cite{ackermann_2012a_fermi-large}
performed a monte carlo simulation to show that for power-law point-like
sources, the bias introduced by ignoring energy dispersion was on the
level of a few percent.  Therefore, the instrument response is typically
approximated as
\begin{equation}
  \response(\energy',\solidangle',\time'|\energy,\solidangle,) = 
  \effectivearea(\energy,\time',\solidangle) \psf(\solidangle'|E,\solidangle)
\end{equation}
We caution that for analysis of sources extended to energies below
$100\unitspace\mev$, the effects of energy dispersion are be more
severe.

The differential count rate is typically integrated over time 
assuming that the source model is time independent:
\begin{equation}\eqnlabel{differential_model_counts}
  \eventrate(\energy',\solidangle'|\modelparams)
  = \int \dsolidangle \,
  \fluxdensity(\energy',\vec\Omega|\modelparams) 
\left(
\int \dtime \intspace \effectivearea(\energy',\time,\solidangle) 
\right)
\psf(\solidangle'|E,\solidangle)
\end{equation}
This equation says that the counts expected by the \ac{LAT} from a
given model is the product of the source's flux with the effective
area and then convolved with the \ac{PSF}.  Finally, we note that the
\psf and effective area are also functions of the conversion type of the
$\gamma$-ray (front-entering or back-entering photons), and the azimuthal
angle of the $\gamma$-ray.  \eqnref{differential_model_counts} can be
generalized to include these effects.

