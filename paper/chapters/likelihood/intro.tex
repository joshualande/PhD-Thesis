In this chapter, we discuss maximum-likelihood analysis, the primary
analysis method used to perform spectral and spatial analysis of
\ac{LAT} data.  In \secref{motivations_maximum_likelihood}, we discuss the
reasons necessary for employing this analysis procedure compared to 
simpler analysis methods.  In \secref{description_maximum_likelihood},
we describe the benefits of a maximum-likelihood analysis.
In \secref{defining_model}, we discuss the steps invovled in defining
a complete model of the sky, a necessary part of any likelihood analysis.

In \secref{binned_science_tools}, we discuss the standard implementation
of binned maximum likelihood in the \ac{LAT} Science Tools and
in particular the tool \gtlike.  In \secref{pointlike_package},
we then discuss the \pointlike pacakge, an alterate package for
maximum-likelihood analysis of \ac{LAT} data.  In the next chapter
(\chapref{extended_analysis}), we functionality written into \pointlike
for studying spatially-extended sources.  That much of the notation
and formulation of likelihood analysis in this chapter follows
\cite{kerr_2010a_likelihood-methods}.
