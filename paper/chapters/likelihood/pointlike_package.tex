\section{The Alternate Maximum-Likelihood Package \pointlike}
\seclabel{pointlike_package}

\pointlike is an alternative maximum-likelihood framework developed
for analyzing \ac{LAT} data. In principle, both \pointlike and \gtlike
perform the same binned maximum-likelihood analysis described in
\secref{binned_science_tools}. \pointlike's major design difference
is that it was written with efficiency in mind. The primary use case
for \pointlike is fitting procedures which require multiple iterations
such as source finding, position and extension fitting, computing large
residual \ts maps.

What makes maximum-likelihood analysis of \ac{LAT} data difficult
is the strongly non-linear performance of the \ac{LAT} (see
\subsecref{performance_lat}). At lower energies, one typically finds
lots of photons but each photon is not very significant due to the
poor angular resolution of the instrument. At these energies, a binned
analysis with coarse bins is perfectly adequate to study the sky.
But at higher energies, there are limited numbers of photons due to
the limited source fluxes. But because the angular resolution is much
improved, these photons become much more important. At these energies,
an unbinned analysis which loops over each photon is more appropriate.

The primary efficiency gain of \pointlike comes from scaling the bin
size with energy so that the bin size is always comparable to the
\ac{PSF}.  To do this, \pointlike bins the sky into \healpix pixels
\citep{gorski_2005_healpix:-framework}, but only keeps bins with counts
in them.  At low energy, the bins are large and essentially every 
bin has many counts in it.  But at high energy, bins are very small and rarely
have more than one count in them.  So \pointlike essentially does a binned
analysis at low energy, approximates an unbinned analysis at high energy,
and naturally interpolates between the two extremes.

There is one obvious trade-off for keeping only bins with counts in them.
Using \eqnref{log_likelihood_sum_theta}, we note that the evaluation
of the $\sum_j k_j\log\theta_j$ term can easily be evaluation if only
the counts and model counts are computed in bins with counts in them.
But the $\sum_j \theta_j$ term (the overall model predicted counts in
all bins). To avoid this, \pointlike has to independently compute the
integral of the model counts.

More details about the implementation of \pointlike can be found in
\cite{kerr_2010a_likelihood-methods}. We will discuss the implementation
of extended sources in \pointlike in \chapref{extended_analysis}.
