\chapter{Overview}

In \chapref{gamma_ray_astro}, we discuss the history of $\gamma$-ray
astrophysics. First we present broadly the history of astronomy in
\secref{astronomy_and_the_atmosphere} and the history of $\gamma$-ray
astrophysics in \secref{history_gamma_ray_detectors}.  Then, we discuss
the \fermi Gamma-ray Space Telescope in \secref{fermi_telescope}.  Next,
we discuss historical developments in our understanding of pulsars and
\ac{PWN} in \secref{pulsars_and_pwn}.  We conclude by discussing the major
source classes detected by the \ac{LAT} in \secref{sources_detected_fermi}
adn the major radiation processes that occur in high-energy astrophysics
in \secref{radiation_processes}.

In \chapref{pulsar_pwn_system}, we discuss our current understanding
of the physics of pulsars and \ac{PWN}.  We discuss the formation of a
pulsar in \secref{neutron_star_formation} and the time evolution of a
pulsar in \secref{pulsar_evolution}.  Then, we describe the magnetosphere
of the pulsar in \secref{pulsar_magnetosphere} and the structure of a
typical \ac{PWN} in \secref{pwn_structure}.  Finally, we describe the
energy spectrum emitted from a typical \ac{PWN} in \secref{pwn_emission}.

In \chapref{maximum_likelihood_analysis}, we discuss maximum-likelihood
analysis and how it can be used to analyze \ac{LAT} data.
We describe the motivation for using maximum-likelihood analysis
in \secref{motivations_maximum_likelihood}
and the maximum-likelihood formulation in
\secref{description_maximum_likelihood}.  Then, we describe how to build
a model of the $\gamma$-ray sky in \secref{defining_model} and describe
the \ac{LAT} \acp{IRF} in \secref{lat_irfs}.
Finally, 
we describe the standard
package \gtlike for performing maximum-likelihood analysis of \ac{LAT}
data in \secref{binned_science_tools} and we describe \pointlike,
an alternate package for performing maximum-likelihood analysis of
\ac{LAT} data, in \secref{pointlike_package}.

In \chapref{extended_analysis}, we discuss a new method to study
spatially-extended sources.  We discuss the formulation of this
method in \secref{analysis_methods_section}.  We validate the
extension-significance calculation in \secref{validate_ts} and then
we compute the sensitivity of the \ac{LAT} to spatially-extended
sources in \secref{extension_sensitivity}.  We develop a new method
to compare the hypothesis of multiple point-like sources to one
spatially-extended source in \secref{dual_localization_method} and
finally in \secref{test_2lac_sources} we validate our method by testing
point-like sources from \ac{2LAC} for extension.

In \chapref{extended_search}, we apply the extension test developed in
\chapref{extended_analysis} to search for new spatially-extended sources.
First, we validate the method by studying known spatially-extended
\ac{LAT} sources in \secref{validate_known}.  

In \secref{systematic_errors_on_extension}, we develop a method
to estimate systematic errors associated with studying extended
sources and in \secref{extended_source_search_method} we develop
a method to search for new spatially-extended sources.  Finally,
we discuss the newly-discovered spatially-extended sources in
\secref{new_ext_srcs_section} and the population of spatially-extended
\ac{LAT} sources in \secref{extended_source_discussion}.

In \chapref{offpeak}, we perform a search for new \ac{PWN} in the
off-peak regions of \ac{LAT}-detected pulsars.  First, we develop
a new method to define the off-peak regions used for the search in
\secref{peak_definition}.  Then, we describe the analysis method we
used to search these regions in \secref{off_peak_analysis} and the
results of this search in \secref{off_peak_results}.  Finally,
we discuss some of the sources detected with this method in
\secref{off_peak_individual_source_discussion}.

In \chapref{tevcat}, we perform a search for $\gamma$-ray emission
from \ac{VHE} \ac{PWN}.  We discuss our list of \ac{VHE} candidates
in \secref{tevcat_list_vhe_pwn_candidates} and our analysis method
in \secref{tevcat_analysis_method}.  Finally, we several new \ac{PWN}
detected using this method in \secref{tevcat_sources_detected}.

In \chapref{population_study}, we describe the population of $\gamma$-ray
emitting \ac{PWN} and study how they evolve with the spin-down energy
and age of their associated pulsars.  Finally, in \chapref{outlook} we
remark on potential future opportunities to expand our understanding of
\acp{PWN} and their $\gamma$-ray emission.
