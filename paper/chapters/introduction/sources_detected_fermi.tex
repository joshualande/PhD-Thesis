
\section{Sources Detected by \Actitle{LAT}}
\seclabel{sources_detected_fermi} 

\begin{figure}[htbp]
  \centering
    \includegraphics[width=\textwidth]{chapters/introduction/figures/lat_skymap_2fgl.pdf}
  \caption{
  An Aitoff projection map of the $\gamma$-ray sky observed by the
  \ac{LAT} with a 2 year exposure.  This map is integrated in the
  energy range from $100\unitspace\mev$ to $10\unitspace\gev$ in units
  of $10^{-7}\erg\unitspace\cm^{-2}\second^{-1}\steradian^{-1}$.  This figure is
  from \cite{nolan_2012_fermi-large}.
  }
  \figlabel{lat_skymap_2fgl}
\end{figure}

\figref{lat_skymap_2fgl} shows a map of the $\gamma$-ray sky observed
by the \ac{LAT} with two years of data. One can clearly observed a
strongly-structured anisotropic component of the $\gamma$-ray emission
coming from the galaxy. In addition, many individual sources of $\gamma$-rays
can be viewed. In \subsecref{galactic_diffuse_and_isotropic}, we
discuss the Galactic diffuse and isotropic $\gamma$-ray background. In
\subsecref{2fgl}, we discuss \ac{2FGL}, a catalog of point-like
sources detected by the \ac{LAT}. In \subsecref{2pc}, we discuss
\ac{2PC}, a catalog of pulsars detected by the \ac{LAT}.  Finally,
in \subsecref{pwn_detected_lat} we discuss \acp{PWN} detected by the
\ac{LAT}.


\subsection{The Galactic Diffuse and Isotropic Gamma-ray Background}
\subseclabel{galactic_diffuse_and_isotropic}

The structured Galactic diffuse $\gamma$-ray emission in our galaxy is
caused by The interaction of cosmic-ray electrons and protons with the
gas in our Milky Way (through the \pion and bremsstrahlung process)
and with Galactic radiation fields (through the \ac{IC} process).

Much work has gone into theoretically modeling this
diffuse $\gamma$-ray emission. The most advanced
theoretical model of the Galactic emission is \galprop
\citep{strong_1998a_propagation-cosmic-ray,moskalenko_2000a_anisotropic-inverse}.
In addition, significant work has gone into comparing these
models to the observed $\gamma$-ray intensity
distribution observed by the \ac{LAT} \citep{abdo_2009a_fermi-large,
ackermann_2012a_fermi-lat-observations}.

In addition to the Galactic diffuse background, the \ac{LAT}
observes an isotropic component to the $\gamma$-ray distribution.
This emission is believed to be a composite of unresolved extragalactic
point-like sources as well as a residual charged-particle background.
\cite{abdo_2010a_spectrum-isotropic} presents detailed measurements of
the isotropic background observed by the \ac{LAT}.

The \galprop predictions for the $\gamma$-ray background are not
accurate enough for the analysis of point-like and $\sim1\degree$
large extended sources.  Therefore, an improved data-driven model
of the Galactic diffuse background has been devised where components
of the \galprop model are fit to the observed $\gamma$-ray emission.
This data-driven model is described in \cite{nolan_2012_fermi-large}.


\subsection{\Actitle{2FGL}}
\subseclabel{2fgl}

Using 2 years of observations, the \ac{LAT} collaboration produced a list
of 1873 $\gamma$-ray-emitting sources detected in the $100\unitspace\mev$
to $100\unitspace\gev$ energy range \citep{nolan_2012_fermi-large}.
Primarily, the catalog assumed sources to be point like. But twelve
previously-published sources were included as being spatially extended
with the spatial model taken from prior publications.

Of these 1873 sources, 127 were firmly identified with a multiwavelength
counterpart.  A source is only firmly identified if it meets one of
three criteria. First, it can have periodic variability (pulsars and
high-mass binaries).  Second, it could have a matching spatial morphology
(\acp{SNR} and \acp{PWN}). Finally, it could have correlated variability
(\ac{AGN}).  In total, \ac{2FGL} firmly identified 83 pulsars, 28
\acp{AGN}, 6 \acp{SNR}, 4 \acp{HMB}, 3 \acp{PWN}, 2 normal galaxies,
and one nova \cite{nolan_2012_fermi-large}.

In addition, 1171 sources are included in the looser criteria that
they were potentially associated with a multiwavelength counterpart.
Using this criteria, 86 sources are associated with pulsars, 25 with
\acp{PWN},  98 with \acp{SNR}, and 162 were flagged as being potentially
spurious due to residuals included by incorrectly modeling the galactic
diffuse emission.

\subsection{\Actitle{2PC}}
\subseclabel{2pc}

Using 3 years of data, the \ac{LAT} collaboration produced \acf{2PC},
a list of 117 pulsars significantly detected by the \ac{LAT}
\citep{abdo_2013a_second-fermi}.  Typically, a \ac{LAT}-detected pulsar is
first detected at either radio or X-ray energies.  This method was used
to discover 61 of the $\gamma$-ray emitting pulsars.  But some pulsars
are known to emit only $\gamma$-rays.  These sources can be searched
for blindly using $\gamma$-ray data.  This method was used to detect
36 pulsars.  Finally, in the third method, the positions of unidentified
\ac{LAT} sources which could potentially be associated with pulsars.
These regions are often searched for in radio to look for pulsar
emission. This method has lead to the detection of 20 new \acp{MSP}.
In total \ac{2PC} detected 42 radio-loud pulsars, 35 radio-quiet pulsars,
and 40 $\gamma$-ray \acp{MSP}.

\subsection{\Acptitle{PWN} Detected by \Actitle{LAT}}
\subseclabel{pwn_detected_lat}

In addition to detecting over 100 pulsars, the \ac{LAT} has detected
several \acp{PWN}.  In situations where the \acp{PWN} has an associated
\ac{LAT}-detected pulsar, typically the spectral analysis of the
\ac{PWN} is performed during times in the pulsar phase where the
pulsar emission is at a minimum. For some pulsars, such as \hessj{1825},
there is no associated \ac{LAT}-detected pulsar and the spectral analysis
can be performed without cutting on pulsar phase.

\subsubsection{Crab}

\begin{figure}[htbp]
  \centering
    \includegraphics{chapters/introduction/figures/crab_spectrum.pdf}
  \caption{
    The \ac{SED} of the Crab nebula observed by the \ac{LAT} as
    well as several other instruments.
    This figure is from \cite{abdo_2010a_fermi-large}.
  }
  \figlabel{crab_spectrum}
\end{figure}

Observations of the Crab nebula by the
\ac{LAT} provided detailed spectral resolution of Crab's spectrum
\cite{abdo_2010a_fermi-large}.  The Crab nebula shows a very strong
spectral break in the \ac{LAT} energy band, and the $\gamma$-ray emission
is interpreted as being the combination of a synchrotron component at low
energy and an \ac{IC} component at high energy.

In addition, $\gamma$-ray emission from the Crab nebula has
been observed to be variability in time and have flaring periods
\citep{abdo_2011a_gamma-ray-flares}.  The Crab was observed to have
an extreme flare in 2011 \citep{buehler_2012a_gamma-ray-activity}.
This variability is challenging to understand given conventional models
of \ac{PWN} emission.

\subsubsection{\velax}

\begin{figure}[htbp]
  \centering
    \includegraphics{chapters/introduction/figures/vela_x_sed_two_populations.pdf}
    \caption{The \ac{SED} of \velax observed at radio, x-ray,
      $\gamma$-ray, and \ac{VHE} energies. The emission was suggested
      by \citep{abdo_2010c_fermi-large} to be driven by two pollutions
      of electrons.  In this model, the lower-energy electron population
      powers the radio and $\gamma$-ray emission adn the higher-energy
      electron population powers the x-ray and \ac{VHE} emission.
      This figure is from \cite{abdo_2010c_fermi-large}.}
  \figlabel{vela_x_sed_two_populations}
\end{figure}

\velax is a \ac{PWN} powered by the Vela pulsar.  It was first observed
by \cite{rishbeth_1958a_radio-emission}.  It was observed at \ac{VHE}
energies by \cite{aharonian_2006a_first-detection} and at \gev
energies by \ac{AGILE} \citep{pellizzoni_2010a_detection-gamma-ray}.
The detailed multiwavelength spectra of \velax is plotted in
\figref{vela_x_sed_two_populations}.  Based upon the morphological
and spectral disconnect between the \gev and \tev emission,
\citep{abdo_2010c_fermi-large} argued that emission was not consistent
with a single population of accelerated electrons.  They suggested
instead that the emission comes instead from two populations of electrons.

\subsubsection{\mshfifteenfiftytwo}

\ac{SNR}
\citep[\mshfifteenfiftytwo][]{caswell_1981a_high-resolution-radio} is
commonly associated with \psrb{1509} \citep{seward_1982a_x-ray-pulsar}.
A diffuse nebula was observed surrounding the pulsar
\citep{seward_1982a_x-ray-pulsar}, adn interpreted as an \ac{PWN}
\cite{trussoni_1996a_rosat-observations}.  The \ac{PWN} was detected at
\ac{VHE} energies by \cite{aharonian_2005a_discovery-extended} and at
\gev energies by \cite{abdo_2010a_detection-energetic}

\subsubsection{\hessj{1825}}

\hessj{1825} is an extended ($\sim0\fdg5$) \ac{VHE} sources
first detected during the \ac{HESS} survey of the inner
galaxy \citep{aharonian_2006a_h.e.s.s.-survey}.  It was
interpreted by \cite{aharonian_2005a_possible-association}
as being a \ac{PWN} powered by \psrj{1826} \citep[also known
as \psrb{1823},][]{clifton_1992a_high-frequency-survey}.
Surrounding the pulsar is a diffuse $\sim 5'$ nebula
\citep{finley_1996a_morphology-young}.  The large size difference
can be understood in terms of the different lifetimes for
the synchrotron-emitting and \ac{IC}-emitting electrons
\citep{aharonian_2006a_h.e.s.s.-survey}.

This source was subsequently detected by
\cite{grondin_2011a_detection-pulsar} at \gev energies.  Interestingly,
the \ac{VHE} emission from \hessj{1825} was observed to have an
energy-dependent morphology, with the size decreasing with increasing
energy \citep{aharonian_2006a_energy-dependent}.  This can be explained
by the \ac{IC} emission model if the electron injection decreases
with time.

\subsubsection{\hessj{1640}}

The \ac{VHE} source \hessj{1640} 
\citep{aharonian_2006a_h.e.s.s.-survey}
is spatially-coincident with \snrg{338.3}
\citep{shaver_1970a_galactic-radio}.  X-ray observations by
\xmmnewton uncovered a spatially-coincident X-ray nebula and within it
a point-like source \cite{funk_2007a_xmm-newton-observations}.
This point-like source is believed to be a neutron star powering the
\ac{PWN}, but pulsations have not yet been detected from it.
\cite{slane_2010_fermi-detection} discovered an associated \gev source.

\subsubsection{\hessj{1857}}

\hessj{1857} was also discovered by \ac{HESS}
\citep{aharonian_2008a_very-high-energy-gamma-ray}
\cite{hessels_2008a_j18560245:-arecibo} suggested that \hessj{1857}
is a \ac{PWN} powered by \psrj{1856}.  \hessj{1857} was also detected
by the \ac{LAT}.

\subsubsection{\hessj{1023}}

The \ac{VHE} source \hessj{1023} was discovered in the region of the young
stellar cluster Westerlund 2 \cite{aharonian_2007a_detection-extended}.
This same source was subsequently detected by the \ac{LAT} in the off-peak
region surrounding \psrj{1023} \citep{ackermann_2011a_fermi-lat-search}.
\cite{h.e.s.s.collaboration_2011a_revisiting-westerlund} proposed that
the emission could either be due to an \acp{PWN} or due to hadronic
interactions of cosmic rays accelerated in the open stellar cluster
interacting with molecular clouds.
