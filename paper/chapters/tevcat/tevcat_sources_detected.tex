\section{Sources Detected} 
\seclabel{tevcat_sources_detected}


\tabletypesize{\scriptsize}
\begin{deluxetable}{lllllrrll}
\tablewidth{0pt}
\tablecolumns{9}
\tablecaption{Spatial results 
\label{tab:Spat_results}}

\tablehead{ \colhead{Name} & \colhead{ID} & \colhead{$\text{TS}_{\text{TeV}}$} & \colhead{$\text{TS}_{\text{GeV}}$} & \colhead{$\text{TS}_{\text{ext}}$} & \colhead{GLON} & \colhead{GLAT} & \colhead{POSERR} & \colhead{$\sigma$}\\ 
 \colhead{} & \colhead{} & \colhead{} & \colhead{} & \colhead{} & \colhead{(deg)} & \colhead{(deg)} & \colhead{(deg)}}
\startdata
VER~J0006+727 & PSR & 655 & 1206 & 0 & $119.68$ & $10.47$ & $0.01 \pm 0.01$ & $< 0.07$ \\ 
\tablenotemark{a} & & 2 & - & - & - & - & - \\
MGRO~J0632+17 & PSR  & 699 & 2056 & 1 & $195.13$ & $4.28$ & $0.01 \pm 0.01$ & $< 0.08$ \\ 
\tablenotemark{a} & & 5 & - & - & - & - & - \\
HESS~J1018-589 & O  & 29 & 29 & 0 & $284.33$ & $-1.66$ & $0.04 \pm 0.02$ & $< 0.87$ \\ 
\tablenotemark{a} &   & 25 & 25 & 2 & $284.34$ & $-1.65$ & $0.04 \pm 0.02$ & $< 0.87$ \\
HESS~J1023-575 & PWNc  & 52 & 58 & 8 & $284.13$ & $-0.45$ & $0.03 \pm 0.02$ & $< 0.77$ \\ 
\tablenotemark{a} &   & 52 & 58 & 8 & $284.13$ & $-0.45$ & $0.03 \pm 0.02$ & $< 0.77$ \\
HESS~J1119-614 & PWNc  & 27 & 27 & 9 & $292.16$ & $-0.56$ & $ 0.05 \pm 0.02 $ & $< 0.31$ \\
\tablenotemark{a} &  & 16 & 16 & 9 & $292.18$ & $-0.57$ & $0.05 \pm 0.02$ & $< 0.32$ \\ 
HESS~J1303-631 & PWNc & 37 & 58 & 29 & $304.56$ & $-0.11$ & $0.04 \pm 0.03$ & $0.45 \pm 0.09 \pm 0.10$\\
HESS~J1356-645 & PWN  & 24 & 26 & 3 & $309.71$ & $-2.33$ & $0.05 \pm 0.01$ & $< 0.39$ \\ 
\tablenotemark{a} &   & 24 & 26 & 3 & $309.71$ & $-2.32$ & $0.05 \pm 0.01$ & $< 0.39$ \\
HESS~J1418-609 & PSR  & 31 & 32 & 0 & $313.28$ & $0.13$ & $0.03 \pm 0.01$ & $< 0.32$ \\ 
\tablenotemark{a} & & 15 & - & - & - & - & - \\
HESS~J1420-607 & PWNc  & 42 & 42 & 0 & $313.55$ & $0.27$ & $0.04 \pm 0.02$ & $< 0.38$ \\
\tablenotemark{a} & & 36 & 36 & 0 & $313.55$ & $0.28$ & $0.04 \pm 0.02$ & $< 0.39$ \\ 
HESS~J1507-622 & O  & 21 & 23 & 7 & $317.77$ & $-3.60$ & $0.05 \pm 0.03$ & $< 1.04$ \\ 
\tablenotemark{a} & & 21 & 23 & 7 & $317.76$ & $-3.61$ & $0.05 \pm 0.03$ & $< 1.04$ \\
HESS~J1514-591 & PWN  & 156 & 147 & 10 & $320.35$ & $-1.25$ & $0.03 \pm 0.01$ & $< 0.16$ \\ 
\tablenotemark{a} & & 156 & 147 & 10 & $320.35$ & $-1.25$ & $0.03 \pm 0.01$ & $< 0.16$ \\
HESS~J1614-518 & O  & 110 & 133 & 47 & $331.62$ & $-0.74$ & $0.04 \pm 0.03$ & $0.28 \pm 0.03 \pm 0.05$\\
HESS~J1616-508 & PWNc  & 75 & 94 & 31 & $332.39$ & $-0.27$ & $0.04 \pm 0.02$ & $0.25 \pm 0.03 \pm 0.05$\\
HESS~J1632-478 & PWNc  & 137 & 161 & 56 & $336.50$ & $0.10$ & $0.03 \pm 0.02$ & $0.30 \pm 0.06 \pm 0.06$\\
HESS~J1634-472 & O  & 33 & 34 & 1 & $337.23$ & $0.35$ & $0.03 \pm 0.01$ &$< 1.21$ \\ 
HESS~J1640-465 & PWNc  & 47 & 42 & 9 & $338.33$ & $0.05$ & $0.05 \pm 0.01$ & $< 1.17$ \\ 
HESS~J1708-443 & PSR  & 722 & 1153 & 0 & $343.11$ & $-2.70$ & $0.01 \pm 0.01$ & $< 0.05$ \\
\tablenotemark{a} & & 33 & 64 & 0 & $343.12$ & $-2.70$ & $0.01 \pm 0.01$ & $< 0.09$ \\ 
HESS~J1804-216 & O  & 138 & 141 & 37 & $8.40$ & $-0.09$ & $0.04 \pm 0.01$ & $0.25 \pm 0.03 \pm 0.04$\\
\tablenotemark{a} & & 124 & 128 & 30 & $8.42$ & $-0.10$ & $0.04 \pm 0.01$ & $0.24 \pm 0.03 \pm 0.04$\\
HESS~J1825-137 & PWN  & 56 & 82 & 30 & $17.55$ & $-0.47$ & $0.05 \pm 0.03$ & $0.44 \pm 0.08 \pm 0.09$\\
HESS~J1834-087 & O  & 27 & 36 & 4 & $23.20$ & $-0.26$ & $0.05 \pm 0.01$ & $< 0.22$ \\ 
HESS~J1837-069 & PWNc  & 73 & 119 & 46 & $25.17$ & $0.00$  & $0.05 \pm 0.03$ & $0.36 \pm 0.06 \pm 0.04$\\
HESS~J1841-055 & PWNc  & 64 & 70 & 32 & $27.01$ & $-0.15$ & $0.05 \pm 0.03$ & $0.38 \pm 0.06 \pm 0.06$\\
HESS~J1848-018 & PWNc  & 19 & 19 & 0 & $30.90$ & $-0.20$ & $0.04 \pm 0.01$ & $< 1.50$ \\ 
HESS~J1857+026 & PWNc  & 53 & 55 & 8 & $36.08$ & $0.02$ & $0.04 \pm 0.01$ & $< 0.28$ \\ 
MGRO~J1908+06 & PSR  & 16 & 37 & 2 & $40.11$ & $-0.84$ & $0.03 \pm 0.01$ &$< 0.19$ \\ 
\tablenotemark{a} & & 9 & - & - & - & - & - & - \\
MGRO~J1958+2848 & PSR  & 21 & 24 & 0 & $65.88$ & $-0.34$ & $0.04 \pm 0.01$ & $< 0.56$ \\ 
\tablenotemark{a} & & 8 & - & - & - & - & - \\
VER~J2016+372 & O  & 31 & 33 & 1 & $74.86$ & $1.22$ & $0.05 \pm 0.02$ & $< 1.16$ \\ 
MGRO~J2019+37 & PSR & 31 & 100 & 1 & $75.23$ & $0.13$ & $0.02 \pm 0.01$ & $< 0.07$ \\ 
\tablenotemark{a} & & 5 & - & - & - & - & - \\
MGRO~J2031+41 B & PSR  & 58 & 66 & 5 & $80.20$ & $1.03$ & $0.05 \pm 0.01$ & $< 2.47$ \\ 
\tablenotemark{a} & & 12 & - & - & - & - & - \\
MGRO~J2228+61 & PSR & 94 & 113 & 0 & $106.65$ & $2.94$ & $0.02 \pm 0.01$ & $< 0.10$ \\ 
\tablenotemark{a} & & 15 & - & - & - & - & - \\
\enddata


\tablecomments{Results of the maximum likelihood spatial fits for LAT-detected TeV sources. ``a" in the first column corresponds to the results with contribution of the pulsar associated in Table~\ref{tab:pulsars} subtracted from the emission of the source just above. Column 2 lists the classification for the LAT emission: either ``PWN" for clearly identified PWNe, ``PWNc" for PWN candidates, ``PSR" for pulsar emission, and ``O" for anything else. Column 3 is the TS when the source is modeled with the spatial model obtained from TeV energies. Column 4 is the TS when the source is modeled assuming it is point-like, and column 5 is the TS of the source assuming it is spatially-extended with a Gaussian spatial model. Columns 6 and 7 give the position of the source fit in the GeV energy range. The method for determining systematic uncertainties on the spatial parameters are described in Section \ref{systext}. Column 8 gives the extension fit in the GeV energy range if $\text{TS}_{\text{ext}} > 16$ or a 99 \% c.l. upper limit on the extension otherwise.}
\end{deluxetable}



We detected 22 sources at \gev energies.  For significantly-detected
sources, we present the spatial and spectral results for these sources
in \tabref{tevcat_spatial_spectral}.  Flux upper limits for non-detected
sources as well as spectral points in three independent energy bins are
can be found in \cite{acero_2013a_constraints-galactic}.

We attempt to classify the \gev emission into four categories:
\PWNClass-type for sources where the \gev emission is clearly identified
as \aac{PWN}, \PWNcClass for sources where the \gev emission could potentially
be due to a \ac{PWN}, \PSRClass-type for sources where the emission is
most likely due to pulsed emission inside the pulsar's magnetosphere,
and \OtherClass-type (for other) when the true nature of emission is
uncertain.

We categorize a source as \PWNClass-type or \PWNcClass-type when the
emission has a hard spectrum which connects spectrally to the \ac{VHE}
spectrum and when there is some multiwavelength evidence that the
\gev and \ac{VHE} emission should be due to a \ac{PWN}.  We label
a source as \PWNClass-type when the \ac{VHE} emission suggests
more strongly that the emission is due to a \ac{PWN}.  We include
in \tabref{tevcat_spatial_spectral} the source classifications
for each source.  We will discuss the \PWNClass-type source
in \subsecref{tevcat_pwn_pwnc_type_sources}.  We label a source
as \PSRClass-type if the emission is soft, point-like, and strongly
effected by our inclusion of its associated \ac{2FGL} pulsar in the
background model.  We label a source as \OtherClass-type otherwise.

\subsection{\PWNClass-type and \PWNcClass-type Sources}
\subseclabel{tevcat_pwn_pwnc_type_sources}

In total, we detect fourteen sources which we classify as \PWNClass-type
or \PWNcClass-type. Five of these \ac{PWN} and \ac{PWN} candidates are
first reported in this analysis.

Of these fourteen sources, three are classified as \PWNClass-type.  They
are \hessj{1356}, \mshfifteenfiftytwo (\hessj{1514}), and \hessj{1825}.
\hessj{1356} and \mshfifteenfiftytwo are classified as \PWNClass-type
because of the correlation between the X-ray and \ac{VHE} emission
\citep{h.e.s.s.collaboration_2011a_discovery-source,aharonian_2005a_discovery-extended}.
\hessj{1825} is classified as \PWNClass-type because of
the energy-dependent morphology observed at \ac{VHE} energies
\citep{aharonian_2006a_energy-dependent}.  We note that \hessj{1356}
is first presented as a $\gamma$-ray emitting \ac{PWN} in this work.
Once we add the Crab Nebula and \velax (not analyzed in this work) to
this list, the total number of clearly-identified \acp{PWN} detected at
\gev energies is five.

In addition, we detect eleven \PWNcClass-type sources.  Four of these
sources are first reported in this work: \hessj{1119}, \hessj{1303},
\hessj{1420}, and \hessj{1841}.  These sources are all powered by
pulsars energetic enough to power the observed emission (\psrj{1119},
\psrj{1301}, \psrj{1420}, \psrj{1838}), and they all have a hard
spectrum which connects to the spectra observed at \ac{VHE} energies.  The
multiwavelength interpretation of the new \ac{PWN} and \ac{PWN} candidates
is discussed more thoroughly in \cite{acero_2013a_constraints-galactic}.

The remaining seven \PWNcClass-type sources have been previously
published: \hessj{1023} \citep{ackermann_2011a_fermi-lat-search},
\hessj{1640} \citep{slane_2010a_fermi-detection}, \hessj{1616}
\citep{lande_2012_search-spatially}, \hessj{1632}
\citep{lande_2012_search-spatially}, \hessj{1837}
\citep{lande_2012_search-spatially}, \hessj{1848}
\citep{tam_2010a_search-counterparts}, and \hessj{1857}
\citep{rousseau_2012a_fermi-lat-constraints}.  We classify \hessj{1848}
as \PWNcClass even though the \gev emission has a soft spectrum based
on the analysis from \cite{lemoine-goumard_2011a_fermi-lat-detection}.

We mention that three of these \PWNcClass-type sources have
\ac{LAT}-detected pulsars (\psrj{1119}, \psrj{1420}, and \psrj{1838}) and
therefore were also studied in \chapref{offpeak}.  In \chapref{offpeak},
\psrj{1119} has $\ts=61.3$ and is classified as a ``U''-type source
because the spectrum is relatively soft (spectral index $\sim2.2$).
The off-peak spectrum of this source shows both a low-energy component and
high-energy component, so most likely the off-peak emission is composted
of the pulsar at low energy and the \ac{PWN} at high energy.

For \psrj{1420}, the off-peak emission (at the position of the pulsar)
has $\tspoint=8.1$ which is significantly less then the emission observed
in the high-energy analysis (\tstev=36).  It is possible that for this
source, \tstev is overestimated due to undersubtracting the emission
of \psrj{1420}.

Finally, \psrj{1838} is significantly detected in the off-peak, but
as a soft and significantly-cutoff spectrum. This emission is also
spatially-extended and the best-fit extension incorporates both the
emission at the position of \psrj{1838} and also residual towards the
center of \hessj{1841}. Most likely, the off-peak emission of \psrj{1838}
includes both a magnetospheric component at the position of the pulsar
and \aac{PWN} component from \hessj{1841}.

\subsection{\OtherClass-type Sources}

We detected six \OtherClass-type sources.  Two of these sources
(\hessj{1634} and \hessj{1804}) have a hard spectrum which connects
spectrally to the \ac{VHE} emission but are not classified as \ac{PWN}
based upon multiwavelength considerations.  \hessj{1634} is not
a \ac{PWN} candidate because there are no pulsar counterparts able to
power it.  \hessj{1804} was suggested to be \snrg{8.7}
\citep[W30,][]{ajello_2012a_fermi-large}.

The remaining four \OtherClass-type sources have a soft spectrum
which does not connect with the \ac{VHE} emission: \hessj{1018},
\hessj{1507}, \hessj{1834}, and \verj{2016}.  \hessj{1018}
is in the region of the $\gamma$-ray binary \onefglj{1018.6}
\citep{the-fermi-lat-collaboration_2012a_periodic-emission} and also
\snrg{284.3}.  \gev emission from the region of \hessj{1507} is studied
in \cite{domainko_2012a_exploring-nature}.  \hessj{1834} and \verj{2016}
both lack pulsars energetic enough to power the observed emission.



\subsection{\PSRClass-type Sources}

In \citep{acero_2013a_constraints-galactic}, the
$\energy>10\unitspace\gev$ search for \ac{PWN} was performed both
with and without associated pulsars included in the background model.
When we did not include the \ac{LAT}-detected pulsars included in the
background model, we detected nine sources which were consistent with
magnetospheric emission.  After modeling the associated pulsars in the
background, only \hessj{1708} remained significant.  Even so, the source
was strongly influenced by the inclusion of the pulsar in the background
model, so we suspect that the emission primarily magnetospheric and that
our pulsar emission model underpredicts the true magnetospheric emission.


