\subsection{\PWNClass-type and \PWNcClass-type Sources}
\subseclabel{tevcat_pwn_pwnc_type_sources}

In total, we detect fourteen sources which we classify as \PWNClass-type
or \PWNcClass-type. Five of these \ac{PWN} and \ac{PWN} candidates are
first reported in this analysis.

Of these fourteen sources, three are classified as \PWNClass-type.  They
are \hessj{1356}, \mshfifteenfiftytwo (\hessj{1514}), and \hessj{1825}.
\hessj{1356} and \mshfifteenfiftytwo are classified as \PWNClass-type
because of the correlation between the X-ray and \ac{VHE} emission
\citep{h.e.s.s.collaboration_2011a_discovery-source,aharonian_2005a_discovery-extended}.
\hessj{1825} is classified as \PWNClass-type because of
the energy-dependent morphology observed at \ac{VHE} energies
\citep{aharonian_2006a_energy-dependent}.  We note that \hessj{1356}
is first presented as a $\gamma$-ray emitting \ac{PWN} in this work.
Once we add the Crab Nebula and \velax (not analyzed in this work) to
this list, the total number of clearly-identified \acp{PWN} detected at
\gev energies is five.

In addition, we detect eleven \PWNcClass-type sources.  Four of these
sources are first reported in this work: \hessj{1119}, \hessj{1303},
\hessj{1420}, and \hessj{1841}.  These sources are all powered by
pulsars energetic enough to power the observed emission (\psrj{1119},
\psrj{1301}, \psrj{1420}, \psrj{1838}), and they all have a hard
spectrum which connects to the spectra observed at \ac{VHE} energies.  The
multiwavelength interpretation of the new \ac{PWN} and \ac{PWN} candidates
is discussed more thoroughly in \cite{acero_2013a_constraints-galactic}.

The remaining seven \PWNcClass-type sources have been previously
published: \hessj{1023} \citep{ackermann_2011a_fermi-lat-search},
\hessj{1640} \citep{slane_2010a_fermi-detection}, \hessj{1616}
\citep{lande_2012_search-spatially}, \hessj{1632}
\citep{lande_2012_search-spatially}, \hessj{1837}
\citep{lande_2012_search-spatially}, \hessj{1848}
\citep{tam_2010a_search-counterparts}, and \hessj{1857}
\citep{rousseau_2012a_fermi-lat-constraints}.  We classify \hessj{1848}
as \PWNcClass even though the \gev emission has a soft spectrum based
on the analysis from \cite{lemoine-goumard_2011a_fermi-lat-detection}.

We mention that three of these \PWNcClass-type sources have
\ac{LAT}-detected pulsars (\psrj{1119}, \psrj{1420}, and \psrj{1838}) and
therefore were also studied in \chapref{offpeak}.  In \chapref{offpeak},
\psrj{1119} has $\ts=61.3$ and is classified as a ``U''-type source
because the spectrum is relatively soft (spectral index $\sim2.2$).
The off-peak spectrum of this source shows both a low-energy component and
high-energy component, so most likely the off-peak emission is composted
of the pulsar at low energy and the \ac{PWN} at high energy.

For \psrj{1420}, the off-peak emission (at the position of the pulsar)
has $\tspoint=8.1$ which is significantly less then the emission observed
in the high-energy analysis (\tstev=36).  It is possible that for this
source, \tstev is overestimated due to undersubtracting the emission
of \psrj{1420}.

Finally, \psrj{1838} is significantly detected in the off-peak, but
as a soft and significantly-cutoff spectrum. This emission is also
spatially-extended and the best-fit extension incorporates both the
emission at the position of \psrj{1838} and also residual towards the
center of \hessj{1841}. Most likely, the off-peak emission of \psrj{1838}
includes both a magnetospheric component at the position of the pulsar
and \aac{PWN} component from \hessj{1841}.
