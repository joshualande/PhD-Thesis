\section{List of \ac{VHE} \Acstitle{PWN} Candidates}
\seclabel{tevcat_list_vhe_pwn_candidates}

We took all sources detected at \ac{VHE} energies and potentially
associated with \acp{PWN} and performed a search at \gev energies for
$\gamma$-ray emission. As was seen in previous chapters, many \acp{PWN}
have been detected both at \gev and \ac{VHE} energies, we suspect that
we might find new $\gamma$-ray emitting \ac{PWN} by searching \ac{LAT}
data in the regions of \ac{VHE} \acp{PWN}.

In addition, there are several \acp{PWN} which have been detected at
\ac{VHE} energies which do not have associated $\gamma$-ray pulsars
(such as \hessj{1825} and \hessj{1837}).  We therefore suspected that
this search could find new $\gamma$-ray emitting \ac{PWN} which were
not previously discovered either in the off-peak search discussed in
\chapref{offpeak} or in another dedicated analyses.

We used \tevcat\footnote{\tevcat can be found at
\url{http://tevcat.uchicago.edu}.} to define our list of \ac{VHE}
sources.  \tevcat is a catalog of sources detected at \ac{VHE} energies by
\acp{IACT}. We selected all sources from this catalog where the emission
was classified as being due to a \ac{PWN}. In addition, we included
all \ac{UNID} sources within 5\degree of the galactic plane since they
could be due to a \ac{PWN}.  Finally, we included \hessj{1023} because,
although this source is classified as a massive star cluster in the
\tevcat, \cite{de-naurois_2013a_galactic-h.e.s.s.} suggested that the
emission could originate in a \ac{PWN}.  \tabref{TeV_sources} presents
a list of all sources included in our analysis

\begin{deluxetable}{llrrl}
\tabletypesize{\tiny}
\tablecaption{List of analyzed \tev sources
\label{tab:TeV_sources}}
\tablewidth{0pt}
\tablecolumns{5}

\tablehead{\colhead{Name} & \colhead{Class} & \colhead{$l$} & \colhead{$b$} & \colhead{Reference}\\
\colhead{} & \colhead{} & \colhead{(deg.)} & \colhead{(deg.)} & \colhead{}}
\startdata
  VER~J0006$+$727 &  PWN & 119.58 &   10.20 &     \cite{2011arXiv1111.2591M}\\
 MGRO~J0631$+$105 &  PWN & 201.30 &    0.51 &     \cite{2009ApJ...700L.127A}\\
  MGRO~J0632$+$17 &  PWN & 195.34 &    3.78 &    \cite{2009ApJ...700L.127A} \\
 HESS~J1018$-$589 & UNID & 284.23 & $-1.72$ &     \cite{2012AA...541A...5H} \\
 HESS~J1023$-$575 &  MSC & 284.22 & $-0.40$ &      \cite{2011AA...525A..46H}\\
 HESS~J1026$-$582 &  PWN & 284.80 & $-0.52$ &     \cite{2011AA...525A..46H} \\
 HESS~J1119$-$614 &  PWN & 292.10 & $-0.49$ &  Presentation\tablenotemark{a}\\
 HESS~J1303$-$631 &  PWN & 304.24 & $-0.36$ &      \cite{2005AA...439.1013A}\\
 HESS~J1356$-$645 &  PWN & 309.81 & $-2.49$ &      \cite{2011AA...533A.103H}\\
 HESS~J1418$-$609 &  PWN & 313.25 &    0.15 &      \cite{2006AA...456..245A}\\
 HESS~J1420$-$607 &  PWN & 313.56 &    0.27 &      \cite{2006AA...456..245A}\\
 HESS~J1427$-$608 & UNID & 314.41 & $-0.14$ &      \cite{2008AA...477..353A}\\
 HESS~J1458$-$608 &  PWN & 317.75 & $-1.70$ &     \cite{2012arXiv1205.0719D}\\
 HESS~J1503$-$582 & UNID & 319.62 &    0.29 &     \cite{2008AIPC.1085..281R}\\
 HESS~J1507$-$622 & UNID & 317.95 & $-3.49$ &      \cite{2011AA...525A..45H}\\
 HESS~J1514$-$591 &  PWN & 320.33 & $-1.19$ &      \cite{2005AA...435L..17A}\\
 HESS~J1554$-$550 &  PWN & 327.16 & $-1.07$ &     \cite{2012arXiv1201.0481A}\\
 HESS~J1616$-$508 &  PWN & 332.39 & $-0.14$ &     \cite{2006ApJ...636..777A}\\
 HESS~J1626$-$490 & UNID & 334.77 &    0.05 &      \cite{2008AA...477..353A}\\
 HESS~J1632$-$478 &  PWN & 336.38 &    0.19 &     \cite{2006ApJ...636..777A}\\
 HESS~J1634$-$472 & UNID & 337.11 &    0.22 &     \cite{2006ApJ...636..777A}\\
 HESS~J1640$-$465 &  PWN & 338.32 & $-0.02$ &     \cite{2006ApJ...636..777A}\\
 HESS~J1702$-$420 & UNID & 344.30 & $-0.18$ &     \cite{2006ApJ...636..777A}\\
 HESS~J1708$-$443 &  PWN & 343.06 & $-2.38$ &      \cite{2011AA...528A.143H}\\
 HESS~J1718$-$385 &  PWN & 348.83 & $-0.49$ &      \cite{2007AA...472..489A}\\
 HESS~J1729$-$345 & UNID & 353.44 & $-0.13$ &      \cite{2011AA...531A..81H}\\
 HESS~J1804$-$216 & UNID &   8.40 & $-0.03$ &    \cite{2006ApJ...636..777A} \\
 HESS~J1809$-$193 &  PWN &  11.18 & $-0.09$ &      \cite{2007AA...472..489A}\\
 HESS~J1813$-$178 &  PWN &  12.81 & $-0.03$ &     \cite{2006ApJ...636..777A}\\
 HESS~J1818$-$154 &  PWN &  15.41 &    0.17 &    \cite{2011arXiv1112.2901H} \\
 HESS~J1825$-$137 &  PWN &  17.71 & $-0.70$ &      \cite{2006AA...460..365A}\\
 HESS~J1831$-$098 &  PWN &  21.85 & $-0.11$ &     \cite{2011ICRC....7..243S}\\
 HESS~J1833$-$105 &  PWN &  21.51 & $-0.88$ &     \cite{2008ICRC....2..823D}\\
 HESS~J1834$-$087 & UNID &  23.24 & $-0.31$ &     \cite{2006ApJ...636..777A}\\
 HESS~J1837$-$069 & UNID &  25.18 & $-0.12$ &     \cite{2006ApJ...636..777A}\\
 HESS~J1841$-$055 & UNID &  26.80 & $-0.20$ &      \cite{2008AA...477..353A}\\
 HESS~J1843$-$033 & UNID &  29.30 &    0.51 &     \cite{2008ICRC....2..579H}\\
 MGRO~J1844$-$035 & UNID &  28.91 & $-0.02$ &     \cite{2009ApJ...700L.127A}\\
 HESS~J1846$-$029 &  PWN &  29.70 & $-0.24$ &     \cite{2008ICRC....2..823D}\\
 HESS~J1848$-$018 & UNID &  31.00 & $-0.16$ &     \cite{2008AIPC.1085..372C}\\
 HESS~J1849$-$000 &  PWN &  32.64 &    0.53 &     \cite{2008AIPC.1085..312T}\\
 HESS~J1857$+$026 & UNID &  35.96 & $-0.06$ &      \cite{2008AA...477..353A}\\
 HESS~J1858$+$020 & UNID &  35.58 & $-0.58$ &      \cite{2008AA...477..353A}\\
 MGRO~J1900$+$039 & UNID &  37.42 & $-0.11$ &     \cite{2009ApJ...700L.127A}\\
  MGRO~J1908$+$06 & UNID &  40.39 & $-0.79$ &      \cite{2009AA...499..723A}\\
 HESS~J1912$+$101 &  PWN &  44.39 & $-0.07$ &      \cite{2008AA...484..435A}\\
  VER~J1930$+$188 &  PWN &  54.10 &    0.26 &    \cite{2010ApJ...719L..69A} \\
MGRO~J1958$+$2848 &  PWN &  65.85 & $-0.23$ &     \cite{2009ApJ...700L.127A}\\
  VER~J1959$+$208 &  PSR &  59.20 & $-4.70$ &     \cite{2003ApJ...583..853H}\\
  VER~J2016$+$372 & UNID &  74.94 &    1.15 &     \cite{2011arXiv1110.4656A}\\
  MGRO~J2019$+$37 &  PWN &  75.00 &    0.39 &     \cite{2007ApJ...664L..91A}\\
 MGRO~J2031$+$41A & UNID &  79.53 &    0.64 &     \cite{2007ApJ...664L..91A}\\
 MGRO~J2031$+$41B & UNID &  80.25 &    1.07 &     \cite{2012ApJ...745L..22B}\\
  MGRO~J2228$+$61 &  PWN & 106.57 &    2.91 &     \cite{2009ApJ...700L.127A}\\
\enddata


\tablenotetext{a}{This source was presented at the "Supernova Remnants and Pulsar Wind
Nebulae in the Chandra Era", 2009. See \url{http://cxc.harvard.edu/cdo/snr09/pres/DjannatiAtai\_Arache\_v2.pdf}.}

\tablecomments{The \tev sources that we searched for using \ac{LAT} observations.
The classifications come form \tevcat and are \acs{PWN} for \acl{PWN}, \ac{UNID} for \acl{UNID}, and \acs{MSC} for \acl{MSC}.
We include \hess{J1023} because it is potentially \aac{PWN}.}
\end{deluxetable}
\clearpage


