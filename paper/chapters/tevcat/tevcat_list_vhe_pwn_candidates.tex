\section{List of \ac{VHE} \Acstitle{PWN} Candidates}
\seclabel{tevcat_list_vhe_pwn_candidates}

We took all sources detected at \ac{VHE} energies and potentially
associated with \acp{PWN} and performed a search at \gev energies for
$\gamma$-ray emission. As was seen in previous chapters, many \acp{PWN}
have been detected both at \gev and \ac{VHE} energies, we suspect that
we might find new $\gamma$-ray emitting \ac{PWN} by searching \ac{LAT}
data in the regions of \ac{VHE} \acp{PWN}.

In addition, there are several \acp{PWN} which have been detected at
\ac{VHE} energies which do not have associated $\gamma$-ray pulsars
(such as \hessj{1825} and \hessj{1837}).  We therefore suspected that
this search could find new $\gamma$-ray emitting \ac{PWN} which were
not previously discovered either in the off-peak search discussed in
\chapref{offpeak} or in another dedicated analyses.

We used \tevcat\footnote{\tevcat can be found at
\url{http://tevcat.uchicago.edu}.} to define our list of \ac{VHE}
sources.  \tevcat is a catalog of sources detected at \ac{VHE} energies by
\acp{IACT}. We selected all sources from this catalog where the emission
was classified as being due to a \ac{PWN}. In addition, we included
all \ac{UNID} sources within 5\degree of the galactic plane since they
could be due to a \ac{PWN}.  Finally, we included \hessj{1023} because,
although this source is classified as a massive star cluster in the
\tevcat, \cite{de-naurois_2013a_galactic-h.e.s.s.} suggested that the
emission could originate in a \ac{PWN}.  \tabref{TeV_sources} presents
a list of all sources included in our analysis

\begin{deluxetable}{llrrlll}
\tabletypesize{\scriptsize}
\tablecaption{List of analyzed \ac{VHE} sources
\label{tab:TeV_sources}}
\tablewidth{0pt}
\tablecolumns{7}

\tablehead{\colhead{Name} & \colhead{Class} & \colhead{$l$} & \colhead{$b$} & \colhead{Pulsar} & \colhead{\acs{2PC}} & \colhead{Reference}\\
\colhead{} & \colhead{} & \colhead{(deg.)} & \colhead{(deg.)} & \colhead{} & \colhead{} & \colhead{}}
\startdata
  VER~J0006$+$727 &  PWN & 119.58 &   10.20 &   PSR~J0007+7303 & Y &    \cite{mcarthur_2011a_observation-veritas} \\
 MGRO~J0631$+$105 &  PWN & 201.30 &    0.51 &   PSR~J0631+1036 & Y &    \cite{abdo_2009a_milagro-observations} \\
  MGRO~J0632$+$17 &  PWN & 195.34 &    3.78 &   PSR~J0633+1746 & Y &   \cite{abdo_2009a_milagro-observations}  \\
 HESS~J1018$-$589 & UNID & 284.23 & $-1.72$ & PSR~J1016$-$5857 & Y &    \cite{h.e.s.s.collaboration_2012a_discovery-emission}  \\
 HESS~J1023$-$575 &  MSC & 284.22 & $-0.40$ & PSR~J1023$-$5746 & Y &     \cite{h.e.s.s.collaboration_2011a_revisiting-westerlund} \\
 HESS~J1026$-$582 &  PWN & 284.80 & $-0.52$ & PSR~J1028$-$5819 & Y &    \cite{h.e.s.s.collaboration_2011a_revisiting-westerlund}  \\
 HESS~J1119$-$614 &  PWN & 292.10 & $-0.49$ & PSR~J1119$-$6127 & Y & Presentation\tablenotemark{a} \\
 HESS~J1303$-$631 &  PWN & 304.24 & $-0.36$ & PSR~J1301$-$6305 & N &     \cite{aharonian_2005a_serendipitous-discovery} \\
 HESS~J1356$-$645 &  PWN & 309.81 & $-2.49$ & PSR~J1357$-$6429 & Y &     \cite{h.e.s.s.collaboration_2011a_discovery-source} \\
 HESS~J1418$-$609 &  PWN & 313.25 &    0.15 & PSR~J1418$-$6058 & Y &     \cite{aharonian_2006a_discovery-wings} \\
 HESS~J1420$-$607 &  PWN & 313.56 &    0.27 & PSR~J1420$-$6048 & Y &     \cite{aharonian_2006a_discovery-wings} \\
 HESS~J1427$-$608 & UNID & 314.41 & $-0.14$ &          \nodata & N &     \cite{aharonian_2008a_very-high-energy-gamma-ray} \\
 HESS~J1458$-$608 &  PWN & 317.75 & $-1.70$ & PSR~J1459$-$6053 & Y &    \cite{de-los-reyes_2012a_newly-discovered} \\
 HESS~J1503$-$582 & UNID & 319.62 &    0.29 &          \nodata & N &    \cite{renaud_2008a_nature-j1503-582} \\
 HESS~J1507$-$622 & UNID & 317.95 & $-3.49$ &          \nodata & N &     \cite{h.e.s.s.collaboration_2011a_discovery-follow-up} \\
 HESS~J1514$-$591 &  PWN & 320.33 & $-1.19$ & PSR~J1513$-$5908 & Y &     \cite{aharonian_2005a_discovery-extended} \\
 HESS~J1554$-$550 &  PWN & 327.16 & $-1.07$ &          \nodata & N &    \cite{acero_2012a_detection-emission} \\
 HESS~J1616$-$508 &  PWN & 332.39 & $-0.14$ & PSR~J1617$-$5055 & N &    \cite{aharonian_2006a_h.e.s.s.-survey} \\
 HESS~J1626$-$490 & UNID & 334.77 &    0.05 &          \nodata & N &     \cite{aharonian_2008a_very-high-energy-gamma-ray} \\
 HESS~J1632$-$478 &  PWN & 336.38 &    0.19 &          \nodata & N &    \cite{aharonian_2006a_h.e.s.s.-survey} \\
 HESS~J1634$-$472 & UNID & 337.11 &    0.22 &          \nodata & N &    \cite{aharonian_2006a_h.e.s.s.-survey} \\
 HESS~J1640$-$465 &  PWN & 338.32 & $-0.02$ &          \nodata & N &    \cite{aharonian_2006a_h.e.s.s.-survey} \\
 HESS~J1702$-$420 & UNID & 344.30 & $-0.18$ & PSR~J1702$-$4128 & Y &    \cite{aharonian_2006a_h.e.s.s.-survey} \\
 HESS~J1708$-$443 &  PWN & 343.06 & $-2.38$ & PSR~J1709$-$4429 & Y &     \cite{h.e.s.s.collaboration_2011a_detection-very-high-energy} \\
 HESS~J1718$-$385 &  PWN & 348.83 & $-0.49$ & PSR~J1718$-$3825 & Y &     \cite{aharonian_2007a_discovery-candidate} \\
 HESS~J1729$-$345 & UNID & 353.44 & $-0.13$ &          \nodata & N &     \cite{h.e.s.s.collaboration_2011a_shell-type-morphology:} \\
 HESS~J1804$-$216 & UNID &   8.40 & $-0.03$ & PSR~J1803$-$2149 & Y &   \cite{aharonian_2006a_h.e.s.s.-survey}  \\
 HESS~J1809$-$193 &  PWN &  11.18 & $-0.09$ & PSR~J1809$-$1917 & N &     \cite{aharonian_2007a_discovery-candidate} \\
 HESS~J1813$-$178 &  PWN &  12.81 & $-0.03$ & PSR~J1813$-$1749 & N &    \cite{aharonian_2006a_h.e.s.s.-survey} \\
 HESS~J1818$-$154 &  PWN &  15.41 &    0.17 &          \nodata & N &   \cite{hofverberg_2011a_discovery-gamma-ray}  \\
 HESS~J1825$-$137 &  PWN &  17.71 & $-0.70$ & PSR~J1826$-$1334 & N &     \cite{aharonian_2006a_energy-dependent} \\
 HESS~J1831$-$098 &  PWN &  21.85 & $-0.11$ & PSR~J1831$-$0952 & N &    \cite{sheidaei_2011a_discovery-very-high-energy} \\
 HESS~J1833$-$105 &  PWN &  21.51 & $-0.88$ & PSR~J1833$-$1034 & Y &    \cite{djannati-atai_2008a_companions-lonely} \\
 HESS~J1834$-$087 & UNID &  23.24 & $-0.31$ &          \nodata & N &    \cite{aharonian_2006a_h.e.s.s.-survey} \\
 HESS~J1837$-$069 & UNID &  25.18 & $-0.12$ & PSR~J1836$-$0655 & N &    \cite{aharonian_2006a_h.e.s.s.-survey} \\
 HESS~J1841$-$055 & UNID &  26.80 & $-0.20$ & PSR~J1838$-$0537 & Y &     \cite{aharonian_2008a_very-high-energy-gamma-ray} \\
 HESS~J1843$-$033 & UNID &  29.30 &    0.51 &          \nodata & N &    \cite{hoppe_2008a_h.e.s.s.-survey} \\
 MGRO~J1844$-$035 & UNID &  28.91 & $-0.02$ &          \nodata & N &    \cite{abdo_2009a_milagro-observations} \\
 HESS~J1846$-$029 &  PWN &  29.70 & $-0.24$ & PSR~J1846$-$0258 & N &    \cite{djannati-atai_2008a_companions-lonely} \\
 HESS~J1848$-$018 & UNID &  31.00 & $-0.16$ &          \nodata & N &    \cite{chaves_2008a_j1848-018:-discovery} \\
 HESS~J1849$-$000 &  PWN &  32.64 &    0.53 &  PSR~J1849$-$001 & N &    \cite{terrier_2008a_discovery-pulsar} \\
 HESS~J1857$+$026 & UNID &  35.96 & $-0.06$ &   PSR~J1856+0245 & N &     \cite{aharonian_2008a_very-high-energy-gamma-ray} \\
 HESS~J1858$+$020 & UNID &  35.58 & $-0.58$ &          \nodata & N &     \cite{aharonian_2008a_very-high-energy-gamma-ray} \\
 MGRO~J1900$+$039 & UNID &  37.42 & $-0.11$ &          \nodata & N &    \cite{abdo_2009a_milagro-observations} \\
  MGRO~J1908$+$06 & UNID &  40.39 & $-0.79$ &   PSR~J1907+0602 & Y &     \cite{aharonian_2009a_detection-energy} \\
 HESS~J1912$+$101 &  PWN &  44.39 & $-0.07$ &   PSR~J1913+1011 & N &     \cite{aharonian_2008a_discovery-very-high-energy} \\
  VER~J1930$+$188 &  PWN &  54.10 &    0.26 &   PSR~J1930+1852 & N &   \cite{acciari_2010a_discovery-energy}  \\
MGRO~J1958$+$2848 &  PWN &  65.85 & $-0.23$ &   PSR~J1958+2846 & Y &    \cite{abdo_2009a_milagro-observations} \\
  VER~J1959$+$208 &  PSR &  59.20 & $-4.70$ &   PSR~J1959+2048 & Y &    \cite{hall_2003a_search-emissions} \\
  VER~J2016$+$372 & UNID &  74.94 &    1.15 &          \nodata & N &    \cite{aliu_2011a_observations-region} \\
  MGRO~J2019$+$37 &  PWN &  75.00 &    0.39 &   PSR~J2021+3651 & Y &    \cite{abdo_2007a_gamma-ray-sources} \\
 MGRO~J2031$+$41A & UNID &  79.53 &    0.64 &          \nodata & N &    \cite{abdo_2007a_gamma-ray-sources} \\
 MGRO~J2031$+$41B & UNID &  80.25 &    1.07 &   PSR~J2032+4127 & Y &    \cite{bartoli_2012a_observation-gamma} \\
  MGRO~J2228$+$61 &  PWN & 106.57 &    2.91 &   PSR~J2229+6114 & Y &    \cite{abdo_2009a_milagro-observations} \\
\enddata

\tablenotetext{a}{This source was presented at the "Supernova Remnants and Pulsar Wind
Nebulae in the Chandra Era", 2009. See \url{http://cxc.harvard.edu/cdo/snr09/pres/DjannatiAtai\_Arache\_v2.pdf}.}

\tablecomments{The \ac{VHE} sources that we searched for using \ac{LAT}
observations.  The classifications come from \tevcat and are \acs{PWN}
for \acl{PWN}, \ac{UNID} for \acl{UNID}, and \acs{MSC} for \acl{MSC}.
We include \hess{J1023} because it is potentially \aac{PWN}.  
For sources with an associated pulsar, column 4 includes the pulsar's name.
Column 5 describes if the pulsar has been detected by the LAT and 
included in \ac{2PC} (See \chapref{offpeak}).
}
\end{deluxetable}
\clearpage


