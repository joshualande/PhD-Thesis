\section{Off-peak Analysis Method}
\seclabel{off_peak_analysis}

Characterizing both the spatial and spectral characteristics of any
off-peak emission helps discern its origin.  We employ a somewhat
different analysis procedure here than for the phase-averaged analysis
described in \citep{abdo_2013a_second-fermi}.  To evaluate the spatial
characteristics of any off-peak emission we use the likelihood fitting
package \pointlike \citep[detailed in][]{lande_2012_search-spatially}, and
to fit the spectrum we use \gtlike in binned mode via {\it pyLikelihood}
as was done for the phase-averaged analysis.

For each pulsar we start from the same temporal and spatial event
selections described in \citep{abdo_2013a_second-fermi} but we
increase the maximum energy to 400 GeV (the highest event energy for
any ROI under this selection is $\sim$316 GeV).  For the \pointlike
analysis we further select a 10$\degree$ radius ROI and for \gtlike
a $14\degree\times14\degree$ square ROI, both centered on the pulsar
position.  Finally, we only consider photons with pulse phases within
the corresponding off-peak interval.

We search for off-peak emission assuming a point source and (except for
the Crab Nebula and Vela-X, described below) a power-law spectral model.
We fit the position of this putative off-peak source using \pointlike
as described by \citet{nolan_2012_fermi-large} and then use the best-fit
position in a spectral analysis with \gtlike.  From the spectral analysis
we require $TS\geq25$ (just over $4\sigma$) to claim a detection.
If $TS<25$, we compute upper limits on the flux in the energy range
from 100 \mev to 316 \gev assuming a power law with photon index fixed
to 2.0 and a PLEC1 model with $\Gamma=1.7$ and $E_{\rm cut}=3$ GeV.

The spectrum of the Crab Nebula (associated with PSR J0534+2200)
is uniquely challenging because the GeV spectrum contains
both a falling synchrotron and a rising inverse Compton
component \citep{abdo_2010a_fermi-large}.  For this particular
source we used the best-fit two-component spectral model from
\cite{buehler_2012a_gamma-ray-activity} and fit only the overall
normalization of the source.  In addition, for Vela-X (associated
with PSR J0835-4510) we took the best-fit spectral model from
\cite{grondin_2013a_vela-x-pulsar} and fit only the overall normalization
of this source. This spectrum has a smoothly broken power law spectral
model and was fit assuming Vela-X to have an elliptical disk spatial
model.

If the off-peak source is significant, we test whether
the spectrum shows evidence for a cutoff, as described in
\citet{ackermann_2011a_fermi-lat-search}, assuming the source is at the
pulsar position.  We say that the off-peak emission shows evidence for
a cutoff if $TS_{\rm cut}\geq9$, corresponding to a $3\sigma$ detection.

For a significant off-peak point source, we use \pointlike to test if the
emission is significantly extended.  We assume a radially-symmetric
Gaussian source and fit the position and extension parameter
($\sigma$) as described in \citet{lande_2012_search-spatially}.
The best-fit extended source parameters are then given to \gtlike,
which is used to fit the spectral parameters and the significance of the
extension over a point source, $TS_{\rm ext}$, evaluated as described in
\citet{lande_2012_search-spatially}.  That paper established that $TS_{\rm
ext}\geq16$ means highly probable source extension.  In the present work
we aim only to flag possible extension, and use  $TS_{\rm ext}\geq9$.

To test for variability, even without significant emission over the
3-year time range, we divide the dataset into 36 intervals and fit the
point-source flux independently in each interval, computing $TS_{\rm
var}$ as in 2FGL.  For sources with potential magnetospheric off-peak
emission and for regions with no detection, we performed the test at
the pulsar's position.  Otherwise, we test at the best-fit position.
The off-peak emission is said to show evidence for variability if
$TS_{\rm var}\geq91.7$, corresponding to a $4\sigma$ significance.  %
As noted in \citep{abdo_2013a_second-fermi}, our timing solutions for
PSRs J0205+6449 and J1838$-$0537 are not coherent across all three years.
For these two pulsars, we excluded the time ranges without ephemerides and
only tested for variability during months that were completely covered.
For J1838$-$0537 only one month is lost, whereas for J0205+6449 the 7\%
data loss is spread across three separate months.  As a result, $TS_{\rm
var}$ for these pulsars is a conservative estimate of variability
significance.

The procedure described above, especially the extension analysis, is
particularly sensitive to sources not included in 2FGL that are near the
pulsar of interest, for two reasons.  First, we are using an additional
year of data and second, when ``turning off'' a bright pulsar nearby,
faint sources become more important to the global fit.  Therefore, in
many situations we had to run the analysis several times, iteratively
improving the model by including new sources, until we removed all $TS>25$
residuals. The final \gtlike-formatted XML source model for each off-peak
region is included in the auxiliary material.

There are still, however, pulsars for which we were unable to obtain an
unbiased fit of the off-peak emission, most likely due to inaccuracies in
the model of the Galactic diffuse emission and incorrectly modeled
nearby sources.  The most common symptom of a biased fit is an
unphysically large extension.  In these cases, the extended source
attempts to account for multiple point sources or incorrectly-modeled
diffuse emission, not just the putative off-peak emission.  Systematics
associated with modeling extended sources are discussed more thoroughly
in \citet{lande_2012_search-spatially}.  For the purposes of this catalog,
we have flagged the pulsars where off-peak analysis suffered from these
issues and do not attempt a complete understanding of the emission.

Observations of magnetospheric off-peak emission can be used to constrain
pulsar geometry. Therefore, it is important to know if off-peak emission
that is otherwise pulsar-like might instead be incorrectly-modeled
Galactic diffuse emission.  We therefore performed a limited study of the
systematics associated with our model of the Galactic diffuse emission.

For sources which otherwise would be classified as magnetospheric,
we tested the significance of the emission assuming eight
different Galactic diffuse emission models as described in
\citet{de-palma_2013a_method-exploring}.  These models were constructed
using a different model building strategy, vary parameters of the Galactic
diffuse emission model, and have additional degrees of freedom in the fit.

We define $\tsaltdiff$ as the minimum test statistic of any of
the eight alternate diffuse models. This test statistic is computed
assuming the emission to be pointlike at the best-fit position and is
therefore comparable to \tspoint.  For sources which would otherwise be
considered magnetospheric, we flag the emission as potentially spurious
if $\tsaltdiff<25$.  We caution that although this test can help to
flag problematic regions, these models do not probe the entirety of the
uncertainty associated with our model of the Galactic diffuse emission.
Therefore, some diffuse emission could still be incorrectly classified
as magnetospheric.

