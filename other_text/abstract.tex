\prefacesection{Abstract}

Pulsars are rapidly-rotating neutron stars born out of the dealth
of stars.  A diffuse nebula is formed when particle stream from these
neutron stars and interact with the ambient medium. These \ac{PWN}
are visible across the electromagnetic spectrum,  producing some of
the most brilliant objects ever witnessed.  The launch of the \fermi
Gamma-ray space telescope in 2008 has offered us an unprecedented view
of the cosmic $\gamma$-ray sky.  Using data from the \acrlong{LAT} on board
the \fermi telescope, we search for new $\gamma$-ray-emitting \ac{PWN}.
With these new observations, we vastly expand the number \ac{PWN} observed
at these energies. We interpret the observed $\gamma$-ray emission from
these \ac{PWN} in terms of a model where accelerated electrons radiate
synchrotron photons when they interact with the magnetic field of the
pulsar and inverse Compton photons when they interact with interstellar
photon fields.  We conclude by studying how the observed \ac{PWN}
emission evolves with the age and spin-down energy of the host pulsar.
