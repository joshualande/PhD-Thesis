\prefacesection{Abstract}

Pulsars are rapidly-rotating neutron stars born out of the death of
stars.  A diffuse nebula is formed when particles stream from these
neutron stars and interact with the ambient medium. These \ac{PWN}
are visible across the electromagnetic spectrum,  producing some of the
most brilliant objects ever observed.  The launch of the \fermi Gamma-ray
Space Telescope in 2008 has offered us an unprecedented view of the cosmic
$\gamma$-ray sky.  Using data from the \acrlong{LAT} on board the \fermi
telescope, we search for new $\gamma$-ray-emitting \ac{PWN}.  With these
new observations, we vastly expand the number of \ac{PWN} observed
at these energies. We interpret the observed $\gamma$-ray emission
from these \ac{PWN} in terms of a model where accelerated electrons
produce $\gamma$-rays through inverse Compton upscattering when they
interact with interstellar photon fields.  We conclude by studying how
the observed \ac{PWN} evolve with the age and spin-down power
of the host pulsar.
