\section{Off-Peak Individual Source Discussion}
\label{off_peak_individual_source_discussion}

Here we discuss several interesting sources found in the off-peak
analysis presented in Section~\ref{unpulsed}.

% J0007+7303
The off-peak emission from PSR J0007+7303 in the SNR CTA1 was previously studied by \cite{Fermi_CTA1_2012}.
They found a soft and not-significantly cut off source in the off-peak region that
is marginally extended.
We find a similar spectrum and extension significance ($TS_{\rm ext}=10.8$), and therefore classify this
source as type `U'.


% J0205+6449
The new type `W' source is associated with PSR J0205+6449 \citep{LATPSR0205}.
The off-peak spectrum for this source is shown in panel b of Figure
\ref{off_peak_seds}.  The emission is best fit as a point source at
$(l,b)=(130\fdg73,3\fdg11)$ with a 95\% confidence-level radius of $0\fdg03$.
The source has a hard spectrum (power law with $\Gamma=1.61\pm0.21$)
and is therefore consistent with a PWN hypothesis.
This nebula has been observed at infrared
\citep{3C58_IR_PWN} and X-ray \citep{3C58_Xray_PWN} energies. This
suggests that we could be observing the inverse Compton emission from
the same electrons powering synchrotron emission at lower energies.
The PWN hypothesis is supported by the associated pulsar's
very high $\dot E=2.6\times10^{36}$ erg s$^{-1}$ and relatively young
characteristic age, $\tau_c = 5400$ yr. This is consistent with the properties
of other pulsars with LAT-detected PWN, and we favor a PWN
interpretation.
We note that the discrepancy between our spectrum and the upper limit
quoted in \citet{LAT_collaboration_PWNCAT_2011} is mainly caused by
our expanded energy range and because the flux upper limit was computed
assuming a different spectral index.


However, we note that PSR J0205+6449 is associated to the SNR 3C58
(G130.7+3.1).  Given the 2 kpc distance estimate from Section \ref{Distances}
and the density of thermal material estimated by \cite{3C58_Xray_PWN}, we can estimate
the energetics required for the LAT emission to originate in the SNR.
Following the prescription in \cite{Gamma_Visibility_SNRs_Drury_1994},
we assume the LAT emission to be hadronic and estimate a cosmic-ray
efficiency for the SNR of $\sim10$\%, which is energetically allowed.
We therefore cannot rule out the SNR hypothesis.

No TeV detection of this source has been reported, but given the hard photon
index at GeV energies this is a good candidate for observations by
an atmospheric Cherenkov telescope. Improved spectral and spatial observations 
at TeV energies might help to uniquely classify the emission.

We obtain a flux for Vela-X which is $\sim10\%$ larger than the flux
obtained in \cite{FermiVelaX2nd}. This discrepancy is most-likey due to
assuming a different spatial model for the emission (radially-symmetric
Gaussian compared to elliptical Gaussian).

% J1023-575
PSR J1023$-$5746 is associated with the TeV PWN HESS J1023$-$575
\citep{HESS_J1023-575_HESS_Collaboration_2007}.  LAT emission from
this PWN was first reported in \citet{LAT_collaboration_PWNCAT_2011}.
Because of the dominant low-energy magnetospheric emission, we classify
this as type `M' and not as a PWN.
A phase-averaged analysis of this source for energies above 10 GeV is
reported in \citet{Rousseau2013}. 


% J1119-6127
PSR J1119$-$6127 \citep{FermiMagnetars} is associated with the TeV source HESS
J1119$-$614\footnote{The discovery of HESS J1119$-$614 was presented at the
``Supernova Remnants and Pulsar Wind Nebulae in the Chandra Era'' in
2009. See \url{http://cxc.harvard.edu/cdo/snr09/pres/DjannatiAtai\_Arache\_
v2.pdf}.}. Our off-peak analysis 
classifies this source as `U' because its spectrum is soft and not significantly
cut off. However, the SED appears to represent a cutoff spectrum at low
energy and a hard rising spectrum at high energy.  
\citet{Rousseau2013} significantly detect this PWN using the analysis procedure as described for J1023$-$575.
We are likely detecting a composite of magnetospheric emission at low energy and pulsar-wind emission
at high energy.


% J1357-6429
PSR J1357$-$6429 \citep{LAT_J1357} has an associated PWN HESS J1356$-$645 detected at
TeV energies \citep{HESS_HESS_J1356-645_2011}.  Our analysis of the off-peak regions surrounding PSR
J1357$-$6429 shows a source positionally and spectrally consistent with
HESS J1356$-$645, but with significance just below detection threshold ($TS=21.0$).  
\citet{Rousseau2013} present significant emission from this source.

% J1410-6132
The off-peak region of PSR J1410$-$6132 \citep{OBrien_2008} shows a relatively hard
spectral index of $1.90\pm0.15$, and the spectrum is not significantly
cut off.  There is no associated TeV PWN
and enough low-energy GeV emission is present to caution against a clear
PWN interpretation.  We classify this source as `U', but
further observations could reveal interesting emission.

% J2021+4026
PSR J2021+4026 is spatially coincident with the
LAT-detected and spatially extended Gamma Cygni SNR
\citep{LAT_collaboration_extended_search_2012}.  The off-peak emission
from this pulsar is consistent with an exponentially-cutoff spectrum and
is therefore classified as type `M'.  The source's marginal extension
($TS_{\rm ext}=8.7$) is likely due to some contamination from the SNR.

