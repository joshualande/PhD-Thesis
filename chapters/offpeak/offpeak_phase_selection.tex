\section{Off-peak Phase Selection}
\seclabel{peak_definition}

We first developed a systematic, model-independent, and computationally-efficient method 
to define the off-peak interval of a pulsar light curve.

We begin by deconstructing the light curve 
into simple Bayesian Blocks using the algorithm described in
\citet{jackson_2005a_algorithm-optimal} and \citet{scargle_2013a_studies-astronomical}.  
We could not apply the Bayesian Block algorithm to the weighted-counts
light curves because they do not follow Poisson statistics, required
by the algorithm.
We therefore use an unweighted-counts light curve in which the angular radius and minimum energy selection have been varied to maximize the H-test statistic.  To produce Bayesian Blocks on a periodic light curve,
we extend the data over three rotations, by copying and shifting the
observed phases to cover the phase range from $-$1 to 2.  We do, however,
define the final blocks to be between phases 0 and 1.  
To avoid potential contamination from the trailing or leading edges of
the peaks, we reduce the extent of the block by 10\% on either side,
referenced to the center of the block.

There is one free parameter in the Bayesian Block algorithm called
ncp$\rm _{prior}$ which modifies the probability that the
algorithm will divide a block into smaller intervals.
We found that, in most cases, setting ncp$\rm _{prior}=8$ protects against
the Bayesian Block decomposition containing unphysically small blocks.
For a few marginally-detected pulsars, the algorithm failed 
to find more than one block and we had to decrease ncp$\rm _{prior}$ until the
algorithm found a variable light curve. Finally, for a few pulsars the 
Bayesian-block decomposition of the light curves failed to model 
weak peaks found by the light-curve fitting method
presented in \citep{abdo_2013a_second-fermi} or extended
too far into the other peaks. For these pulsars,
we conservatively shrink the off-peak region.

For some pulsars, the observed light curve has two well-separated peaks
with no significant bridge emission, which leads to two well-defined
off-peak intervals.  We account for this possibility by finding the second-lowest 
Bayesian block and accepting it as a second off-peak interval if
the emission is consistent with that in the lowest block (at the 99\%
confidence level) and if the extent of the second block is at least half
that of the first block.

\figref{off_peak_select} shows the energy-and-radius optimized
light curves, the Bayesian block decompositions, 
and the off-peak intervals for six pulsars.  
\citep{abdo_2013a_second-fermi} overlay off-peak intervals
over the weighted light curves of several pulsars.
The off-peak intervals for all pulsars are given in 
\citep{abdo_2013a_second-fermi}.

\begin{figure}
  \includegraphics{chapters/offpeak/figures/off_peak_phase_color.pdf}
  \caption{The energy-and-radius optimized light curve, Bayesian block decomposition of the        
  light curve, and off-peak interval for
  (a) PSR J0007+7303, (b) PSR J0205+6449, (c) PSR J1410$-$6132,
  (d) PSR J1747$-$2958, (e) PSR J2021+4026, and (f) PSR J2124$-$3358.
  The black histograms represent the light curves,
  the gray lines (colored red in the electronic version)
  represent the Bayesian block decompositions of the pulsar light curves, and
  the hatched areas represent the off-peak intervals selected by this method.}
  \figlabel{off_peak_select}
\end{figure}

