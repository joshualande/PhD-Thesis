Some pulsars have magnetospheric emission over their full rotation phase
with similar spectral characteristics to the emission seen through
their peaks.  This emission appears in the observed light curves as
a low-level, unpulsed component above the estimated background level
(i.e., not attributable to diffuse emission or nearby point sources)
and can be a powerful discriminator for the emission models.

On the other hand a PWN around the pulsar, or a photon excess due
to imprecise knowledge of diffuse emission around the pulsar, would
not be modulated at the rotational period and could be confused with
a constant magnetospheric signal.  Including the discovery of the GeV
PWN 3C 58 associated with PSR J0205+6449 described in this section, the
LAT sees 17 sources potentially associated with PWNe at GeV energies
\citep{acero_2013a_constraints-galactic}.  Some are highlighted in
\secref{App-off_peak_individual_source_discussion}.  This off-peak
emission should be properly modeled when searching for pulsar emission
at all rotation phases.

We can discriminate between these two possible signals through
spectral and spatial analysis.  If the emission is magnetospheric,
it is more likely to appear as a non-variable point source with
an exponentially cutoff spectrum with a well-known range of cutoff
energies.  On the other hand, PWNe and diffuse excesses have spectra
with a power-law shape and either a hard index continuing up to tens
of GeV in the PWN case or present only at lower energies with a very
soft index in the diffuse case.  In addition, PWNe are often spatially
resolvable at GeV energies \citep[e.g., Vela-X has been spatially
resolved with the LAT and \textit{AGILE} and HESS J1825$-$137 with the
LAT;][respectively]{abdo_2010c_fermi-large,pellizzoni_2010a_detection-gamma-ray,grondin_2011a_detection-pulsar}
so an extended source would argue against a magnetospheric origin
of the emission.  However, given the finite angular resolution
of the LAT (see \secref{obsvSection}) not all PWNe will appear
spatially extended at GeV energies.  The Crab Nebula, for instance,
cannot be resolved by the LAT but can be distinguished from the
gamma-bright Crab pulsar, in the off-peak interval, by its hard
spectrum above $\sim$1 GeV \citep{abdo_2010a_fermi-large}.  In addition, GeV
emission from the Crab Nebula was discovered to be time-variable
\citep[e.g.,][]{abdo_2011a_gamma-ray-flares} providing another
possible way to discern the nature of any observed off-peak signal.


Therefore, to identify pulsars with magnetospheric emission across
the entire rotation, we define and search the off-peak intervals
of the pulsars in this catalog for significant emission, except PSR
J2215+5135 for which the rotation ephemeris covers a short time interval
and the profile is noisy.  We then evaluate the spectral and spatial
characteristics of any off-peak emission to determine if it is likely
magnetospheric, related to the pulsar wind, or physically unrelated to
the pulsar (e.g., unmodeled diffuse emission).
