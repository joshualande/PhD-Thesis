Some pulsars have magnetospheric emission over their full rotation phase with similar spectral characteristics to the emission seen through their peaks.
%GeV emission from pulsars is modulated at the rotational
%period, with the emission concentrated in one or more
%narrow peaks.  Depending on the viewing geometry and emission model,
%gamma-ray light curves can have a 100\% duty cycle.
This emission appears in the observed light curves as a low-level,
unpulsed component above the estimated background level (i.e., not
attributable to diffuse emission or nearby point sources) and can be a
powerful discriminator for the emission models.

On the other hand a PWN around the pulsar, or a photon excess due to imprecise knowledge of
diffuse emission around the pulsar, would not be modulated at the rotational period and could be confused with a constant magnetospheric signal. 
Including the discovery of the GeV PWN 3C 58 associated with PSR J0205+6449 described in this section, 
the LAT sees 17 sources potentially associated with PWNe at GeV energies \citep{Rousseau2013}. 
Some are highlighted in Appendix~\ref{App-off_peak_individual_source_discussion}.
This off-peak emission should be properly modeled when searching for pulsar emission at all rotation phases.


%Young pulsars are also known to power PWNe
%which can be bright at GeV and TeV energies.  This emission
%would not be modulated at the rotational period and could be
%confused with the magnetospheric signal.
%While the LAT has detected emission from several bright PWNe
%\citep[e.g.;][]{FermiCrab,LAT_collaboration_Vela_X_2010}, fainter sources
%may contribute at low levels and have been undetected with less data.
%These sources would, therefore, not be included in the model of the
%region or the light curve background estimate (see Sections \ref{profiles}
%and \ref{spectralMethodSection}).

We can discriminate between these two possible
signals through spectral and spatial analysis.  If the emission is
magnetospheric, it is more likely to appear as a non-variable point source
with an exponentially cutoff spectrum with a well-known range of cutoff energies.  
On the other hand, PWNe and diffuse excesses have spectra with a power-law shape and either a hard index continuing up to tens of GeV in the PWN case or present only at lower energies with a very soft index in the diffuse case.  
In addition, PWNe
are often spatially resolvable at GeV energies \citep[e.g., Vela-X has been spatially resolved with the LAT and \textit{AGILE}
and HESS J1825$-$137 with the LAT;][respectively]{LAT_collaboration_Vela_X_2010,AGILE_VelaX,LAT_collaboration_HESS_J1825_2011}
so an extended source would argue against a magnetospheric origin
of the emission.  However, given the finite angular resolution of
the LAT (see Sec. \ref{obsvSection}) not all PWNe will appear
spatially extended at GeV energies.  The Crab Nebula, for instance,
cannot be resolved by the LAT but can be distinguished from the
gamma-bright Crab pulsar, in the off-peak interval, by its hard
spectrum above $\sim$1 GeV \citep{FermiCrab}.  In addition, GeV
emission from the Crab Nebula was discovered to be time-variable
\citep[e.g.,][]{LAT_Collaboration_Crab_Flare_2011} providing another
possible way to discern the nature of any observed off-peak signal.


Therefore, to identify pulsars with magnetospheric emission across
the entire rotation, we define and search the off-peak intervals of
the pulsars in this catalog for significant emission, except PSR J2215+5135 for
which the rotation ephemeris covers a short time interval and the profile is noisy.
We then evaluate the spectral and spatial characteristics of any off-peak emission
to determine if it is likely magnetospheric, related to the pulsar wind,
or physically unrelated to the pulsar (e.g., unmodeled diffuse emission).
