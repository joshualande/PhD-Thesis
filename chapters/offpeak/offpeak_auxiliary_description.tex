% \clearpage
\subsection{Description of Auxiliary information for Off-peak Analysis}
\label{off_peak_auxiliary}

A more complete list of the off-peak spectral and spatial 
results from the analysis described in \secref{unpulsed}
is contained
in a supplemental FITS table.
\tabref{off_peak_aux} describes each of its columns.

For sources reported with a PL spectral model,
the differential spectrum is defined as
\begin{equation}\eqnlabel{powerlaw_definition}
  \dnde = \fitsformat{Prefactor} \left(\frac{E}{\fitsformat{Scale}}\right)^{-\fitsformat{Index}}
\end{equation}
\fitsformat{Prefactor}, \fitsformat{Index}, and \fitsformat{Scale} are specified in the table.
For sources reported with a PLEC1 spectral model (\fitsformat{\PLSuperExpCutoff}), the differential spectrum is defined as:
\begin{equation}\eqnlabel{cutoff_powerlaw_definition}
  \dnde = \fitsformat{Prefactor} \left(\frac{E}{\fitsformat{Scale}}\right)^{-\fitsformat{Index}} \exp\left(-\frac{E}{\fitsformat{Energy\_Cutoff}}\right)
\end{equation}
and \fitsformat{Prefactor}, \fitsformat{Index}, \fitsformat{Scale}, and \fitsformat{Energy\_Cutoff} are specified in the table.

For the Crab Nebula and Vela-X, we took the spectral shape and
initial normalization from \citet{LAT_collaboration_crab_2012} and
\cite{FermiVelaX2nd} respectively
and fit only a
multiplicative offset (see \secref{off_peak_analysis}). For these two
sources, the differential spectrum was defined as
\begin{equation}\eqnlabel{filefunction_definition}
  \dnde = \fitsformat{Normalization} \left.\frac{dN}{dE}\right|_\mathrm{file}
\end{equation}
and the \fitsformat{Normalization} is defined in the auxiliary table.

% \clearpage
\begin{deluxetable}{p{1in} p{1in} p{3.5in}}
\tablewidth{0pt}
\tabletypesize{\scriptsize}
\tablecaption{Description of the Columns in the Off-Peak Auxiliary Fits Table
\tablabel{off_peak_aux}
}
\tablewidth{6in}
\input{OffPeak/tables/off_peak_auxiliary_table_description.tex}
\end{deluxetable}
\clearpage
