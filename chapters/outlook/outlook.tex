\chapter{Outlook}
\chaplabel{outlook}

Since the observation of the Crab Nebula
in \citeyear{weekes_1989a_observation-gamma}
\citep{weekes_1989a_observation-gamma}, we have learned much about the
high-energy \ac{IC} emission from \ac{PWN}. The current generation of
\ac{VHE} experiments (\ac{HESS}, Magic, and Veritas) have drastically
expanded the population of \ac{PWN} observed at $\gamma$-ray energies and
\acp{PWN} are the most populous class of \ac{VHE} sources in the Galaxy.
Now, using the \ac{LAT} on board \fermi, we have detected a large
fraction of these \ac{VHE} \ac{PWN} at \gev energies and one \ac{PWN}
not yet detected at \ac{VHE} energies.

The next great improvement in our knowledge of \ac{PWN} will most
likely come from next-generation \acp{IACT}. The proposed \ac{CTA}
\citep{actis_2011a_design-concepts} will have a much improved effective
area and angular resolution, allowing for the discovery of more \ac{VHE}
\ac{PWN} as well as improved imaging of \ac{PWN} candidates.

As was the case for \hessj{1825}, energy-dependent morphology at \ac{VHE}
energies can be used to unambiguous identify \ac{VHE} emission as being
caused by a pulsar \citep{aharonian_2006a_energy-dependent}.  Similarly,
\cite{van-etten_2011a_multi-zone-modeling} showed for \hessj{1825} that
detailed spatial and spectral observations combined with multi-zone
modeling can constrain the properties of the \ac{PWN}.  Detailed
energy-dependent imaging of a larger sample of \ac{PWN} by \ac{CTA}
will allow us a greater understanding of the physics of pulsar winds.

In addition, the Crab nebula has challenged our basic understand of the
physics of \acp{PWN}. It is possible that more detailed observations could
uncover additional variable \ac{PWN} and this could help to explain the
nature of this variable emission.

Finally, because of the high density of \ac{VHE} \ac{PWN} in the galactic
plane, it is important to identify \ac{VHE} sources as \ac{PWN} to assist
in the search for new source classes.  There is significant potential
for the discovery of new \ac{VHE} source classes, but only after the
numerous \ac{VHE} \ac{PWN} are classified.  If the past is any guide to
the future, there is much still to be learned.
