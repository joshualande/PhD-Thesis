\chapter{Search for \Acsptitle{PWN} associated with TeV Pulsars}
\chaplabel{tevcat}

\paperref{This chapter is based the first part of the the paper
  ``Constraints on the Galactic Population of TeV Pulsar Wind Nebulae using Fermi Large Area Telescope Observations''
  by Acero et al which is currently in prep.}

TODO:
\begin{itemize}
  \item Strip out sources not classified as PWN/PSR
  \item Remove the fit without the LAT PSR included in the background model.
\end{itemize}

In \chapref{extended_search}, we searched for spatially-extended sources
in the 2FGL catalog.  This search showed that the spatial analysis of
\fermi sources is important in identifying $\gamma$-ray emitting \acp{PWN}.
In this work, we analyzed the $\gamma$-ray emitting \acp{PWN} \hess{J1825}
and \mshfifteenfiftytwo which had previously been detected in \ac{2FGL}.
In addition, this analysis discovered that there were three additional
spatially-extended \fermi soruces coincident with \acp{PWN} candidates
(\hess{J1616}, \hess{J1632}, \hess{J1837}).
In \chapref{offpeak}, we then searched for $\gamma$-ray emitting \acp{PWN}
by looking in the off-peak emission of \ac{LAT}-detected
pulsars. In this analysis, we detected four $\gamma$-ray emitting \acp{PWN}
candidates (\velax, the Crab Nebula, \mshfifteenfiftytwo, and \threecfiftyeight).

In this chapter, we continue our search for $\gamma$-ray emitting
\acp{PWN} by searching for \acp{PWN} which had previously been detected
at \tev energies by \acp{IACT}. We note that the work presented here is a
very condensed version of the results presented in the accompanying work
\citep{acero_2013a_constraints-galactic}.  We refer to that publication
for a more detailed discussion of the analysis.

\section{Introduction}

In this analysis, we took all sources detected at \tev energies and
potentially associated with \acp{PWN} and performed a search at \gev
energies for $\gamma$-ray emission. As was seen in previous sections,
many sources such as the Crab Nebula and \velax had been observed at
both \gev and \tev energies. In addition, there are several \ac{PWN}
which have been detected at \tev energies which do not have an associated
$\gamma$-ray pulsar (such as \hess{J1825} and \hess{J1837}).  We therefore
suspect that a thorough search of all \tev \acp{PWN} candidates might
discover new $\gamma$-ray emitting \acp{PWN} not previously discovered
either in the off-peak search discussed in \chapref{offpeak} or in other
dedicated analyses.

\section{List of \tev \Acstitle{PWN} Candidates}

\begin{deluxetable}{llrrl}
\tabletypesize{\tiny}
\tablecaption{List of analyzed \tev sources
\label{tab:TeV_sources}}
\tablewidth{0pt}
\tablecolumns{5}

\tablehead{\colhead{Name} & \colhead{Class} & \colhead{$l$} & \colhead{$b$} & \colhead{Reference}\\
\colhead{} & \colhead{} & \colhead{(deg.)} & \colhead{(deg.)} & \colhead{}}
\startdata
  VER~J0006$+$727 &  PWN & 119.58 &   10.20 &     \cite{2011arXiv1111.2591M}\\
 MGRO~J0631$+$105 &  PWN & 201.30 &    0.51 &     \cite{2009ApJ...700L.127A}\\
  MGRO~J0632$+$17 &  PWN & 195.34 &    3.78 &    \cite{2009ApJ...700L.127A} \\
 HESS~J1018$-$589 & UNID & 284.23 & $-1.72$ &     \cite{2012AA...541A...5H} \\
 HESS~J1023$-$575 &  MSC & 284.22 & $-0.40$ &      \cite{2011AA...525A..46H}\\
 HESS~J1026$-$582 &  PWN & 284.80 & $-0.52$ &     \cite{2011AA...525A..46H} \\
 HESS~J1119$-$614 &  PWN & 292.10 & $-0.49$ &  Presentation\tablenotemark{a}\\
 HESS~J1303$-$631 &  PWN & 304.24 & $-0.36$ &      \cite{2005AA...439.1013A}\\
 HESS~J1356$-$645 &  PWN & 309.81 & $-2.49$ &      \cite{2011AA...533A.103H}\\
 HESS~J1418$-$609 &  PWN & 313.25 &    0.15 &      \cite{2006AA...456..245A}\\
 HESS~J1420$-$607 &  PWN & 313.56 &    0.27 &      \cite{2006AA...456..245A}\\
 HESS~J1427$-$608 & UNID & 314.41 & $-0.14$ &      \cite{2008AA...477..353A}\\
 HESS~J1458$-$608 &  PWN & 317.75 & $-1.70$ &     \cite{2012arXiv1205.0719D}\\
 HESS~J1503$-$582 & UNID & 319.62 &    0.29 &     \cite{2008AIPC.1085..281R}\\
 HESS~J1507$-$622 & UNID & 317.95 & $-3.49$ &      \cite{2011AA...525A..45H}\\
 HESS~J1514$-$591 &  PWN & 320.33 & $-1.19$ &      \cite{2005AA...435L..17A}\\
 HESS~J1554$-$550 &  PWN & 327.16 & $-1.07$ &     \cite{2012arXiv1201.0481A}\\
 HESS~J1616$-$508 &  PWN & 332.39 & $-0.14$ &     \cite{2006ApJ...636..777A}\\
 HESS~J1626$-$490 & UNID & 334.77 &    0.05 &      \cite{2008AA...477..353A}\\
 HESS~J1632$-$478 &  PWN & 336.38 &    0.19 &     \cite{2006ApJ...636..777A}\\
 HESS~J1634$-$472 & UNID & 337.11 &    0.22 &     \cite{2006ApJ...636..777A}\\
 HESS~J1640$-$465 &  PWN & 338.32 & $-0.02$ &     \cite{2006ApJ...636..777A}\\
 HESS~J1702$-$420 & UNID & 344.30 & $-0.18$ &     \cite{2006ApJ...636..777A}\\
 HESS~J1708$-$443 &  PWN & 343.06 & $-2.38$ &      \cite{2011AA...528A.143H}\\
 HESS~J1718$-$385 &  PWN & 348.83 & $-0.49$ &      \cite{2007AA...472..489A}\\
 HESS~J1729$-$345 & UNID & 353.44 & $-0.13$ &      \cite{2011AA...531A..81H}\\
 HESS~J1804$-$216 & UNID &   8.40 & $-0.03$ &    \cite{2006ApJ...636..777A} \\
 HESS~J1809$-$193 &  PWN &  11.18 & $-0.09$ &      \cite{2007AA...472..489A}\\
 HESS~J1813$-$178 &  PWN &  12.81 & $-0.03$ &     \cite{2006ApJ...636..777A}\\
 HESS~J1818$-$154 &  PWN &  15.41 &    0.17 &    \cite{2011arXiv1112.2901H} \\
 HESS~J1825$-$137 &  PWN &  17.71 & $-0.70$ &      \cite{2006AA...460..365A}\\
 HESS~J1831$-$098 &  PWN &  21.85 & $-0.11$ &     \cite{2011ICRC....7..243S}\\
 HESS~J1833$-$105 &  PWN &  21.51 & $-0.88$ &     \cite{2008ICRC....2..823D}\\
 HESS~J1834$-$087 & UNID &  23.24 & $-0.31$ &     \cite{2006ApJ...636..777A}\\
 HESS~J1837$-$069 & UNID &  25.18 & $-0.12$ &     \cite{2006ApJ...636..777A}\\
 HESS~J1841$-$055 & UNID &  26.80 & $-0.20$ &      \cite{2008AA...477..353A}\\
 HESS~J1843$-$033 & UNID &  29.30 &    0.51 &     \cite{2008ICRC....2..579H}\\
 MGRO~J1844$-$035 & UNID &  28.91 & $-0.02$ &     \cite{2009ApJ...700L.127A}\\
 HESS~J1846$-$029 &  PWN &  29.70 & $-0.24$ &     \cite{2008ICRC....2..823D}\\
 HESS~J1848$-$018 & UNID &  31.00 & $-0.16$ &     \cite{2008AIPC.1085..372C}\\
 HESS~J1849$-$000 &  PWN &  32.64 &    0.53 &     \cite{2008AIPC.1085..312T}\\
 HESS~J1857$+$026 & UNID &  35.96 & $-0.06$ &      \cite{2008AA...477..353A}\\
 HESS~J1858$+$020 & UNID &  35.58 & $-0.58$ &      \cite{2008AA...477..353A}\\
 MGRO~J1900$+$039 & UNID &  37.42 & $-0.11$ &     \cite{2009ApJ...700L.127A}\\
  MGRO~J1908$+$06 & UNID &  40.39 & $-0.79$ &      \cite{2009AA...499..723A}\\
 HESS~J1912$+$101 &  PWN &  44.39 & $-0.07$ &      \cite{2008AA...484..435A}\\
  VER~J1930$+$188 &  PWN &  54.10 &    0.26 &    \cite{2010ApJ...719L..69A} \\
MGRO~J1958$+$2848 &  PWN &  65.85 & $-0.23$ &     \cite{2009ApJ...700L.127A}\\
  VER~J1959$+$208 &  PSR &  59.20 & $-4.70$ &     \cite{2003ApJ...583..853H}\\
  VER~J2016$+$372 & UNID &  74.94 &    1.15 &     \cite{2011arXiv1110.4656A}\\
  MGRO~J2019$+$37 &  PWN &  75.00 &    0.39 &     \cite{2007ApJ...664L..91A}\\
 MGRO~J2031$+$41A & UNID &  79.53 &    0.64 &     \cite{2007ApJ...664L..91A}\\
 MGRO~J2031$+$41B & UNID &  80.25 &    1.07 &     \cite{2012ApJ...745L..22B}\\
  MGRO~J2228$+$61 &  PWN & 106.57 &    2.91 &     \cite{2009ApJ...700L.127A}\\
\enddata


\tablenotetext{a}{This source was presented at the "Supernova Remnants and Pulsar Wind
Nebulae in the Chandra Era", 2009. See \url{http://cxc.harvard.edu/cdo/snr09/pres/DjannatiAtai\_Arache\_v2.pdf}.}

\tablecomments{The \tev sources that we searched for using \ac{LAT} observations.
The classifications come form \tevcat and are \acs{PWN} for \acl{PWN}, \ac{UNID} for \acl{UNID}, and \acs{MSC} for \acl{MSC}.
We include \hess{J1023} because it is potentially \aac{PWN}.}
\end{deluxetable}
\clearpage



We used \tevcat to define the list of \tev sources to saerch for \gev
\acp{PWN}. \tevcat is a catalog of sources detcted at \tev energies by 
\acp{IACT}\footnote{\tevcat can be found at \url{http://tevcat.uchicago.edu}.}.
We selected all sources from this catalog 
where the emission was classified as being
due to a \ac{PWN}. In addition, we included all sources with an UNID classification
(unidientifed emission) within 5\degree of the galactic plane, since they
could potentially be due to a \ac{PWN}.
Finally, we included \hess{J1023} which, although classified as a
massive star cluster in the \tevcat, was suggested to be a \ac{PWN} in 
\cite{de-naurois_2013a_galactic-h.e.s.s.}. The list of all sources
included in our analysis as well as their classification in \tevcat
can be found in \tabref{TeV_sources}.

\section{Analysis Method}

In this search, our analysis method was very similar to the analysis
in \chapref{offpeak}. We used the same hybrid \pointlike/\gtlike
approach for studying the spatial and spectral character of each source
and modeled the region using the same standard background models.

The major difference was that this analysis
was performed only for $\energy>10\unitspace\gev$.  As can be seen in
\chapref{offpeak}, for energies much lower than $10\unitspace\gev$,
source analysis becomes strongly influenced by Galactic-diffuse
emission and systematic errors associated with incorrectly modeling
the emission. On the other hand, the $\gamma$-ray emission from \ac{PWN}
is expected to be the rising component of an \ac{IC} peak which falls
at \tev energies. Therefore, the emission observed by the \ac{LAT} is
expected to be hard and most significant at higher energies. Therefore,
we expect that starting the analysis  at $10\unitspace\gev$ will
significantly reduce systematics associated with this analysis while
preserving most of the space for discovery.

Because the analysis was performed only in this high energy range where
the \ac{PSF} of the \ac{LAT} is much improved, we were able to use
a smaller region of interest (a radius of $5\degree$ in \pointlike 
and a square of size $7\degree\times7\degree$ in \gtlike).

Another differences it that we used an event class with less background
contamination (Pass 7 Clean instead of Pass 7 Source) and modeled
nearby background sources using \ac{1FHL} \cite{ackermann_2013a_first-fermi-lat}.

For our analysis, we assume the \gev emission to have a power law spectral
model and to have whatever was the best-fit spatial model observed at
\tev energies.  We define \tstev as the likelihood-ratio test for the
significance of the source assuming it to have the power-law spectral
model and \tev spatial model.  We consider a source to be detected when
$\tstev > 16$.  Our significance test has only two degrees of freedom:
the flux and spectral index.  Therefore, following Wilk's Theorem (see
\subsecref{monte_carlo_validation}), this corresponds to a $3.6\sigma$
detection threshold. When the source is significantly-detected, we quote
the best-fit spectral parameters and otherwise we derive a upper limit
on the flux of any potential emission.

We note that \cite{acero_2013a_constraints-galactic} performs a more
detailed morphological analysis which fits the positions of the sources
assuming the emissions to be point-like and spatially-extended.  The work
uses the \tsext test defined in \subsecref{extension_fitting} to test if
the emission is spatially-extended and otherwise computes an upper limit
on any potential spatial extension.  This additional analysis is relevant
for comparing the spatial overlap between the GeV and TeV emission. But
for brevity, we omit the details and refer to that publication.

TODO, discuss issue about soruces with a nearby LAT-detcted pulsar.
Which soruces have a LAT-dtected pulasr, what does 
\cite{acero_2013a_constraints-galactic} do about this?

In addition \cite{acero_2013a_constraints-galactic} performs
a careful study of the systematics associated with this analysi.
\todo[inline]{WHAT SYSTEMATICS}.

\section{Sources Detected}

The spectral and spatial results for significanlty-detected sources is prestended in
\todo[inline]{WHERE ARE RESULTS PRESENTED}. We don't quote upper limits, for some reason\ldots.


\tabletypesize{\scriptsize}
\begin{deluxetable}{lllllrrll}
\tablewidth{0pt}
\tablecolumns{9}
\tablecaption{Spatial results 
\label{tab:Spat_results}}

\tablehead{ \colhead{Name} & \colhead{ID} & \colhead{$\text{TS}_{\text{TeV}}$} & \colhead{$\text{TS}_{\text{GeV}}$} & \colhead{$\text{TS}_{\text{ext}}$} & \colhead{GLON} & \colhead{GLAT} & \colhead{POSERR} & \colhead{$\sigma$}\\ 
 \colhead{} & \colhead{} & \colhead{} & \colhead{} & \colhead{} & \colhead{(deg)} & \colhead{(deg)} & \colhead{(deg)}}
\startdata
VER~J0006+727 & PSR & 655 & 1206 & 0 & $119.68$ & $10.47$ & $0.01 \pm 0.01$ & $< 0.07$ \\ 
\tablenotemark{a} & & 2 & - & - & - & - & - \\
MGRO~J0632+17 & PSR  & 699 & 2056 & 1 & $195.13$ & $4.28$ & $0.01 \pm 0.01$ & $< 0.08$ \\ 
\tablenotemark{a} & & 5 & - & - & - & - & - \\
HESS~J1018-589 & O  & 29 & 29 & 0 & $284.33$ & $-1.66$ & $0.04 \pm 0.02$ & $< 0.87$ \\ 
\tablenotemark{a} &   & 25 & 25 & 2 & $284.34$ & $-1.65$ & $0.04 \pm 0.02$ & $< 0.87$ \\
HESS~J1023-575 & PWNc  & 52 & 58 & 8 & $284.13$ & $-0.45$ & $0.03 \pm 0.02$ & $< 0.77$ \\ 
\tablenotemark{a} &   & 52 & 58 & 8 & $284.13$ & $-0.45$ & $0.03 \pm 0.02$ & $< 0.77$ \\
HESS~J1119-614 & PWNc  & 27 & 27 & 9 & $292.16$ & $-0.56$ & $ 0.05 \pm 0.02 $ & $< 0.31$ \\
\tablenotemark{a} &  & 16 & 16 & 9 & $292.18$ & $-0.57$ & $0.05 \pm 0.02$ & $< 0.32$ \\ 
HESS~J1303-631 & PWNc & 37 & 58 & 29 & $304.56$ & $-0.11$ & $0.04 \pm 0.03$ & $0.45 \pm 0.09 \pm 0.10$\\
HESS~J1356-645 & PWN  & 24 & 26 & 3 & $309.71$ & $-2.33$ & $0.05 \pm 0.01$ & $< 0.39$ \\ 
\tablenotemark{a} &   & 24 & 26 & 3 & $309.71$ & $-2.32$ & $0.05 \pm 0.01$ & $< 0.39$ \\
HESS~J1418-609 & PSR  & 31 & 32 & 0 & $313.28$ & $0.13$ & $0.03 \pm 0.01$ & $< 0.32$ \\ 
\tablenotemark{a} & & 15 & - & - & - & - & - \\
HESS~J1420-607 & PWNc  & 42 & 42 & 0 & $313.55$ & $0.27$ & $0.04 \pm 0.02$ & $< 0.38$ \\
\tablenotemark{a} & & 36 & 36 & 0 & $313.55$ & $0.28$ & $0.04 \pm 0.02$ & $< 0.39$ \\ 
HESS~J1507-622 & O  & 21 & 23 & 7 & $317.77$ & $-3.60$ & $0.05 \pm 0.03$ & $< 1.04$ \\ 
\tablenotemark{a} & & 21 & 23 & 7 & $317.76$ & $-3.61$ & $0.05 \pm 0.03$ & $< 1.04$ \\
HESS~J1514-591 & PWN  & 156 & 147 & 10 & $320.35$ & $-1.25$ & $0.03 \pm 0.01$ & $< 0.16$ \\ 
\tablenotemark{a} & & 156 & 147 & 10 & $320.35$ & $-1.25$ & $0.03 \pm 0.01$ & $< 0.16$ \\
HESS~J1614-518 & O  & 110 & 133 & 47 & $331.62$ & $-0.74$ & $0.04 \pm 0.03$ & $0.28 \pm 0.03 \pm 0.05$\\
HESS~J1616-508 & PWNc  & 75 & 94 & 31 & $332.39$ & $-0.27$ & $0.04 \pm 0.02$ & $0.25 \pm 0.03 \pm 0.05$\\
HESS~J1632-478 & PWNc  & 137 & 161 & 56 & $336.50$ & $0.10$ & $0.03 \pm 0.02$ & $0.30 \pm 0.06 \pm 0.06$\\
HESS~J1634-472 & O  & 33 & 34 & 1 & $337.23$ & $0.35$ & $0.03 \pm 0.01$ &$< 1.21$ \\ 
HESS~J1640-465 & PWNc  & 47 & 42 & 9 & $338.33$ & $0.05$ & $0.05 \pm 0.01$ & $< 1.17$ \\ 
HESS~J1708-443 & PSR  & 722 & 1153 & 0 & $343.11$ & $-2.70$ & $0.01 \pm 0.01$ & $< 0.05$ \\
\tablenotemark{a} & & 33 & 64 & 0 & $343.12$ & $-2.70$ & $0.01 \pm 0.01$ & $< 0.09$ \\ 
HESS~J1804-216 & O  & 138 & 141 & 37 & $8.40$ & $-0.09$ & $0.04 \pm 0.01$ & $0.25 \pm 0.03 \pm 0.04$\\
\tablenotemark{a} & & 124 & 128 & 30 & $8.42$ & $-0.10$ & $0.04 \pm 0.01$ & $0.24 \pm 0.03 \pm 0.04$\\
HESS~J1825-137 & PWN  & 56 & 82 & 30 & $17.55$ & $-0.47$ & $0.05 \pm 0.03$ & $0.44 \pm 0.08 \pm 0.09$\\
HESS~J1834-087 & O  & 27 & 36 & 4 & $23.20$ & $-0.26$ & $0.05 \pm 0.01$ & $< 0.22$ \\ 
HESS~J1837-069 & PWNc  & 73 & 119 & 46 & $25.17$ & $0.00$  & $0.05 \pm 0.03$ & $0.36 \pm 0.06 \pm 0.04$\\
HESS~J1841-055 & PWNc  & 64 & 70 & 32 & $27.01$ & $-0.15$ & $0.05 \pm 0.03$ & $0.38 \pm 0.06 \pm 0.06$\\
HESS~J1848-018 & PWNc  & 19 & 19 & 0 & $30.90$ & $-0.20$ & $0.04 \pm 0.01$ & $< 1.50$ \\ 
HESS~J1857+026 & PWNc  & 53 & 55 & 8 & $36.08$ & $0.02$ & $0.04 \pm 0.01$ & $< 0.28$ \\ 
MGRO~J1908+06 & PSR  & 16 & 37 & 2 & $40.11$ & $-0.84$ & $0.03 \pm 0.01$ &$< 0.19$ \\ 
\tablenotemark{a} & & 9 & - & - & - & - & - & - \\
MGRO~J1958+2848 & PSR  & 21 & 24 & 0 & $65.88$ & $-0.34$ & $0.04 \pm 0.01$ & $< 0.56$ \\ 
\tablenotemark{a} & & 8 & - & - & - & - & - \\
VER~J2016+372 & O  & 31 & 33 & 1 & $74.86$ & $1.22$ & $0.05 \pm 0.02$ & $< 1.16$ \\ 
MGRO~J2019+37 & PSR & 31 & 100 & 1 & $75.23$ & $0.13$ & $0.02 \pm 0.01$ & $< 0.07$ \\ 
\tablenotemark{a} & & 5 & - & - & - & - & - \\
MGRO~J2031+41 B & PSR  & 58 & 66 & 5 & $80.20$ & $1.03$ & $0.05 \pm 0.01$ & $< 2.47$ \\ 
\tablenotemark{a} & & 12 & - & - & - & - & - \\
MGRO~J2228+61 & PSR & 94 & 113 & 0 & $106.65$ & $2.94$ & $0.02 \pm 0.01$ & $< 0.10$ \\ 
\tablenotemark{a} & & 15 & - & - & - & - & - \\
\enddata


\tablecomments{Results of the maximum likelihood spatial fits for LAT-detected TeV sources. ``a" in the first column corresponds to the results with contribution of the pulsar associated in Table~\ref{tab:pulsars} subtracted from the emission of the source just above. Column 2 lists the classification for the LAT emission: either ``PWN" for clearly identified PWNe, ``PWNc" for PWN candidates, ``PSR" for pulsar emission, and ``O" for anything else. Column 3 is the TS when the source is modeled with the spatial model obtained from TeV energies. Column 4 is the TS when the source is modeled assuming it is point-like, and column 5 is the TS of the source assuming it is spatially-extended with a Gaussian spatial model. Columns 6 and 7 give the position of the source fit in the GeV energy range. The method for determining systematic uncertainties on the spatial parameters are described in Section \ref{systext}. Column 8 gives the extension fit in the GeV energy range if $\text{TS}_{\text{ext}} > 16$ or a 99 \% c.l. upper limit on the extension otherwise.}
\end{deluxetable}




\begin{itemize}
  \item
    We detected 4 new PWNe candidates (\hess{J1119}, \hess{J1303},
    \hess{J1420},
    and \hess{J1841})
    and 1 new PWN (\hess{J1356})
  \item
\end{itemize}
