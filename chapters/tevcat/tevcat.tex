\chapter{Search for \Acsptitle{PWN} associated with TeV Pulsars}
\chaplabel{tevcat}

\paperref{This chapter is based the first part of the the paper
  ``Constraints on the Galactic Population of TeV Pulsar Wind Nebulae using Fermi Large Area Telescope Observations''
  by Acero et al which is currently in prep.}

In \chapref{extended_search}, we searched for spatially-extended sources
in the 2FGL catalog.  This search showed that the spatial analysis of
\fermi sources is important in identifying $\gamma$-ray emitting \acp{PWN}.
In this work, we analyzed the $\gamma$-ray emitting \acp{PWN} \hess{J1825}
and \mshfifteenfiftytwo which had previously been detected in \ac{2FGL}.
In addition, this analysis discovered that there were three additional
spatially-extended \fermi soruces coincident with \acp{PWN} candidates
(\hess{J1616}, \hess{J1632}, \hess{J1837}).
In \chapref{offpeak}, we then searched for $\gamma$-ray emitting \acp{PWN}
by looking in the off-peak emission of \ac{LAT}-detected
pulsars. In this analysis, we detected four $\gamma$-ray emitting \acp{PWN}
candidates (\velax, the Crab Nebula, \mshfifteenfiftytwo, and \threecfiftyeight).

In this chapter, we continue our search for $\gamma$-ray emitting
\acp{PWN} by searching for \acp{PWN} which had previously been detected
at \tev energies by \acp{IACT}. We note that the work presented here is
a very condensed version of the results presented in the accompanying work.
We refer to that publication for a more detailed discussion of
the analysis.

\section{Introduction}

In this analysis, we took all sources detected at \tev energies and
potentially associated with \acp{PWN} and performed a search at \gev
energies for $\gamma$-ray emission. As was seen in previous sections,
many sources such as the Crab Nebula and \velax had been observed at
both \gev and \tev energies. In addition, there are several \ac{PWN}
which have been detected at \tev energies which do not have an associated
$\gamma$-ray pulsar (such as \hess{J1825} and \hess{J1837}).  We therefore
suspect that a thorough search of all \tev \acp{PWN} candidates might
discover new $\gamma$-ray emitting \acp{PWN} not previously discovered
either in the off-peak search discussed in \chapref{offpeak} or in other
dedicated analyses.

\section{List of Candidates}

We used \tevcat to define the list of \tev sources to saerch for \gev
\acp{PWN}. \tevcat is a catalog of sources detcted at \tev energies by 
\acp{IACT}\footnote{\tevcat can be found at \url{http://tevcat.uchicago.edu}.}.
We selected all sources from this catalog 
where the emission was classified as being
due to a \ac{PWN}. In addition, we included all sources with an UNID classification
(unidientifed emission) within 5\degree of the galactic plane, since they
could potentially be due to a \ac{PWN}.
Finally, we included \hess{J1023} which, although classified as a
massive star cluster in the \tevcat, was suggested to be a \ac{PWN} in 
\cite{de-naurois_2013a_galactic-h.e.s.s.}. The list of all sources
included in our analysis is listed in \todo[inline]{ADD TABLE}.



\section{Analysis Method}

\begin{itemize}
  \item $\energy>10\unitspace\gev$.
  \item 
\end{itemize}

\section{Sources Detected}

\begin{itemize}
  \item
    We detected 4 new PWNe candidates (\hess{J1119}, \hess{J1303},
    \hess{J1420},
    and \hess{J1841})
    and 1 new PWN (\hess{J1356})
  \item
\end{itemize}
