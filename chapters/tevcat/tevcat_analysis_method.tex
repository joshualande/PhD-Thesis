\section{Analysis Method} 
\seclabel{tevcat_analysis_method}

Our spectral and spatial analysis method was very similar to the analysis
in \chapref{offpeak}. We used the same hybrid \pointlike/\gtlike approach
for fitting \ac{LAT} data and modeled the regions using the same standard
background models.

The largest difference was that this analysis was performed only for
$\energy>10\unitspace\gev$.  As can be seen in \chapref{offpeak},
for energies much lower than $10\unitspace\gev$ source analysis becomes
strongly biased by systematic errors associated with incorrectly modeling
the Galactic-diffuse emission.  On the other hand, the $\gamma$-ray
emission from \ac{PWN} is expected to be the rising component of an
\ac{IC} peak which falls at \ac{VHE} energies. Therefore, the emission
observed by the \ac{LAT} is expected to be hard and most significant
at higher energies. Therefore, we expect that starting the analysis  at
$10\unitspace\gev$ will significantly reduce systematics associated with
this analysis while preserving most of the space for discovery.

Because the analysis was performed only in this high energy range where
the \ac{PSF} of the \ac{LAT} is much improved, we used a smaller region
of interest (a radius of $5\degree$ in \pointlike and a square of size
$7\degree\times7\degree$ in \gtlike).  Another differences is that we
used an event class with less background contamination (Pass 7 Clean
instead of Pass 7 Source) and modeled nearby background sources using
\ac{1FHL} \citep{ackermann_2013a_first-fermi-lat}.

For our analysis, we assume the \gev emission from our source to have
a power-law spectral model and that the \gev spatial model was the
same as the published \ac{VHE} spatial model.  We define \tstev as
the likelihood-ratio test for the significance of the source assuming
this source model and claim a detection when $\tstev > 16$.  Since our
significance test has two degrees of freedom, the flux and spectral
index, following this corresponds to a $3.6\sigma$ detection threshold
(see \subsecref{monte_carlo_validation}). When a source is significantly
detected, we quote the best-fit spectral parameters.  Otherwise, we
derive an upper limit on the flux of any potential emission.  We note
that \cite{acero_2013a_constraints-galactic} performed a more detailed
morphological analysis which studied the overlap between the \gev and
\ac{VHE} emission for these sources. For brevity, we omit the details
and simply use the results.

Many of these \acp{PWN} candidates are in regions with \ac{LAT}-detected
pulsars.  For these sources, \cite{acero_2013a_constraints-galactic}
included the spectral and spatial results both with and without the
\ac{LAT}-detected pulsar in the background model. For simplicity,
we include only the analysis with the pulsar included in the
background model. We caution that this method could be biased in either
oversubtracting or undersubtracting the pulsar depending upon systematics
associated with the \ac{2FGL} fits of the pulsars.

There are three major sources of systematic uncertainties effecting
the spectrum of these sources. The first is due to uncertainty in our
modeling of the Galactic diffuse emission, which we estimate following the
method of \secref{systematic_errors_on_extension}.  The second is due to
uncertainty in the effective area, which We estimated using the method
described in \cite{ackermann_2012a_fermi-large}.  The final systematic
is due to our uncertainty in the true morphology of of the source. We
use as our systematic error the difference in spectrum when the source
is fit assuming the published \ac{VHE} spatial model and spatial model
fit from \ac{LAT} data.
