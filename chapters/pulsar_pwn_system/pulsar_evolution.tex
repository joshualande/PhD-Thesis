\section{Pulsar Evolution}
\seclabel{pulsar_evolution}

\begin{figure}[htbp]
  \centering
    \includegraphics[width=0.5\textwidth]{chapters/pulsar_pwn_system/figures/pulsar_model.pdf}
    \caption{The rotating dipole model of a pulsar. This figure is taken
    from \citep{carroll_2006_introduction-modern}.}
  \figlabel{pulsar_model}
\end{figure}

The simplest model of a pulsar is that it is a rotating dipole
magnet with the rotation axis and the magnetic axis offset by an
angle $\PulsarRotationAngle$.  A diagram of this model is shown
in \figref{pulsar_model}.  The energy output from the pulsar is then
assumed to come from rotational kinetic energy stored in the neutron
star which is released as the pulsar spins down.

Both the period \period and the period derivative
$\perioddot=d\period/d\time$ can be directly observed for a pulsar.
Except in a few \acp{MSP} which are being sped up through accretion
(see for example \cite{falanga_2005_integral-observations}), pulsars
are slowing down ($\perioddot<0$).

We write the rotational kinetic energy as
\begin{equation}\eqnlabel{rotational_energy}
  \energyrotational = \tfrac{1}{2} I \PulsarAngularFrequency^2
\end{equation}
where $\PulsarAngularFrequency = 2\pi/\period$ is the angular frequency
of the pulsar and \momentofinertia is the moment of inertia.

For a uniform sphere,
\begin{equation}
  \momentofinertia = \frac{2}{5} M R^2
\end{equation}
Assuming a canonical pulsar as was described in
\chapref{pulsar_pwn_system}, we find a canonical moment of inertia of
$\momentofinertia=10^{45}\unitspace\gram\unitspace\cm^{-2}$.

We make the connection between the pulsar's spin-down
energy and the rotational kinetic energy as $\energydot
= - \derivative\energyrotational/\dtime$. Using this,
\eqnref{rotational_energy} can be rewritten as
\begin{equation}\eqnlabel{edot_from_rotation}
  \energydot = I \PulsarAngularFrequency \PulsarAngularFrequencyDot.
\end{equation}

It is believed that as the pulsar spins down, the this rotational energy
is released as pulsed electromagnetic radiation and also as a wind of
electrons and positrons accelerated in the magnetic field of the pulsar.

If the pulsar were a pure dipole magnet, its radiation would be described
as \citep{gunn_1969_magnetic-dipole}
\begin{equation}\eqnlabel{edot_pure_dipole}
  \energydot = \frac{2\MagneticField^2 \PulsarRadius^6 
  \PulsarAngularFrequency^4 \sin^2\PulsarRotationAngle}{
  3\speedoflight^3}.
\end{equation}

Combining equations \eqnref{edot_from_rotation} and
\eqnref{edot_pure_dipole}, we find that for a pure dipole magnet,
\begin{equation}\eqnlabel{breaking_index_dipole}
  \PulsarAngularFrequencyDot \propto \PulsarAngularFrequency^3.
\end{equation}

In the few situations in which this relationship has been
definitively measured, this relationship does has not hold.
We generalize \eqnref{breaking_index_dipole} as:
\begin{equation}\eqnlabel{angular_frequency_derivative_relation}
  \PulsarAngularFrequencyDot \propto \PulsarAngularFrequency^\breakingindex
\end{equation}
where $\breakingindex$ is what we call the pulsar breaking index.
And we note that \eqnref{angular_frequency_derivative_relation} we can
can solve for \breakingindex by taking the derivative of the equation
\begin{equation}
  \breakingindex = \frac{\PulsarAngularFrequency \PulsarAngularFrequencyDotDot}{\PulsarAngularFrequencyDot^2}
\end{equation}

The breaking index is hard to measure due to timing noise and glitches in
the pulsar's phase. To this date, it has been measured in eight pulsars
\citep[See][ and references therein]{espinoza_2011_braking-index},
and in all situations $\breakingindex<3$. This suggests that there are
additional processes besides magnetic dipole radiation that contribute
to the energy release \citep{blandford_1988_interpretation-pulsar}.

\eqnref{angular_frequency_derivative_relation} is a Bernoulli differential
equation which can be integrated to solve for time:
\begin{equation}\eqnlabel{pulsar_age}
  T = \frac{\period}{(\breakingindex-1) \absval{\perioddot}}
  \left(
  1-\left(\frac{\period_0}{\period}\right)^{(\breakingindex-1)}
  \right)
\end{equation}
For a canonical $\breakingindex=3$ pulsars which is relatively old
$\period_0 \ll \period$, we obtain what is called the characteristic
age of the pulsar:
\begin{equation}
  \PulsarAge = \period/2\perioddot.
\end{equation}

Using \eqnref{edot_from_rotation} and \eqnref{breaking_index_dipole},
we can solve for the spin-down evolution of the pulsar as a function of
time \citep{pacini_1973_evolution-supernova}
\begin{equation}\eqnlabel{energy_dot_vs_time}
    \energydot(t) = \energydot_0
    \left(
    1 + \frac{\time}{\SpinDownTimescale}
    \right)^{-\frac{(\breakingindex+1)}{(\breakingindex-1)}}
\end{equation}
where
\begin{equation}
  \SpinDownTimescale \equiv \frac{\period_0}{(\breakingindex-1)\absval{\perioddot_0}}.
\end{equation}


\eqnref{angular_frequency_derivative_relation}, \eqnref{pulsar_age},
and \eqnref{energy_dot_vs_time} show us that given the current \period,
\perioddot, \energydot, $\period_0$, and breaking index \breakingindex,
we can calculate the pulsar's age and energy-emission history.

In a few situations, the pulsar's age is well known and the
breaking index can be measured, so $\period_0$ can be inferred. See
\cite{kaspi_2002_constraining-birth} for a review of the topic. For
other sources, attempts have been made to infer the initial spin-down
age based on the dynamics of an associated \ac{SNR}/\ac{PWN}
\citep{van-der-swaluw_2001_inferring-initial}.

Finally, if we assume dipole radiation is the only source of energy
release, we can combine equation \eqnref{edot_from_rotation} and
\eqnref{edot_pure_dipole} to solve for the magnetic field:
\begin{equation}
  \MagneticField = \sqrt{\frac{3\momentofinertia\speedoflight^3}{
  8\pi^2\PulsarRadius^6\sin^2\PulsarRotationAngle}\period\perioddot}
  = 3.2\times 10^{19} \sqrt{\period\perioddot} \unitspace\gauss
\end{equation}
where in the last step we assumed the canonical values of
$\momentofinertia=10^{45}\unitspace\gram\unitspace\cm^{-2}$,
$\PulsarRadius=10\unitspace\km$, $\PulsarRotationAngle=90\degree$, and we
assume that $\period$ is measured in units of seconds.  For example,
for the Crab nebula, $\period\approx33\unitspace\millisecond$
\citep{staelin_1968_pulsating-radio} and
$\perioddot\approx36\unitspace\nanosecond$
per day \citep{richards_1969a_period-pulsar} so
$\MagneticField\approx10^{12}\unitspace\gauss$.
