\section{\Actitle{PWN} Emission}

In the Pulsar/\ac{PWN} model, a population of electrons
is emitted from the magnetosphere, released as an unshocked wind,
and then accelerated at the termination shock. In the surrounding
nebula, the electrons emmit synchotron and inverse Compton

\begin{itemize}
  \item What is the characteristic synchotron energy
  \item What is the characteristic \ac{IC} energy
  \item look at: etten\_2012a\_particle-populations
\end{itemize}




Energetics of PWNe:

``The efficiency of conversion of spin-down luminosity into synchrotron
emission is defined by efficiency factors η and
η LX/E  Typical values are η
and η (Becker \& Trumper 1997, Frail
\& Scharringhausen 1997), although wide excur- sions from this are
observed. Note that if the synchrotron lifetime of emitting particles is
a significant fraction of the PWN age (as is almost always the case at
radio wavelengths, and sometimes also in X-rays), then the PWN emission
represents an integrated history of the pulsas spin down,
and ηR and ηX are not true instantaneous efficiency factor'' --gaensler\_2006\_evolution-structure

X-ray energetics tabulated from: \url{http://arxiv.org/pdf/astro-ph/0006030v1.pdf}


\todo[inline]{Include discussion of cooling in Matanna et al 2009 (equation 12}

\todo[inline]{Look up scaling relationsips for IC and Sync
radiation from Adam Van Etten's thesis}

