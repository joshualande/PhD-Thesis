\section{\Actitle{PWN} Emission}
\seclabel{pwn_emission}

In the pulsar wind nebula, accelerated electrons emitted radiation across
the electromagnetic spectrum through synchrotron and \ac{IC} emission.
Typically characteristic photon energies for synchrotron and \ac{IC}
emission from \ac{PWN} are $\sim1\kev$ and $\sim1\tev$ respectively.
Typical magnetic field strengths are $\sim10\micro\gauss$.

Using \eqnref{characteristic_freqnecy_synctotron}, we can show that
photons with an energy $\energy_\kev$ in a magnetic field of strength
$\MagneticField_{-5}$ radiate electrons with a typical energy of
$E_\electron$ given by
\begin{equation}
  \energy_\electron \approx  70\unitspace\tev \MagneticField_{-5}^{-1/2} \energy_\kev^{1/2},
\end{equation}
where the magnetic field is $\MagneticField=10^{-5}
B_{-5}\gauss$ and $\energy_\kev$ is written in units of \kev
\citep{de-jager_2009a_implications-observations}.

Similarly, if we assume the \ac{PWN} \ac{IC} emission is due to scattering
off the \ac{CMB}, the electron energy which will produce characteristic
\tev $\gamma$-rays is:
\begin{equation}
  \energy_\electron \approx 20\unitspace\tev \energy_\tev^{1/2}
\end{equation}
where $\energy_\tev$ is the scattered photon energy in units of $\tev$
\citep{de-jager_2009a_implications-observations}.  This shows that for a
typical \ac{PWN}, $\sim70\unitspace\tev$ electrons power the synchrotron
emission and $\sim20\unitspace\tev$ electrons power the \ac{IC} emission.

Similarly, we can write down the livetime of electrons due to
synchrotron and \ac{IC} emission. We define the lifetime as $\lifetime =
\energy/\energydot$ and, using \cite{rybicki_1979a_radiative-processes},
we find
\begin{equation}
  \lifetime(\energy_\electron) = 
  \left(
  \tfrac{4}{3} \CrossSection_T \speedoflight (\EnergyDensity_\MagneticField + \EnergyDensity_\text{ph}) 
  \energy_\electron/ \mass_\electron^2\speedoflight^4
  \right)^{-1}
\end{equation}
where $\EnergyDensity_\MagneticField = \MagneticField^2/8\pi$
is the magnetic field energy density and
$\EnergyDensity_\text{ph}$ is the energy density of the photon field
($\EnergyDensity_\text{ph}=0.25\unitspace\electronvolt\unitspace\cm^{-3}$
for the \ac{CMB} radiation field).  If $B>3\micro\gauss$, the synchrotron
radiation dominates the cooling ($\EnergyDensity_\MagneticField >
\EnergyDensity_\text{ph}$.

For the synchrotron-emitting and \ac{IC}-emitting electrons.
For synchrotron-emitting electrons, the cooling time is
\begin{equation}
  \lifetime_\text{sync} = (1.2\unitspace\kyr) B_{-5}^{-3/2} \energy_\kev^{-1/2}.
\end{equation}
For \ac{IC}-scattering electrons, the cooling time (in the Thomson
limit) is
\begin{equation}
  \eqnlabel{ic_timescale}
  \lifetime_\text{IC} = (4.8\kyr) B_{-5}^{-2} \energy_\tev^{-1/2}
\end{equation}
From this, we see that the typical timescale for cooling of
synchrotron-emitting electrons ($1\unitspace\kyr$) is much shorter than
the timescale for cooling of \ac{IC}-emitting electrons ($5\kyr$). Because
of this, the \ac{IC}-emitting electrons have a longer time to diffuse
away from the pulsar and \ac{PWN}.  For older \ac{PWN}, we therefore
expected the observed \ac{VHE} emission to be larger than the observed
X-ray emission.  This has been observed in many \acp{PWN} such as
\hessj{1825} \cite{aharonian_2006a_h.e.s.s.-survey}, but also makes
identification of \ac{VHE} sources as \ac{PWN} more difficult.

We also note that \eqnref{ic_timescale} predicts that the timescale
of \ac{IC}-emitting electrons scales with inverse square root of
the emitted photon energy. From This leads to the prediction that
the size of the \ac{VHE} $\gamma$-ray emission should decrease
with increasing energy. This has been observed for \hessj{1825}
\citep{aharonian_2006a_energy-dependent}.

Finally, we mention that \cite{mattana_2009_evolution-gamma-} discussed
the relationship between the X-ray and $\gamma$-ray luminosity as a
function of the pulsar spin-down energy \energydot and age \PulsarAge.
The time integral of \eqnref{energy_dot_vs_time} can be used to compute
the total number of particles that emit synchrotron and \ac{IC} photons.

Most $\gamma$-ray emitting \ac{PWN} have pulsars have a characteristic
age of $\PulsarAge\sim1-20\unitspace\kyr$, so for most \ac{PWN}:
\begin{equation}
  \lifetime_\text{sync} < \PulsarAge < \lifetime_\text{IC}
\end{equation}
Therefore, for synchrotron emission the number of synchrotron-emitting
particles goes as
\begin{equation}
  n_\text{sync} \approx \energydot(\PulsarAge) \lifetime_\text{sync} \sim \energydot_0 \PulsarAge^{-2}
\end{equation}
where in the last step we have assumed a pure dipole magnetic field
($\breakingindex=3$).

On the other hand, the number of \ac{IC}-emitting particles
is approximately independent of time because $\PulsarAge \gg
\SpinDownTimescale$.  Furthermore, \cite{mattana_2009_evolution-gamma-}
argues that $n_\text{IC}$ should be independent of \energydot because it
more-strongly depends on other environmental factors.  Combining these
relations, \cite{mattana_2009_evolution-gamma-} proposes
\begin{equation}
  \luminosity_\text{IC}/\luminosity_\text{sync} \approx n_\text{IC}/n_\text{sync}
  \propto \PulsarAge^2 \propto \energydot^{-1}
\end{equation}
From observations of \ac{VHE} sources, they find empirically
that $\luminosity_\text{IC}/\luminosity_\text{sync} \propto
\PulsarAge^{2.2}$ and $\luminosity_\text{IC}/\luminosity_\text{sync}
\propto \energydot^{-1.9}$ which is qualitatively consistent with the
simple picture described above.  We will compare these scaling relations
with \ac{LAT} observations in \chapref{population_study}

