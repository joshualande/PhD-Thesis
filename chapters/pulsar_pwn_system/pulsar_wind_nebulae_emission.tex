\section{\Actitle{PWN} Emission}

In the pulsar wind nebula, accelerated electrons emitted radiation across
the electromagnetic spectrum through synchrotron and \ac{IC} emission.
Typically characteristic photon energies for Synctrotron and \ac{IC}
emission from \ac{PWN} are $\sim1\kev$ and $\sim1\tev$ respectivly.
Typical magnetic field strenghts are $\sim10\micro\gauss$.

Using \eqnref{characteristic_freqnecy_synctotron}, we can show that
photons with an energy $\energy_\kev$ in a magnetic field of strenth
$\MagneticField_{-5}$ radiate electrons with a typical energy of
$E_\electron$ given by
\begin{equation}
  \energy_\electron \approx  70\unitspace\tev \MagneticField_{-5}^{-1/2} \energy_\kev^{1/2},
\end{equation}
where the magnetic field is $\MagneticField=10^{-5}
B_{-5}\gauss$ and $\energy_\kev$ is written in units of \kev
\citep{de-jager_2009a_implications-observations}.  

Similarly, if we assume the \ac{PWN} \ac{IC} emission
is due to scattering off the \ac{CMB},
the electron energy which will produce characteristic
\tev $\gamma$-rays is:
\begin{equation}
  \energy_\electron \approx 20\unitspace\tev \energy_\tev^{1/2}
\end{equation}
where $\energy_\tev$ is the scattered photon energy in units
of $\tev$ \citep{de-jager_2009a_implications-observations}.  
This shows that for a typical \ac{PWN}, $\sim70\unitspace\tev$ electrons
power the synchrotron emission and $\sim20\unitspace\tev$ electrons
power the \ac{IC} emission.

Similarly, we can write down the livetime of electrons
due to Synctrotron and \ac{IC} emission. We define the lifetime as
$\tau = \energy/\energydot$ and, using the power formulas from
\cite{rybicki_1979a_radiative-processes}, we find
\begin{equation}
  \tau(\energy_\electron) = 
  \left(
  \tfrac{4}{3} \CrossSection_T \speedoflight (\EnergyDensity_\MagneticField + \EnergyDensity_\text{ph}) 
  \energy_\electron/ \mass_\electron^2\speedoflight^4)^{-1}
  \right)
\end{equation}
where $\EnergyDensity_\MagneticField = \MagneticField^2/8\pi$
is the magnetic field energy density and and
$\EnergyDensity_\text{ph}$ is the energy density of the photon field
($\EnergyDensity_\text{ph}=0.25\unitspace\electronvolt\unitspace\cm^{-3}$ for the
\ac{CMB} radiation field).

for the synchrotron-emitting and \ac{IC}-emitting electrons.
For synchrotron, the cooling time is
\begin{equation}
  \tau_\text{sync} = (1.2\unitspace\kilo\yearunit) B_{-5}^{-3/2} \energy_\kev^{-1/2}.
\end{equation}
For \ac{IC} scattering, the time scale is
\begin{equation}
  \tau_\text{IC} = 
\end{equation}


\todo[inline]{Describe Mattana's work on \glspl{PWN}: ``On the evolution of
the Gamma- and X-ray luminosities of Pulsar Wind Nebulae''}


Good stuff in ``anada\_2008a\_x-ray-study'' -- section 2.7.2 

Another good approximation for PWN energetics:

\url{http://arxiv.org/pdf/0803.0116v1.pdf}
``3 Energy scales and lifetimes of X-ray synchrotron and
VHE IC emitting electrons''

``Implications of H.E.S.S. observations of pulsar wind nebulae'' de
Jager \& Djannati-Atai

"As has been discussed, pulsar wind nebulae are prominent sources of
very high energy γ-rays. At these wavelengths, emission is caused
by inverse Compton acceler- ation of seed photons by the relativistic
electrons present in the nebula (see Section 1.3.5). As the electron
energies needed to boost photons to these energies are not as large as
that required for synchrotron emission to occur, inverse Compton emission
is seen at the extremes of the nebula where synchrotron emission can no
longer be observed and as such older pulsar wind nebulae are observed to
be much larger in VHE γ-rays than their X-ray counterparts. In addition
to this, inverse Compton emission is seen in the unshocked wind area of
the PWN as the electrons present are energetic enough to produce inverse
Compton radiation even if they are unable to produce synchroton radiation"
-- keogh\_2010\_search-pulsar



In the Pulsar/\ac{PWN} model, a population of electrons
is emitted from the magnetosphere, released as an unshocked wind,
and then accelerated at the termination shock. In the surrounding
nebula, the electrons emmit synchotron and inverse Compton

\begin{itemize}
  \item What is the characteristic synchotron energy
  \item What is the characteristic \ac{IC} energy
  \item look at: etten\_2012a\_particle-populations
\end{itemize}




Energetics of PWNe:

``The efficiency of conversion of spin-down luminosity into synchrotron
emission is defined by efficiency factors η and
η LX/E  Typical values are η
and η (Becker \& Trumper 1997, Frail
\& Scharringhausen 1997), although wide excur- sions from this are
observed. Note that if the synchrotron lifetime of emitting particles is
a significant fraction of the PWN age (as is almost always the case at
radio wavelengths, and sometimes also in X-rays), then the PWN emission
represents an integrated history of the pulsas spin down,
and ηR and ηX are not true instantaneous efficiency factor'' --gaensler\_2006\_evolution-structure

X-ray energetics tabulated from: \url{http://arxiv.org/pdf/astro-ph/0006030v1.pdf}


\todo[inline]{Include discussion of cooling in Matanna et al 2009 (equation 12}

\todo[inline]{Look up scaling relationsips for IC and Sync
radiation from Adam Van Etten's thesis}

