\section{Neutron Star Formation}

As was discussed in \secref{pulsars_and_pwn}, pulsars, 
\acp{PWN},
and supernova remnants are all the end products of supernovas.  When a
star undergoes a supernova, the ejecta forms a supernova rememant.  If the
remaining stellar core has a mass above the Chandrasekhar limit,
then the core's electron degeneracy pressure cannot counteract the core's
gravitational force and the core will collapse into a \ac{NS}.
The Chandrasekhar mass limit can be
approximated as \citep{chandrasekhar_1931_maximum-ideal}
\begin{equation}
  \MassChandrasekhar \approx 
  \frac{3\sqrt{2\pi}}{8}
  \left(\frac{\hbar\speedoflight}{\gravitationalconstant}\right)^{3/2}
  \left[
  \left(\frac{\NumberProtons}{\NumberNucleons}\right)
  \frac{1}{\MassHydrogen^2}
  \right]
\end{equation}
where $\hbar$ is the reduced Planck constant, \speedoflight is the
speed of light, \gravitationalconstant is the gravitational constant,
\MassHydrogen is the mass of hydrogen, \NumberProtons is the
number of protons, \NumberNucleons is the number of nucleons, and
\solarmass is the mass of the sun.  This formula can be found in
\citep{carroll_2006_introduction-modern}.
When this formula is computed more exactly, 
one finds $\MassChandrasekhar = 1.44 \solarmass$.

Because \acp{NS} are supported by a
neutron degeneracy pressure,
the radius of a neutron star can be approximated as
\begin{equation}
  \RadiusNeutronStar \approx \frac{(18 \pi)^{2/3}}{10}
  \frac{\hbar^2}{\gravitationalconstant \MassNeutronStar^{1/3}}
  \left(\frac{1}{\MassHydrogen}\right)^{8/3}
\end{equation}
This formula can be found in \citep{carroll_2006_introduction-modern}.
The canonical radius for \acp{NS} is $\sim 10\unitspace\km$.

In these very dense enviroments, the protons and electrons in the \ac{NS}
form into neutrons through inverse $\beta$ decay:
\begin{equation}
  \proton^\positive + \electron^\negative 
  \processarrow \neutron + \electronneutrino.
\end{equation}

But if a \ac{NS} had a sufficiency large mass, the graviational
force would overpower the neutron degeneracy pressure and the
object would collapse into a black hole. The maximum mass of a
\ac{NS} is unknown because it depends on the equation of state
inside the star, but is comonly predicted to be $\sim 2.5\solarmass$
Recently, a pulsar with a mass of $\sim 2\solarmass$ was discovered
\citep{demorest_2010_two-solar-mass-neutron}, constrainging
theories of the equation of state.

In addition to rotationally-powered pulsars, the primary class of
observed pulsars, there are two additional source classes of pulsars
with a different mechanism powering the source.  The first class is
accretion-powered pulsars, also called X-ray pulsars, are a bright
and populous source class at X-ray energies. In these sources, the
emission energy comes from the accretion of matter from a donor star. See
\cite{caballero_2012a_x-ray-pulsars:} for a review.  The second class
is magnetars which have a very strong magnetic field and a relativly
slow rotational perioid. In magnetars, the strong magnetic field powers
the emission.  See \cite{rea_2011a_magnetar-outbursts:} for a review.
