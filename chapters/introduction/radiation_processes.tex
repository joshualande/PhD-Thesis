\section{Radiation Processes in Gamma-ray Astrophysics}

\todo[inline]{XXXXXXXXXXXXXXXXXXXXXXXXXXXXXXXXXXXXXXXXXXXXXXXXXXXXXXXXXXXXXXXXXXXXXXXXXXXXXXXXXXXXXXXXXXXXXXXXXXXX}


\begin{itemize}
  \item The non-thermal radiation processes typical
    in astrophysics are most comonly
\end{itemize}

\subsection{Synchrotron}

\todo[inline]{Get image of Synchrotron radiation (from R\&L to include in discussion.}

The syncrotron radiation processes is commonly observed
in astrophysics. It is caused when charged particles,
typically electrons, spiral around magnetic field lines.

This emission is discussed thoroughly in
\cite{blumenthal_1970a_bremsstrahlung-synchrotron} and
\cite{rybicki_1979a_radiative-processes}.
In what follows, we adopt the notation from
\cite{houck_2006a_models-nonthermal}.

A charged particle of 
mass $\mass$ and charge $\charge$ in a magnetic field
of strength 
\MagneticFieldVector
will experince an
electromagnetic force:
\begin{align}
% equation 6.1a in R&L 1979
  \dbydt (\gamma m \VelocityVector) = & \frac{q}{c} \cross{\VelocityVector}{\MagneticFieldVector}
\end{align}
This force will cause a particle to accelerate around
the magnetic field lines, causing it
to radiate by maxwell's equations.

The power emitted at a frequency $\frequency$ 
by particle spiraling will be
\begin{equation}
% equation 6.33 in R&L 1979
  \eqnlabel{power_emitted_particle_sync}
  \power_\text{emitted}(\frequency) = 
  \frac{\sqrt{3} \charge^3 B \sin\alpha}{\mass \speedoflight^2} F(\frequency/\frequency_c)
\end{equation}
where $\alpha$ is the angle between the particle's velocity vector and
magnetic field vector.
Here,
% equation 6.31c in R&L 1979
\begin{equation}
  F(x) \equiv x \int_x^\infty K_{\tfrac{5}{3}} (\xi) d\xi,
\end{equation}
and
\begin{equation}
% equation 6.11a in R&L 1979
\frequency_c = \frac{3q \MagneticField \gamma^2}{4\pi \mass \speedoflight} 
\sin\alpha \equiv \nu_0 \gamma^2 \sin\alpha
\end{equation}

Because power is inversely-proportional to mass, synchotron radiation
is almost always assumed to come from electrons.

Now, we assume a population of particles and compute the total
emission. We say that $N(\momentum,\alpha)$ is the number of particles
per unit momentum and solid angle with a momentum $\momentum$ and
pitch angle $\alpha$.

We find the total power emitted by integrating over particle
momentum and distribution
\begin{equation}
  \frac{\derivative W}{\derivative\time}=
  \int \derivative \momentum 
  \int \derivative \solidangle
  \power_\text{emitted}(\frequency)
  N(\momentum,\alpha)
\end{equation}
If we assume the pitch angles of the particles to be isotropically
distributed and include \eqnref{power_emitted_particle_sync}, we
find that the photon emission per unit energy and time is
\begin{equation}
  \frac{\derivative n}{\derivative \omega \derivative \time} =
  \frac{\sqrt{3}q^3 B}{h m_e c^2 \angularfrequency}
  \int \derivative\momentum
  N(\momentum)
  R \left(\frac{\omega}{\omega_0 \gamma^2}\right)
\end{equation}
where
\begin{equation}
  R(x) \equiv \frac{1}{2} \int_0^\pi
  \derivative \alpha \sin^2 \alpha
  F\left(\frac{x}{\sin\alpha}\right)
\end{equation}

It is typical in astrophysics to assume a 
a power-law distribution of electrons written as
\begin{equation}
% equation 6.20a in R&L 1979
\eqnlabel{ElectronPowerLawEnergyDistribution}
  \ParticleDistribution(\momentum) \derivative\momentum = 
  \kappa \momentum^{-\spectralindex} \derivative\momentum.
\end{equation}
For a powerlaw distribution of photons integrated over
pitch angle, we find
\begin{equation}
\TotalPower(\angularfrequency) \propto \kappa \MagneticField^{(p+1)/2} 
\angularfrequency^{-(p-1)/2}.
\end{equation}
See, \cite{rybicki_1979a_radiative-processes} 
or \cite{longair_2013a_energy-astrophysics} for a full derivation.
This shows that, assuming a power-law electron distribution,
the electron spectral index can be related to the photon spectral
index.

\subsection{\Actitle{IC}}

Normal compoton scattering involves a photon colliding with a free electron
and transfering energy to it. In \ac{IC} scattering, a high-energy 
electron interacts with a low-energy photon imparting energy to it.
This process occurs when highly-energetic electrons interact with
a dense photon field.

The derivation of \Ac{IC} emission requires a quantum
electrodynamical treatment. It was first computed in
\cite{klein_1929a_streuung-strahlung}, and the derivation is described
in \cite{blumenthal_1970a_bremsstrahlung-synchrotron}.  In what follows, we follow
the notational convetion of \cite{houck_2006a_models-nonthermal}.

We assume a population of relativistic ($\gamma\gg1$) electrons
written as $N(\momentum)$ which is contained inside isotropic 
photon distribution with number density $n(\omega_i)$.

The distribution of photons 
emitted by \ac{IC} scatter is
written as
\begin{equation}
  \frac{\derivative\ParticleDistribution}{\derivative\omega\derivative\time} = 
  c \int d \omega_i n(\omega_i)
  \int_{p_\text{min}}^\infty  dp
  N(p) 
 \KleinNishinaCrossSection(\gamma,\omega_i,\omega)
\end{equation}
where $\omega$ is the outgoing photon energy written
in units of the electron rest mass energy $\omega\equiv h\nu/(m_e c^2)$
and $\KleinNishinaCrossSection$ is the Klein-Nishina cross section.

The Klein-Nishina cross section is
\begin{equation}
\KleinNishinaCrossSection(\gamma,\omega_i,\omega) = \frac{2\pi r_0^2}{\omega_i \gamma^2}
  \left[
  1 + q - 2q^2 + 2q\ln q + \frac{\tau^2 q^2 (1-q)}{2(1+\tau q)}
  \right]
\end{equation}
Here,
\begin{equation}
  q \equiv \frac{\omega}{4 \omega_i \gamma (\gamma-\omega)},
\end{equation}
$\tau \equiv 4\omega_i \gamma$, and $r_0 = e^2/(m_e c^2)$ is the classical
electron radius.
The threshold electron lorentz factor is
\begin{equation}
  \gamma_\text{min} =
  \frac{1}{2} 
  \left(
  \omega + \sqrt{\omega^2 + \frac{\omega}{\omega_i}}
  \right)
\end{equation}

Typically, \ac{IC} emision is assumed to originate when
a power-law distribution of electrons
(see \eqnref{ElectronPowerLawEnergyDistribution})
interacts with
a thermal photon distribution
\begin{equation}
  n(\omega_i) = 
  \frac{1}{\pi^2\lambda^3} 
  \frac{\omega_i^2}{e^{\omega_i/\Theta} -1}
\end{equation}
where $\lambda=\hbar/(m_e c)$ and $\Theta=kT/(m_e c^2)$.
Typically, \ac{IC} emission happens off CMB photons
with $T=2.725\unitspace\kelvin$.
We conclude by noting that the free-parameters of 
\ac{IC} emission are the the assumed particle spectrum
and photon field.

\subsection{Bremsstrahlung}

electron-electron bremsstrahlung.
electron-ion bremsstrahlung.

\begin{equation}
  \frac{\derivative n}{\derivative\omega\derivative\time} =
  n_e \int d\momentum N(p) v \frac{\derivative\simga_{ee}}{\derivative\omega} +
  n_Z \int d\momentum N(p) v \frac{\derivative\sigma_{eZ}}{\derivative\omega}
\end{equation}

Detailed clear desciption 
of the calcuation can be found in \cite{houck_2006a_models-nonthermal}
and references therein.

\todo[inline]{Why no Bremsstrahlung radiation from PWN.  Maybe a
back-of-the-eveolope estimate}

\subsection{Pi0 Decay}

\todo[inline]{Describe the characteristic pi0 cutoff energy}

