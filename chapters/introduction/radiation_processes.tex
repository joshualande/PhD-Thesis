\section{Radiation Processes in Gamma-ray Astrophysics}

\todo[inline]{XXXXXXXXXXXXXXXXXXXXXXXXXXXXXXXXXXXXXXXXXXXXXXXXXXXXXXXXXXXXXXXXXXXXXXXXXXXXXXXXXXXXXXXXXXXXXXXXXXXX}

\begin{itemize}
  \item The non-thermal radiation processes typical
    in astrophysics are most comonly
\end{itemize}

\subsection{Synchrotron}

Charged particles in magnetic fields experince an electromagnetic force:
% equation 6.1a in R&L 1979
\begin{align}
  \dbydt (\gamma m \VelocityVector) = & \frac{q}{c} \cross{\VelocityVector}{\MagneticFieldVector}
\end{align}

The energy output per unit frequency due to syncrotron radiation is
% equation 6.33 in R&L 1979
\begin{equation}
  P(\angularfrequency) = \frac{\sqrt{3} q^3 B \sin\alpha}{2\pi \mass \speedoflight^2} F(x)
\end{equation}

Where
% equation 6.31c in R&L 1979
\begin{equation}
  F(x) \equiv x \int_x^\infty K_{\tfrac{5}{3}} (\eta) d\eta
\end{equation}
and
\begin{equation}
x \equiv \omega/\omega_c
\end{equation}

and
% equation 6.4 in R&L 1979
\begin{equation}
  \omega_B = \frac{qB}{\gamma m c}
\end{equation}

We assume a particle distribution of electrons, written 
The electron spectrum is often
% equation 6.20a in R&L 1979
\begin{equation}
  N(\energy) \denergy = C \energy^{-\spectralindex} \denergy
\end{equation}

The total power output per unit frequency 

\begin{equation}
  P(\angularfrequency) = \int P(\angularfrequency) N(\energy) \denergy 
\end{equation}

\subsection{\Actitle{IC}}

\todo[inline]{Why don't protons syncrotron radiate?}

\Gls{IC} emission is \ldots


\subsection{Bremsstrahlung}

\todo[inline]{Why no Bremsstrahlung radiation from PWN.  Maybe a back-of-the-eveolope estimate}

\subsection{Pi0 Decay}

\todo[inline]{Describe the characteristic pi0 cutoff energy}

