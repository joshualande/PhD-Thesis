\section{Radiation Processes in Gamma-ray Astrophysics}

\todo[inline]{XXXXXXXXXXXXXXXXXXXXXXXXXXXXXXXXXXXXXXXXXXXXXXXXXXXXXXXXXXXXXXXXXXXXXXXXXXXXXXXXXXXXXXXXXXXXXXXXXXXX}


\begin{itemize}
  \item The non-thermal radiation processes typical
    in astrophysics are most comonly
\end{itemize}

\subsection{Synchrotron}

\todo[inline]{Get image of Synchrotron radiation (from R\&L to include in discussion.}

The syncrotron radiation processes is commonly observed
in astrophysics. It is caused when charged particles,
typically electrons, spiral around magnetic field lines.
This discussion of syncrotron radiation follows
the derivation in \cite{rybicki_1979a_radiative-processes}.

A charged particle of 
mass $\mass$ and charge $\charge$ in a magnetic field
of strength 
\MagneticFieldVector
will experince an
electromagnetic force:
\begin{align}
% equation 6.1a in R&L 1979
  \dbydt (\gamma m \VelocityVector) = & \frac{q}{c} \cross{\VelocityVector}{\MagneticFieldVector}
\end{align}
This force will cause a particle to accelerate around
the magnetic field lines, causing it
to radiate by maxwell's equations.
This radiation will have the following spectrum 
\begin{equation}
% equation 6.33 in R&L 1979
  \power(\angularfrequency) = \frac{\sqrt{3} \charge^3 B \sin\alpha}{2\pi \mass \speedoflight^2} F(x)
\end{equation}
Here,
% equation 6.31c in R&L 1979
\begin{equation}
  F(x) \equiv x \int_x^\infty K_{\tfrac{5}{3}} (\eta) d\eta,
\end{equation}
$x$ is proportional to the angular frequency as
\begin{equation}
x \equiv \angularfrequency/\angularfrequency_c,
\end{equation}
where
\begin{equation}
% equation 6.11a in R&L 1979
\angularfrequency_c = \frac{3}{2} \gamma^4 \angularfrequency_B \sin\alpha
\end{equation}
and $\angularfrequency_B$ is the gyration frequency:
\begin{equation}
% equation 6.4 in R&L 1979
  \angularfrequency_B = \frac{qB}{\gamma m c}.
\end{equation}

We now assume an an astrophysical object
as a population of electrons with an energy
distribution given by $\ParticleDistribution(\energy)$.
The total power output per unit frequency by this distribution
of electrons is
\begin{equation}
\TotalPower(\angularfrequency) = 
\int \power(\angularfrequency) \ParticleDistribution(\energy) \denergy 
\end{equation}

It is typical in astrophysics to assume a 
a power-law distribution of electrons written as
\begin{equation}
% equation 6.20a in R&L 1979
  \ParticleDistribution(\energy) \denergy = C \energy^{-\spectralindex} \denergy.
\end{equation}
For a powerlaw distribution of photons integrated over
pitch angle, we find 
\begin{equation}
\TotalPower(\angularfrequency) \propto C \MagneticField^{(p+1)/2} 
\omega^{-(p-1)/2}.
\end{equation}
See, \cite{rybicki_1979a_radiative-processes} 
or \cite{longair_2013a_energy-astrophysics} for a full derivation.
This shows that, assuming a power-law electron distribution,
the electron spectrum can be related to the photon spectrum 
for a given magnetic field.

\subsection{\Actitle{IC}}

``Inverse Compton scattering involves the scattering of low
energy photons to high energies by ultrarelativistic electrons
so that the photons gain and the electrons lose energy.'' --
\url{http://eud.gsfc.nasa.gov/Volker.Beckmann/school/download/Longair_Radiation3.pdf}

``In normal Compton scattering a photon
collides with a (quasi-)free electron and transfers to it a part if
its energy. In inverse Compton scattering the inverse process happens:
a photon interacts with an energetic electron and gains energy in the
collision. This process is important in regions with a high density of
photons and with the presence of hot electrons.'' -- rando\_2004\_glast-mission:

\todo[inline]{Why don't protons syncrotron radiate?}

\Gls{IC} emission is \todo[inline]{Describe IC emission...}.

You need a photon and an electron spectrum for \ac{IC} emission.

What about ``Klein-Nishina formula''
% This is equation 2.48 in Blumenthal & Gould 1970
\begin{equation}
  \frac{\derivative N_{\gamma,\epsilon}}{\derivative t \derivative E_1} =
  \frac{2\pi r_0^2 m c^3}{\gamma} 
  \frac{n(\epsilon) \derivative \epsilon}{\epsilon}
  \left[
  2q \ln q + (1+2q)(1-q) 
  + \frac{1}{2} \frac{(\tau_e q)^2}{1+\tau_e q} (1-q)
  \right]
\end{equation}

\todo[inline]{Compare formula to equation 9.44 in longair 2011 }

The total energy released by inverse compton from a population of electrons is
\begin{equation}
  \frac{\derivative N_\text{tot}}{\derivative t \derivative \epsilon_1}
 = 
 \int \int
 \derivative \gamma
 \derivative 
  \frac{\derivative N_{\gamma,\epsilon}}{\derivative t \derivative E_1} =
\end{equation}

\todo[inline]{Look up scaling relationsips for IC and Sync
radiation from Adam Van Etten's thesis}

\subsection{Bremsstrahlung}

\todo[inline]{Why no Bremsstrahlung radiation from PWN.  Maybe a back-of-the-eveolope estimate}

\subsection{Pi0 Decay}

\todo[inline]{Describe the characteristic pi0 cutoff energy}

