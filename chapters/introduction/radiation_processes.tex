\section{Radiation Processes in Gamma-ray Astrophysics}

\todo[inline]{XXXXXXXXXXXXXXXXXXXXXXXXXXXXXXXXXXXXXXXXXXXXXXXXXXXXXXXXXXXXXXXXXXXXXXXXXXXXXXXXXXXXXXXXXXXXXXXXXXXX}


\begin{itemize}
  \item The non-thermal radiation processes typical
    in astrophysics are most comonly
\end{itemize}

\subsection{Synchrotron}

\todo[inline]{Get image of Synchrotron radiation (from R\&L to include in discussion.}

The syncrotron radiation processes is commonly observed
in astrophysics. It is caused when charged particles,
typically electrons, spiral around magnetic field lines.
This discussion of syncrotron radiation follows
the derivation in \cite{rybicki_1979a_radiative-processes}.

A charged particle of 
mass $\mass$ and charge $\charge$ in a magnetic field
of strength 
\MagneticFieldVector
will experince an
electromagnetic force:
\begin{align}
% equation 6.1a in R&L 1979
  \dbydt (\gamma m \VelocityVector) = & \frac{q}{c} \cross{\VelocityVector}{\MagneticFieldVector}
\end{align}
This force will cause a particle to accelerate around
the magnetic field lines, causing it
to radiate by maxwell's equations.
This radiation will have the following spectrum 
\begin{equation}
% equation 6.33 in R&L 1979
  \power(\angularfrequency) = \frac{\sqrt{3} \charge^3 B \sin\alpha}{2\pi \mass \speedoflight^2} F(x)
\end{equation}
Here,
% equation 6.31c in R&L 1979
\begin{equation}
  F(x) \equiv x \int_x^\infty K_{\tfrac{5}{3}} (\eta) d\eta,
\end{equation}
$x$ is proportional to the angular frequency as
\begin{equation}
x \equiv \angularfrequency/\angularfrequency_c,
\end{equation}
where
\begin{equation}
% equation 6.11a in R&L 1979
\angularfrequency_c = \frac{3}{2} \gamma^4 \angularfrequency_B \sin\alpha
\end{equation}
and $\angularfrequency_B$ is the gyration frequency:
\begin{equation}
% equation 6.4 in R&L 1979
  \angularfrequency_B = \frac{qB}{\gamma m c}.
\end{equation}

We now assume an an astrophysical object
as a population of electrons with an energy
distribution given by $\ParticleDistribution(\energy)$.
The total power output per unit frequency by this distribution
of electrons is
\begin{equation}
\TotalPower(\angularfrequency) = 
\int \power(\angularfrequency) \ParticleDistribution(\energy) \denergy 
\end{equation}

It is typical in astrophysics to assume a 
a power-law distribution of electrons written as
\begin{equation}
% equation 6.20a in R&L 1979
\eqnlabel{ElectronPowerLawEnergyDistribution}
  \ParticleDistribution(\energy) \denergy = C \energy^{-\spectralindex} \denergy.
\end{equation}
For a powerlaw distribution of photons integrated over
pitch angle, we find 
\begin{equation}
\TotalPower(\angularfrequency) \propto C \MagneticField^{(p+1)/2} 
\omega^{-(p-1)/2}.
\end{equation}
See, \cite{rybicki_1979a_radiative-processes} 
or \cite{longair_2013a_energy-astrophysics} for a full derivation.
This shows that, assuming a power-law electron distribution,
the electron spectrum can be related to the photon spectrum 
for a given magnetic field.

\subsection{\Actitle{IC}}

Normal compoton scattering involves a photon colliding with a free electron
and transfering energy to it. In \ac{IC} scattering, a high-energy 
electron interacts with a low-energy photon imparting energy to it.
This process occurs when a highly-energetic electrons interact with
a dense photon field.

The derivation of \Ac{IC} emission requires a quantum
electrodynamical treatment. It was first computed in
\cite{klein_1929a_streuung-strahlung}, and the derivation is described
in \cite{blumenthal_1970a_bremsstrahlung-synchrotron}.  In what follows, we follow
the notational convetion of \cite{houck_2006a_models-nonthermal}.

We assume a population of relativistic ($\gamma\gg1$) electrons
written as $N(\momentum)$ which is inside of an isotropic distribution
of photons with a number density $n(\omega_i)$.
The distribution of emitted photons can be written as
\begin{equation}
  \frac{\derivative\ParticleDistribution}{\derivative\omega\derivative\time} = 
  c \int d \omega_i n(\omega_i)
  \int_{p_\text{min}}^\infty  dp
  N(p) 
 \KleinNishinaCrossSection(\gamma,\omega_i,\omega)
\end{equation}
where $\omega$ is the outgoing photon energy written
in units of the electron rest mass energy $\omega\equiv h\nu/(m_e c^2)$.
and $\KleinNishinaCrossSection$ is the Klein-Nishina cross section.

The Klein-Nishina cross section is
\begin{equation}
\KleinNishinaCrossSection(\gamma,\omega_i,\omega) = \frac{2\pi r_0^2}{\omega_i \gamma^2}
  \times 
  \left[
  1 + q - 2q^2 + 2q\ln q + \frac{\tau^2 q^2 (1-q)}{2(1+\tau q}
  \right]
\end{equation}
Here,
\begin{equation}
  q \equiv \frac{\omega}{4 \omega_i \gamma (\gamma-\omega)},
\end{equation}
$\tau \equiv 4\omega_i \gamma$, and $r_0 = e^2/(m_e c^2)$ is the classical
electron radius.
The threshold electron lorgenz factor is
\begin{equation}
  \gamma_\text{min} =
  \frac{1}{2} 
  \left(
  \omega + \sqrt{\omega^2 + \frac{\omega}{\omega_i}}
  \right)
\end{equation}

Typically, \ac{IC} emision is assumed to originate when
a power-law distribution of electrons
(see \eqnref{ElectronPowerLawEnergyDistribution})
interacts with
a thermal photon distribution
\begin{equation}
  n(\omega_i) = 
  \frac{1}{\pi^2\lambda^3} 
  \frac{\omega_i^2}{e^{\omega_i/\Theta} -1}
\end{equation}
where $\lambda=\hbar/(m_e c)$ and $\Theta=kT/(m_e c^2)$.

\todo[inline]{Look up scaling relationsips for IC and Sync
radiation from Adam Van Etten's thesis}

\subsection{Bremsstrahlung}

\todo[inline]{Why no Bremsstrahlung radiation from PWN.  Maybe a back-of-the-eveolope estimate}

\subsection{Pi0 Decay}

\todo[inline]{Describe the characteristic pi0 cutoff energy}

