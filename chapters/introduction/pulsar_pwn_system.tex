\section{The Pulsar/\Acptitle{PWN} System}
\seclabel{pulsar_pwn_system}

As was discussed in \secref{astrophysical_sources}, pulsars, 
\acp{PWN},
and supernova remnants are all the end products of supernovas.  When a
star undergoes a supernova, the ejecta forms a supernova rememant.  If the
remaining stellar core has a mass above the Chandrasekhar limit,
then the core's electron degeneracy pressure cannot counteract the core's
gravilational force and the core can collapse into a neutron star.
The Chandrasekhar mass limit is written as \citep{chandrasekhar_1931_maximum-ideal}
\begin{equation}
  \MassChandrasekhar \approx \left(\frac{\hbar\speedoflight}{\gravitationalconstant}\right)^{\tfrac{2}{3}}
  \frac{1}{\protonmass^2}
  = 1.44 M
\end{equation}
where $\hbar$ is the reduced Planck constant, \speedoflight
is the speed of light, \gravitationalconstant is the gravitational
constant, and \protonmass is the proton mass.

\todo[inline]{If the star is much bigger, then it will collapse into 
black hole}
\todo[inline]{pulsars are heild together by neutron degeneracy pressure}
\todo[inline]{Typical radius/density of neutron star}

\todo[inline]{3 Types of pulsars (rotationalally powered pulsars,
magnetars, accretion powered pulsars,}

\subsection{Pulsar Properties}

\todo[inline]{XXXXXXXXXXXXXXXX}

% Good discussion of PWN physics with some simple formulas: 
%   ``The Evolution and Structure of Pulsar Wind Nebulae''
%   Bryan M. Gaensler and Patrick O. Slane


% good discussion of radius of termination shock in: kargaltsev_2008_pulsar-nebulae

% Another review paper (looks to describe pulsar modeling):
%  ``High Energy Studies of Pulsar Wind Nebulae''
%     Patrick Slane

% Also, Adam Van Etten's thesis: 1.1 Pulsar Wind Nebula Structure

% Good reference: ``Carroll and Ostlie'' page 593

% good refernce on PWN physics: \cite{fuste_2007_g-ray-observations}

\begin{itemize}
  \item ``Following this discovery, a theoretical understanding was
  soon developed in which the central pulsar generates a magnetized
  particle wind, whose ultrarelativistic electrons and positrons radiate
  synchrotron emission across the electromagnetic spectrum (Pacini \&
  Salvati 1973, Rees \& Gunn 1974). The pulsar has steadily released
  about a third of its total reservoir of ???? ergs
  of rotational energy into its surrounding nebula over the last 950
  years. This is in sharp contrast to shell-like SNRs, in which the
  dominant energy source is the ??? ergs of kinetic energy
  released at the moment of the original SN explosion.''  -- \cite{gaensler_2006_evolution-structure}
  \item 
    % described in Adam Van Etten's thesis
    What is terminaltion shock of \glspl{PWN}.
  \item 
    How is pulsar outflow accelerated at shock?
\end{itemize}


\todo[inline]{Find a plot of a rotating pulsar. The simplest
rotating dipole model}

The energy powering pulsars and \glspl{PWN} is comonly
believed to originate in rotational kintetic energy stored in
the netutron star. 
Both the period \period and the period derivative
$\perioddot=d\period/d\time$ can be directly observed for a pulsar
and typically the pulsar is slowing down (\perioddot<0).
We write the rotational kinetic energy as
\begin{equation}
  \energyrotational = \tfrac{1}{2} I \angularfrequency^2
\end{equation}
where $\omega = 2\pi/\period$ and \period is the period of
rotation of the pulsar. We make the conection between
the pulsar's spin-down energy and the rotational kintecit energy as
$\energydot = - \derivative\energyrotational/\dtime$

The rotational kinetic energy in a pulsar can be written as
\begin{equation}
  \energydot = 4\pi^2 \momentofinertia \perioddot/\period^2,
\end{equation}
where \momentofinertia is the moment of inertia.
For a uniform sphwere,
\begin{equation}
  \momentofinertia = \frac{2}{5} M R^2
\end{equation}
Pulsars are assumed to be uniform spheres with $R=10\unitspace\km$
and $M=1.4\solarmass$, which leads to a canonical moment of inertia of
$\momentofinertia=10^{45}\unitspace\gram\cm^{-2}$.

It is believed that as the pulsar spins down, the this rotational energy
is released as pulsed electromagnetic radiation and also as a wind of
electrons and positrons accelerated in the magnetic field of the pulsar.

As the pulsar slows down, it released the rotational kinetic energy
\begin{equation}
  \energydot = - \frac{4 \pi^2 \momentofinertia \perioddot}{\period^3}
\end{equation}
where \perioddot is the rate of decrease in the period of the pulsar.

\todo[inline]{Discuss uniform dipole model}

We conventionally assume that the period and period derivative are related
by the equation
\begin{equation}\eqnlabel{angular_frequency_derivative_relation}
  \angularfrequencydot \propto \angularfrequency^\breakingindex
\end{equation}
where $\angularfrequency=2\pi/\period$, and $\breakingindex$
is the pulsar breaking index. 

\begin{itemize}
  \item ``though n has only been confidently measured for five pulsars,
  in each case falling in the range 2 < n < 3 (Livingstone et al. (2007)
  and references therein).'' -- Adam Van Etten's thesis
\item ``The braking index is the power to which the slowdown in angular
velocity occurs, and is defined as:
  \begin{equation}
    n = \frac{\period \perioddotdot}{\perioddot} + 2
  \end{equation}
  '' -- keogh\_2010\_search-pulsar
\end{itemize}

Equation \eqnref{angular_frequency_derivative_relation} is
a Bernoulli differential equation which can be integrated to solve for time

\begin{equation}
  T = \frac{\period}{(\breakingindex-1) |\perioddot|}
  \left(
  1-\left(\frac{\period_0}{\period}\right)^{(\breakingindex-1)}
  \right)
\end{equation}

\begin{itemize}
  \item TODO: cite Manchester \& Taylor 1977
\end{itemize}



\begin{equation}
  \pulsarage = \period/2\perioddot
\end{equation}


\begin{itemize}
  \item What is a typical moment of inertia, typical dE/dT
  \item What fraction of pulsar energy is released observationally
\end{itemize}

\subsection{Pulsar Magnetosphere}

\subsection{Pulsar Wind Nebulae}

\begin{itemize}
  \item Discuss termination shock (i.e. section 3.3.2 of dalton\_2011\_identication-gamma-ray
  \item The radius of the termination shock is
    \begin{equation}
      \radiusterminationshock = \sqrt{\frac{\energydot}{\tfrac{4}{3}\pi \pressureISM \speedoflight}}
    \end{equation}

\end{itemize}

\todo[inline]{Discuss pulsar evolution ``The Evolution and Structure of
Pulsar Wind Nebulae'' -- Bryan M. Gaensler and Patrick O. Slane}

\todo[inline]{Describe Mattana's work on \glspl{PWN}: ``On the evolution of
the Gamma- and X-ray luminosities of Pulsar Wind Nebulae''}

