\section{Astrophysical Sources of Gamma-rays}
\seclabel{astrophysical_sources}

\subsection{Pulsars}

It is widely accepted that in the collapse of a massive star, a large
amount of ejecta is released as a supernova powering a \ac{SNR}
and that much of the remaining mass collapses into a neutron star
\citep{baade_1934a_remarks-super-novae}.

Pulsars were first discovered observationally in 1967 by Jocelyn Bell
Burnell and Antony Hewish \citep{hewish_1968_observation-rapidly}. They
had constructed a radio telescope that used interplanetary scintillation
wiht the intention of observing quasars.  In the process, they
detected a source with a periodicity of 1.3 \second. We note in
passing that pulsars had been previously observed by the air force
\citep{brumfiel_2007_force-early}.

Even before the discovery, \cite{pacini_1967_energy-emission}
had predicted the existence of \acp{NS}.  Shortly following
the 1967 discovery, \cite{gold_1968_rotating-neutron} and
\cite{pacini_1968_rotating-neutron} argued that the observed pulsar was
a rotating \ac{NS}.

The discovery of many more pulsars came quickly.  In 1968, and the
Vela pulsar \citep{large_1968_pulsar-supernova} and the Crab pulsar
\citep{staelin_1968_pulsating-radio} were discovered.

The first pulsar observed at optical frequencies was the
Crab, discovered in 1969 shortly after its radio discovery
\citep{cocke_1969_discovery-optical}.  In the same year, the first X-ray
pulsations were discovered from the same source. At the time, there were
no space-based X-ray observatories, so observations had to be performed
from rockets.  The discovery was carried out almost concurrently by
a group at \gls{NRL} \citep{fritz_1969_x-ray-pulsar} and at \gls{MIT}
\citep{bradt_1969_x-ray-optical}.  Using proportional counters, these
experiments showed that the pulsed emission from the Crab extended to
X-ray energies and that, for this source, the X-rays emission was a
factor $>100$ more energetic than the observed visible emission.

From these early sources, pulsar physics has blossomed into a vast
field. In the on-line \ac{ATNF} catalog, there are currently over 2,200
pulsars \citep{manchester_2005a_australia-telescope}.

As was discussed in \secref{history_gamma_ray_detectors},
the first pulsar was observed in $\gamma$-ray in 1970
\citep{kniffen_1970_study-gamma}.  Observations by \ac{EGRET}
broguth the total number of $\gamma$-ray-detected pulsars to six
\citep{nolan_1996a_egret-observations}.  \fermi has vastly expanded the
nubmer of pulsars detected in $\gamma$-rays and we will discuss these
observations in \seclabel{second_pulsar_catalog}

\subsection{\Acptitle{PWN}}

% Good history of observations of crab nebulae:
%  * chapter 1 of "The Crab Nebula" by Rodney Deane Davies

% Good history paper: ``Pulsar Wind Nebulae:
%   On their growing diversity and association
%   with highly magnetized neutron stars''
%   by Samar Safi-Harb
%   -- http://arxiv.org/pdf/1211.0852.pdf

% Good paper on Crab Nebula: ``{The Crab Nebula: An Astrophysical Chimera}'' by
%   J. Jeff Hester

% Good review of PWN: 
%   ``The Evolution and Structure of Pulsar Wind Nebulae''
%   Bryan M. Gaensler and Patrick O. Slane

% Good review of PWN:
%  ``{Pulsar Wind Nebulae in the Chandra Era}''
%  O. Kargaltsev and G. G. Pavlov


% lots of INFO on M1 (crab): http://messier.seds.org/m/m001.html


A \gls{PWN} is a diffuse nebula of shocked relativistic particles
that surrounds and is powered by an accompanying pulsar. 
\glspl{PWN} have been observed long before the discovery of pulsars, but
the pulsar/\gls{PWN} connection was not made until
after the detection of pulsars.

The most famous \glspl{PWN} is the Crab nebula, associated with the Crab
pulsar.  The Crab \ac{SN} (SN 1054), which was observed by chinese astrologers 
in 1054 AD \cite{hester_2008_nebula:-astrophysical}.
It was also likely observed in
Japan, europe, by native americans,
and and in the arab world 
\citep[see][and references therein]{collins_1999a_reinterpretation-historical}.

\begin{figure}[htbp]
  \centering
  \includegraphics[width=\textwidth]{chapters/introduction/figures/bevis_crab.pdf}
  \figlabel{bevis_crab}
  \caption{The Orion plate from Bevis' book {\em Uranographia Britannica}.
  The Crab nebula can be found on the horn of Taurus the Bull 
  on the top of the figure and the source is marked by a 
  cloudy symbol.
  This figure was reproduced from \cite{ashworth_1981_bevis-uranographia}.}
\end{figure}

The Crab nebula, in the remains of SN 1054,
was first discovered in 1731 by physician and amateur astronomer
John Bevis.  This source was going to be published in his sky atlas
{\em Uranographia Britannica}, but the work was never publisehd because
his published filed for bankrupcy in 1950.  Figure~\figref{bevis_crab}
shows Beavis' plate containing the Crab nebula.  A detailed history of
John Bevis' work can be found in \cite{ashworth_1981_bevis-uranographia}.
The Crab Nebulae was famously included in Charles Messier's catalog as
M1 in 1758 \cite{hester_2008_nebula:-astrophysical}.
In 1921,
Knut Lundmark proposed a connection etween the Crab Nebula and the 1054 supernova.

\cite{lundmark_1921a_suspected-stars}

\begin{itemize}
  \item 
    ``The Crab Nebula (Fig. 1) is almost certainly associated with a
    supernova (SN) ex- plosion observed in 1054 CE (Stephenson \& Green
    2002, and references therein).'' -- ``The Evolution and Structure of Pulsar Wind Nebulae'' 
    Bryan M. Gaensler and Patrick O. Slane
  \item ``The same year, J.C. Duncan of Mt. Wilson Observatory compared
    photographic plates taken 11.5 years apart, and found that the
    Crab Nebula was expanding at an average of about 0.2'' per year;
    backtracing of this motion showed that this expansion must have
    begun about 900 years ago (Duncan 1921). Also the same year, Knut
    Lundmark noted the proximity of the nebula to the 1054 supernova
    (Lundmark 1921).`` -- http://messier.seds.org/m/m001.html
    ``In 1942, based on investigations with the 100-inch Hooker telescope
    on Mt. Wilson, Walter Baade computed a more acurate figure of 760
    years age from the expansion, which yields a starting date around
    1180 (Baade 1942); later investigations improved this value to about
    1140. The actual 1054 occurrance of the supernova shows that the
    expansion must have been accelerated.'' -- http://messier.seds.org/m/m001.html
  \item ``but it was not until 1942 that Duyvendak (1942) and Mayall \&
    Oort (1942) presented complete studies of modern observations of the
    expanding nebula and of the early Chinese records. It was this work
    that established unambiguously that the Crab is the remnant of SN1054.''
    (p128 of ``The Crab Nebula: An Astrophysical Chimera'')
  \item ``1949, the Crab nebula was identified as a strong source
    of radio radiation (Bolton et.al. 1949), discovered 1948 named
    and listed as Taurus A (Bolton 1948), and later as 3C 144.'' -
    http://messier.seds.org/m/m001.html
  \item Synchrotron emission hypotehsis: ``while the inner, blueish
      nebula emits continuous light consisting of highly polarised so-called
      synchrotron radiation, which is emitted by high-energy (fast moving)
      electrons in a strong magnetic field. This explanation was first
      proposed by the Soviet astronomer J. Shklovsky (1953) and supported
      by observations of Jan H. Oort and T. Walraven (1956).'' -- http://messier.seds.org/m/m001.html
  \item ``X-rays from this object were detected in April 1963 with a
      high-altitude rocket of type Aerobee with an X-ray detector developed
      at the Naval Research Laboratory; the X-ray source was named Taurus
      X-1. Measurements during lunar occultations of the Crab Nebula on
      July 5, 1964, and repeated in 1974 and 1975, demonstrated that the
      X-rays come from a region at least 2 arc minutes in size, and the
      energy emitted in X-rays by the Crab nebula is about 100 times more
      than that emitted in the visual light''
        -- http://messier.seds.org/m/m001.html
  \item Association of Crab pulsar with Crab nebulae were 
    discussed in \citep{staelin_1968_pulsating-radio}.
  \item
    "Just after discovery of pulsars, Gunn \& Ostriker (1969) suggested
    that particles can be accelerated to very high energies in the pulsar
    wind zone." -- "Relativistic Astrophysics and Cosmology" by Shapiro
  \item TEV observations of Crab. brightest soruce in the sky\ldots
  \item Now many radio, x-ray PWNe. Count form PWN catalog:
    http://www.physics.mcgill.ca/~pulsar/pwncat.html
    ``Observations over the last several decades have identified 40 to 50
    further sources, in both our own Galaxy and in the Magellanic Clouds,
    with properties similar to those of the Crab Nebula (Green 2004;
    Kaspi, Roberts \& Harding 2006)'' -- \cite{gaensler_2006_evolution-structure}
  \item How many TeV PWNe in TeVCat? http://tevcat.uchicago.edu/ (31 by my preliminary count)
\end{itemize}




