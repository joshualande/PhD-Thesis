\section{Astronomy and the Atmosphere}
\seclabel{astronomy_and_the_atmosphere}

From the very beginning, humans have surely stared into space and
contemplated its brilliance.  Stone circles in the Nabta Playa in
Egypt are likely the first observed astronomical observatory and
are believed to have acted as a prehistoric calendar.  Dating back
to the 5th century BC, they are 1,000 years older than Stonehenge
\citep{mck-mahille_2007_astronomy-nabta}.

Historically, the field of astronomy concerned the study of visible light
because it is not significantly absorbed in the atmosphere.  But slowly,
over time, astronomers expanded their view across the electromagnetic
spectrum.  Infrared radiation from the sun was first observed by William
Herschel in 1800 \citep{herschel_1800_experiments-refrangibility}.
The first extraterrestrial source of radio waves was detected
by Jansky in 1933 \citep{jansky_1933_electrical-disturbances}.
The expansion of astronomy to other wavelengths required the
development of rockets and satellites in the 20th century.
The first ultraviolet observation of the sun was performed in 1946
from a captured V-2 rocket \citep{baum_1946_ultraviolet-spectrum}.
Observations of x-rays from the sun were first performed in 1949
\citep{burnight_1949_x-radiation-atmosphere}.
