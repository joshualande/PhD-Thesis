
\chapter{Gamma-ray Astrophysics}
\chaplabel{gamma_ray_astro}

\section{The History of Gamma-ray Detectors}
\subsection{The \fermi Gamma-ray Space Telescope}
\subsection{H.E.S.S.}

\section{Galactic Gamma-ray Astrophysics}
\subsection{Pulsars}

\todo[inline]{Describe pulsar physics. See description from Carroll and Ostlie page 593}
\begin{equation}
  \energydot = 
  4\pi^2 \momentofinertia \perioddot/\period
\end{equation}

\begin{equation}
  \pulsarage = \period/2\perioddot
\end{equation}


\subsection{Pulsar Wind Nebulae}
\subsection{Supernova Remenants}

\section{Radiation Processes}

\begin{itemize}
  \item The non-thermal radiation processes typical
    in astrophysics are most comonly
\end{itemize}

\subsection{Synchrotron}

\subsection{\acl{IC}}

\ac{IC} emission is \ldots

\subsection{Bremsstrahlung}

\subsection{Pi0 Decay}

\section{Modeling the Galactic Diffuse and Isotropic Gamma-ray Background}
\seclabel{modeling_background}

\begin{itemize}
  \item Historical Observations of galactic diffuse emission
  \item GALPROP model of diffuse emission.
  Reference: \url{http://arxiv.org/abs/1202.4039}
  \item Emperical Ring model of galactic diffuse emisson.
  \item The isotropic background: \url{http://arxiv.org/abs/1002.3603}
\end{itemize}

\begin{itemize}
  \item Galactic diffuse emission is primarily composed of \ldots
  \item Something about how great galprop is.
  \item Something about
\end{itemize}

\section{Sources Detected by the Fermi \acs{LAT}}

\begin{itemize}
  \item A variety of sources detected by the \acs{LAT}:
\end{itemize}

\subsection{\acf{2FGL}}

\ac{2FGL} was a catalog by the LAT collaboration containing XXX Sources.
\todo[inline]{Describe Catalog}

\begin{itemize}
  \item Citation is \cite{second_lat_catalog_2012}
  \item Source classification method
  \item Number of sources detected by the \ac{LAT}
  \item Forward reference \chapref{maximum_likelihood_analysis},
    which does a more thorough description of likelihood analysis method.
  \item Source classes/associations
\end{itemize}

\subsection{The Second Fermi Pulsar Catalog}

\begin{itemize}
  \item Process of detecting Pulsars with the \ac{LAT}
  \item Number of pulsars detected by the \ac{LAT}
\end{itemize}

\subsection{\acsp{PWN} Detected by the \ac{LAT}}

\subsubsection{Crab}

\subsubsection{Vela X}

\subsubsection{MSH 15-52}

\subsubsection{HESS J1825}

\subsubsection{HESS J1857}

\twofgl{1857}{+026}

\begin{enumerate}
  \item Reference is \cite{hess_j1857_lat_2012}
  \item \url{http://arxiv.org/pdf/1206.3324v1.pdf}
\end{enumerate}
