
\chapter{Introduction}

\section{Gamma-ray Detectors}
\subsection{The \fermi Gamma-ray Space Telescope}
\subsection{H.E.S.S.}

\section{Galactic Gamma-ray Astrophysics}
\subsection{Pulsars}
\subsection{Pulsar Wind Nebulae}
\subsection{Supernova Remenants}

\section{Radiation Processes}

\begin{itemize}
  \item The non-thermal radiation processes typical
    in astrophysics are most comonly
\end{itemize}

\subsection{Synchrotron}

\subsection{Inverse Compton}

\subsection{Bremsstrahlung}

\subsection{Pi0 Decay}

\section{Sources Detected by the Fermi LAT}

\subsection{Modeling the galactic background}

\begin{itemize}
  \item Galactic diffuse emission is primarily composed of \ldots
  \item Something about how great galprop is.
  \item Something about
\end{itemize}

\subsection{The Second Fermi-LAT Source Catalog}

\begin{itemize}
  \item Citation is \cite{second_lat_catalog_2012}
  \item Source classification method
  \item Number of sources detected by the LAT
  \item Forward reference \chapref{maximum_likelihood_analysis},
    which does a more thorough analysis.
\end{itemize}

\subsection{The Second Fermi Pulsar Catalog}

\begin{itemize}
  \item Process of detecting Pulsars with the LAT
  \item Number of pulsars detected by the LAT
\end{itemize}

\subsection{PWN Detected by the LAT}

\subsubsection{Crab}

\subsubsection{Vela X}

\subsubsection{MSH 15-52}

\subsubsection{HESS J1857}

\twofgl{1857}{+026}

\begin{enumerate}
  \item \url{http://arxiv.org/pdf/1206.3324v1.pdf}
\end{enumerate}
