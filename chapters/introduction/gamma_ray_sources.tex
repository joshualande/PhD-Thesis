
\subsection{Pulsar Wind Nebulae}

% Good history of observations of crab nebulae:
%  * chapter 1 of "The Crab Nebula" by Rodney Deane Davies

% good history paper: http://arxiv.org/pdf/1211.0852.pdf

\todo{First nebulae?}


\section{Astrophysical Sources of Gamma-ray}
\subsection{Pulsars}

% good details from "Pulsar Astronomy" By Andrew G. Lyne, Francis Graham-Smith


Pulsars were first discovered in 1967 by Jocelyn Bell Burnell and Antony
Hewish \citep{hewish_1968_observation-rapidly}. They had constructed a
radio telescope that used interplanetary scintillation wiht the intention
of observing quasars.  In the process, they detected a source with a
periodicity of 1.3 \second.

Even before the discovery, \cite{pacini_1967_energy-emission} had predicted
the existence of neutron stars.  Shortly following the 1967 discovery,
\cite{gold_1968_rotating-neutron} and \cite{pacini_1968_rotating-neutron}
argued that the observed pulsar was a rotating neutron star.

The discovery of many more pulsars came quickly.  In 1968, and the
Vela pulsar \citep{large_1968_pulsar-supernova} and the Crab pulsar
\cite{staelin_1968_pulsating-radio} were discovered.

% Info came from ``Pulsar Astronomy'' by Lyne.
The first pulsar observed at optical frequencies was the
Crab, discovered in 1969 shortly after its radio discvoery
\citep{cocke_1969_discovery-optical}.

In the same year, the first X-ray pulsations were discovered from
the same source. At the time, there were no space-based X-ray
observatories, so observations had to be performed from rockets.
The discovery was carried out almost concurrently by a group
at \ac{NRL} \citep{fritz_1969_x-ray-pulsar} and at \ac{mit}
\citep{bradt_1969_x-ray-optical}.  and

\todo{First gamma-ray detection}

ATNF catalog?


\todo[inline]{When was the PSR, PWN connection made}

\todo[inline]{Describe pulsar physics. See description from Carroll and Ostlie page 593}
\begin{equation}
  \energydot = 
  -4\pi^2 \momentofinertia \perioddot/\period^3
\end{equation}

\begin{equation}
  \pulsarage = \period/2\perioddot
\end{equation}



\subsection{Supernova Remenants}

