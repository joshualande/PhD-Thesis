
\section{The History of Gamma-ray Astrophysics}
\seclabel{history_gamma_ray_detectors}

% Thorough discussion of all experiments:
%   http://imagine.gsfc.nasa.gov/docs/sats_n_data/gamma_missions.html
%   http://space.about.com/od/telescopesandoptics/a/Gammaray-Astronomy.htm

% Nice history with lots of pictures:
%  http://fermi.gsfc.nasa.gov/science/mtgs/symposia/2012/program/mon/DKniffen.pdf

% Good books with introduction

%  "Very High Energy Gamma Ray Astronomy" by Trevor C. Weekes
%    * seems to have a discussion of the principles of Gamma-ray detection.
%      scintilation detectors, spark chambers.

%  "Cosmic Gamma-Ray Sources" - edited by K.S. Cheng, Gustavo E. Romero

% Very nice reference:
%   http://adsabs.harvard.edu/abs/1984BASI...12..202P

% Nice summary:
%   http://articles.adsabs.harvard.edu/cgi-bin/nph-iarticle_query?1984BASI...12..202P&amp;data_type=PDF_HIGH&amp;whole_paper=YES&amp;type=PRINTER&amp;filetype=.pdf

Astronomy has historically been almost entirely concerned with studying
the photons that arrive from outer space.  Because of their charge
neutrality, photons are not defected by intergalactic electric and
magnetic fields and therefore point back to the objects 
emitting them. Historically, the field of astronomy concerned
the study of visible light.  Slowly, over time, astronomers 
expanded their view across the electromagnetic spectrum.

Infrared radiation from the sun was first observed by
William Herschel in 1980 \citep{herschel_1800_experiments-refrangibility}.
The first extraterrestrial source of radio waves was detected
by Jansky in 1933 \citep{jansky_1933_electrical-disturbances}.

% x-ray: http://en.wikiversity.org/wiki/Radiation_history#cite_ref-Burnight_183-1
The development of rockets and sattelites in the 20th ceuntry allowed
the field of astronomy to expand futher, allowering observations
at wavelengths that would otherwise be absorbed in the atmosphere.
The first ultraviolet observation of the sun was performed in 1946 from a
captured V-2 rocket \citep{baum_1946_ultraviolet-spectrum}.  Observations of
solar x-rays were also first carried out on a captured V-2 Rocket in
1949 \citep{burnight_1949_x-radiation-atmosphere}

It was only natural to wonder about the universe at even higher energies.
As is common in the field of physics, the prediction of
the detection of cosmic $\gamma$-rays far proceded their discovery.
\cite{feenberg_1948_interaction-cosmic-ray} theorized that the interaction
of starlight with cosmic rays could produce $\gamma$-rays through
\ac{IC} upscattering.  Following the discovery of the neutral
pion in 1949, \cite{hayakawa_1952_propagation-cosmic}
predicted that $\gamma$-ray emission could be observed from the
decay of neutral pions when cosmic rays interacted with interstellar
matter.  And in the same year, \cite{hutchinson_1952_possible-relation}
discussed the bremsstrahlung radiation of cosmic-ray electrons.
\cite{morrison_1958_gamma-ray-astronomy} first predicted the detection
of several sources of $\gamma$-rays including solar flares, \acp{PWN},
and active galaxies.

\todo[inline]{
Why Gamma-rays can't make it to the ground
}
    
\todo[inline]{Discuss
Balloon gamma-ray detectors.
See discussion on p859 (comparison with other 
experiments) of Kraushaar et al 1965. 
What was the background from, earth albedo gammas I think?
See also Kraushaar et al 1972 p342's discussion of the balloon
experiments: Hulsizer and Rossi (1949), ... 
See also William Tomkin's section 2.2.1 on 
Balloon experiments (page 8) for references
to galactic plane emission being measured
by balloon experiments in 1970.}


    % Links on Explorer 11
% en.wikipedia.org/wiki/Explorer_11
%http://www.physics.wisc.edu/news/obits/kraushaar_obit.html
% http://heasarc.nasa.gov/docs/heasarc/missions/explorer11.html#reference
% http://articles.adsabs.harvard.edu/cgi-bin/nph-iarticle_query?1965ApJ...141..845K&amp;data_type=PDF_HIGH&amp;whole_paper=YES&amp;type=PRINTER&amp;filetype=.pdf

% "The instrument package weighed 30 pounds, was 20 inches high and 10 inches in diameter. The experimenters believed that they detected 22 cosmic gamma rays."
%  -> http://imagine.gsfc.nasa.gov/docs/sats_n_data/gamma_missions.html
The first space-based $\gamma$-ray detector was \explorerxi
\cite{kraushaar_1965_explorer-experiment}.  It was developed at \ac{MIT}
under the direction of William L. Kraushaar.  It employed a sandwich
scintillator and a Cherenkov counter to direct the position and energy
of incoming $\gamma$-rays and was surounded by a plastic anticoincidence
scintilation counter. The sandwich detector had an area of $\sim45\cm^2$,
but an effective area of only $\sim 7\cm^2$, corresonding
to a detector efficiency of $\sim 15\%$.

\todo[inline]{What was the energy range of explorer ii}

It was launched on boad \explorerxi on April 27,
1961. The instrument was in opreation for 7 months, but only 141 hours
of data were of acceptable quality.  Using these observations, \explorerxi
observed 31 $\gamma$-rays and, because the distribution a distribution of
these $\gamma$-rays was consistent with being isotropic, the experiment
could not firmly identify the $\gamma$-rays as being cosmic in nature.


% en.wikipedia.org/wiki/OSO_3
% "Their
% next detector, on Orbiting Solar Observatory -3, may be more accurately
% described as having proof of the discovery of cosmic gamma radiation,
% since it found a galactic plane anisotropy of high-energy gammas, much
% later to be confirmed with SAS-2 and COS-B." -- http://imagine.gsfc.nasa.gov/docs/sats_n_data/gamma_missions.html

% Notes: 621 photons, E>100 GeV, 1967, angular resolution +/- 16deg from 
%  ``Cosmic Gamma-Ray Sources`` K.S. Chen, Gustavo E. Romer

\todo[inline]{Describe scintilation detector better.
Read William Tomkin's thesis, page 8.}

The first definitive detection of $\gamma$-ray came in
1962 by an experiment on the Ranger 3 moon
probe \citep{arnold_1962_gamma-space}.  It detected an isotropic flux
of $\gamma$-rays in the 0.5 \mev to 2.1 \mev energy range.

\ac{OSO-3}, also developed by Kraushaar, followed \explorerxi
as the next major astrophysical $\gamma$-ray detector
\cite{kraushaar_1972_high-energy-cosmic}.  The \ac{OSO-3} sattelite
allowed the on board $\gamma$-ray detected to have an improved weight,
power, telemetry, and expsoure, creating a more sensitive experiment.
The experiment operated in the energy range from 50 \mev to $\sim 400$
\mev and had an effective area $\sim 9$ $\cm^2$.

It was launched on March 8, 1967 and operated for 16 months, measuring
621 cosmic $\gamma$-rays.  The most important result of the expirment was
to measure a strong anisotrophy in the distribution of the $\gamma$-rays
with a strong clustering of $\gamma$-rays towards the Galactic plane.
\figref{oso3_skymap} shows a skymap of these $\gamma$-rays.  This
experiment confirmed both a Galactic component to the $\gamma$-ray
sky as well as an additional isotropic component, hypothesised to be
extragalactic in origin.

\todo[inline]{What was the PSF of OSO-3? could it be pointed?}

\begin{figure}[htb]
\includegraphics{chapters/introduction/figures/kraushaar_et_al_1972_skymap.pdf}
\figlabel{oso3_skymap}
\caption{The position of all 621 cosmic $\gamma$-rays
detected by \ac{OSO-3}. This figure is from 
\cite{kraushaar_1972_high-energy-cosmic}. }
\end{figure}

% ``The field of gamma-ray astronomy took great leaps forward'' -- wikipedia

  % ``Additional gamma-ray experiments flew on the OGO, OSO, Vela, and Russian
  % Cosmos series of satellites. However, the first satellite designed as a
  % ``dedicated'' gamma-ray mission was the second Small Astronomy Satellite
  % (SAS-2) in 1972.`` - http://imagine.gsfc.nasa.gov/docs/science/know_l1/history_gamma.html



The next major advancement in $\gamma$-ray astronomy came from
the \ac{SAS-2} and \cosb missions.

\ac{SAS-2} was a dedicated $\gamma$-ray detector
launched by \ac{NASA} in November 15, 1972.  \ac{SAS-2} was
\cite{fichtel_1975_high-energy-gamma-ray} It improved upon \ac{OSO-3}
by incorporating a spark chamber and having an overall larger size.
The size of the active area of the detector was 640 $\cm^2$ and the
experiment had a much improved effective area of $\sim 115\,cm^2$. The
spark chamber allowed for a seperate measurement of the electron and
positron tracks, which allowd for improved directional reconstruction
of the incident $\gamma$-ray. \ac{SAS-2} had a PSF $\sim5\degree$ at 30
\mev and $\sim1\degree$ at 1 \gev.

\ac{SAS-2} collected data for over 6 months before a power supply
failure ended data collection. \ac{SAS-2} Observed over 8,000
$\gamma$-ray photons covering $\sim55\%$ of
the sky including most of the Galactic plane.  
\ac{SAS-2} disovered strong emission
along the Galactic plane and particularly towards the Galactic
cente. It also discovered
pulsations from the
Crab \citep{fichtel_1975_high-energy-gamma-ray} and Vela pulsar
\citep{thompson_1977_sas-2-high-energy}.  In addition, \ac{SAS-2}
discovered Geminga, the first $\gamma$-ray source with no compelling
multiwavelenth counterpart \citep{thompson_1977_final-sas-2}. Gemina
was eventually discovered to be a pulsar by \ac{EGRET}
\citep{bertsch_1992_pulsed-high-energy} and retroactivly by \ac{SAS-2}
\citep{mattox_1992_observation-pulsed}.

% ``Cos-B was ESA's first satellite dedicated to a single experiment. Its
% scientific mission was to study in detail the sources of extra-terrestrial
% gamma radiation at energies above about 30 MeV. The originally foreseen
% duration of the mission was two years, but in fact Cos-B functioned
% successfully for 6 years and 8 months. During this time an extensive
% survey of the Galaxy was made in the energy range 50 MeV to 5 GeV.''
% -- http://sci.esa.int/science-e/www/area/index.cfm?fareaid=34

% ``Description Cos-B was the first ESA mission dedicated to the study of
% gamma-ray sources. Its results created a catalogue of these sources,
% known as the 2CG Catalogue, the first complete map of the gamma-ray
% emission from the disc of our Galaxy, the Milky Way, and the first
% detectable emission from an extra-galactic object 3C273.'' 
% -- http://www.esa.int/Our_Activities/Space_Science/Cos-B_factsheet

% cos b performance is described at:
%   http://www.rssd.esa.int/index.php?page=gr-tele&project=COSB

\cosb, an \ac{ESA} mission, was launched shortly after, on August
9, 1975.  Similar to \ac{SAS-2}, \cosb includd a spark chamber for
reading out the $\gamma$-ray events and had a peak effective area
of $\sim 50\,cm^2$ at $\sim400\,MeV$ It improved upon the design
\ac{SAS-2} by including a calorimiter below the spark chamber which
improved the energy resolution to $<100\%$ for energies $\sim 3\,\gev$
\citep{bignami_1975_cos-b-experiment}.

\cosb operated successfully for over 6 years and produced
the first detailed catalog of the $\gamma$-ray sky.
\ac{2CG} detailed the detection 25 $gamma$-ray sources
for $E>100\,\mev$ \citep{swanenburg_1981_second-catalog}.

\begin{figure}[htb]
\includegraphics{chapters/introduction/figures/cos_b_2nd_catalog.pdf}
\figlabel{cos_b_skymap}
\caption{
The figure is from \cite{swanenburg_1981_second-catalog}.
}
\end{figure}


\todo[inline][3C27]

% 6 years and 8 months. During this time an extensive survey of the Galaxy was made in the energy range 50 MeV to 5 GeV.

: 2 Year catalog: 
\todo[inline]{Describe COS-B. What was its effective area}



\begin{itemize}
  \item EGRET
    % Good summary of EGRET Results:
    %  http://arxiv.org/pdf/0811.0738.pdf
  \item AGILE
  \item \todo[inline]{Short description of the history
    of TeV astronomy}
\end{itemize}



