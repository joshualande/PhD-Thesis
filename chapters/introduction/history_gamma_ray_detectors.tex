
\section{The History of Gamma-ray Astrophysics}

% Thorough discussion of all experiments:
%   http://imagine.gsfc.nasa.gov/docs/sats_n_data/gamma_missions.html
%   http://space.about.com/od/telescopesandoptics/a/Gammaray-Astronomy.htm

% Nice history with lots of pictures:
%  http://fermi.gsfc.nasa.gov/science/mtgs/symposia/2012/program/mon/DKniffen.pdf

% Good books with introduction

%  "Very High Energy Gamma Ray Astronomy" by Trevor C. Weekes
%    * seems to have a discussion of the principles of Gamma-ray detection.
%      scintilation detectors, spark chambers.

%  "Cosmic Gamma-Ray Sources" - edited by K.S. Cheng, Gustavo E. Romero

% Very nice reference:
%   http://adsabs.harvard.edu/abs/1984BASI...12..202P

% Nice summary:
%   http://articles.adsabs.harvard.edu/cgi-bin/nph-iarticle_query?1984BASI...12..202P&amp;data_type=PDF_HIGH&amp;whole_paper=YES&amp;type=PRINTER&amp;filetype=.pdf

As is common in the field of physics, the prediction of
the detection of $\gamma$-rays far proceded their discovery.
\cite{cosmic_rays_feenberg_1948} theorized that the interaction
of starlight with cosmic rays could produce $\gamma-rays$ through
\ac{IC} upscattering.  Following the discovery of the neutral
pion in 1949, \cite{propagation_cosmic_radiation_hayakawa_1952}
predicted that $\gamma$-ray emission could be observed from the
decay of neutral pions when cosmic rays interacted with interstellar
matter.  And in the same year, \cite{cosmic_ray_hutchinson_1952}
discussed the bremsstrahlung radiation of cosmic-ray electrons.
\cite{gamma_ray_astronomy_morrison_1958} first predicted the detection
of several sources of $\gamma$-rays including solar flares, \acp{PWN},
and active galaxies.

\todo[inline]{
Why Gamma-rays can't make it to the ground
}
    
\todo[inline]{Discuss
Ballon gamma-ray detectors.
See discussion on p859 (comparison with other 
experiments) of Kraushaar et al 1965. 
What was the background from, earth albedo gammas I think?}


    % Links on Explorer 11
% en.wikipedia.org/wiki/Explorer_11
%http://www.physics.wisc.edu/news/obits/kraushaar_obit.html
% http://heasarc.nasa.gov/docs/heasarc/missions/explorer11.html#reference
% http://articles.adsabs.harvard.edu/cgi-bin/nph-iarticle_query?1965ApJ...141..845K&amp;data_type=PDF_HIGH&amp;whole_paper=YES&amp;type=PRINTER&amp;filetype=.pdf

% "The instrument package weighed 30 pounds, was 20 inches high and 10 inches in diameter. The experimenters believed that they detected 22 cosmic gamma rays."
%  -> http://imagine.gsfc.nasa.gov/docs/sats_n_data/gamma_missions.html
The first space-based $\gamma$-ray detector was \explorerxi
\cite{explorer_xi_kraushaar_1965}.  It was developed at \ac{MIT}
under the direction of William L. Kraushaar.  It employed a sandwich
scintillator and a Cherenkov counter to direct the position and energy
of incoming $\gamma$-rays and was surounded by a plastic anticoincidence
scintilation counter.  It was launched on boad \explorerxi on April 27,
1961.  The instrument was in opreation for 7 months, but only 141 hours
of data were of acceptable quality.  Using these observations, \explorerxi
observed 31 $\gamma$-rays and, because the distribution a distribution of
these $\gamma$-rays was consistent with being isotropic, the experiment
could not firmly identify the $\gamma$-rays as being cosmic in nature.

\begin{itemize}
\item 
  

% en.wikipedia.org/wiki/OSO_3
% "Their
% next detector, on Orbiting Solar Observatory -3, may be more accurately
% described as having proof of the discovery of cosmic gamma radiation,
% since it found a galactic plane anisotropy of high-energy gammas, much
% later to be confirmed with SAS-2 and COS-B." -- http://imagine.gsfc.nasa.gov/docs/sats_n_data/gamma_missions.html

% Notes: 621 photons, E>100 GeV, 1967, angular resolution +/- 16deg from 
%  ``Cosmic Gamma-Ray Sources`` K.S. Chen, Gustavo E. Romer

  The next maor experiment was \ac{OSO-3} is a
  \item SAS-2
  \item COS-B
  \item EGRET
  \item AGILE
\end{itemize}

\subsection{The \fermi Gamma-ray Space Telescope}

\ldots


\subsection{H.E.S.S.}

\ldots
