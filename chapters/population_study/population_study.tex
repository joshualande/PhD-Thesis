\chapter{Population Study of \Acstitle{LAT}-detected \Acsptitle{PWN}}
\chaplabel{population_study}

\paperref{This chapter is based the second part of the paper
  ``Constraints on the Galactic Population of TeV Pulsar Wind Nebulae using Fermi Large Area Telescope Observations''
  by Acero et al which is currently in prep.}

In \chapref{extended_search}, we search for new spatially-extended \fermi
sources and found that spatial extension was an important characteristic
for detecting new \acp{PWN}. In the process, we discovered three new
$\gamma$-ray emitting \acp{PWN} candidates (\hessj{1616}, \hessj{1632},
\hessj{1837}).

In \chapref{offpeak}, we then searched in the off-peak phase interval
of \ac{LAT}-detected pulsars for new pulsar wind nebula and discovered
\threecfiftyeight.  Finally, in \chapref{tevcat} we searched in the
regions surrounding \acp{PWN} candidates detected at \tev energies for
\gev-emitting \acp{PWN} 4 new PWNe candidates (\hessj{1119}, \hessj{1303},
\hessj{1420}, and \hessj{1841}) and 1 new PWN (\hessj{1356})

In this chapter, we take the population of $\gamma$-ray emitting \acp{PWN}
and \acp{PWN} candidates and study how their multiwavelength properties
vary with properties of the associated pulsar.

\begin{deluxetable}{lcclccc}

\tabletypesize{\tiny}
\tablecolumns{7}
\tablewidth{0pt}

\tablecaption{The muliwavelenth properties of the \ac{VHE} source and their associated \ac{LAT}-detected pulsars.
\tablabel{pwn_multiwavelenth_properties}}

\tablehead{\colhead{Source} & \colhead{\FluxPWNTeV} & \colhead{\FluxPWNKeV} & \colhead{PSR} & \colhead{$\dot{E}$} & \colhead{$\tau$} & \colhead{Distance}\\ \colhead{ } & \colhead{($10^{-12}\erg\unitspace\cm^{-2}\second^{-1}$)} & \colhead{($10^{-12}\erg\unitspace\cm^{-2}\second^{-1}$)} & \colhead{ } & \colhead{($\erg\unitspace\second^{-1}$)} & \colhead{(kyr)} & \colhead{(kpc)}}
\startdata
VER\,J0006+727 & \nodata & \nodata & PSR\,J0007+7303 & 4.5e+35 & 13.9 & $1.4 \pm 0.3$ \\
3C\,58 & $<18$ & 5.5 & PSR\,J0205+6449 & 2.6e+37 & 5.5 & 1.95 \\
Crab & $80 \pm 16$ & $21000 \pm 4200$ & PSR J0534+2200 & 4.6e+38 & 1.2 & $2.0 \pm 0.5$ \\
MGRO\,J0631+105 & \nodata & \nodata & PSR\,J0631+1036 & 1.7e+35 & 43.6 & $1.00 \pm 0.20$ \\
MGRO\,J0632+17 & \nodata & \nodata & PSR\,J0633+1746 & 3.2e+34 & 342 & $0.2_{-0.1}^{+0.2}$ \\
Vela$-$X & $79 \pm 21$ & $54 \pm 11$ & PSR\,J0835$-$4510 & 6.9e+36 & 11.3 & $0.29 \pm 0.02$ \\
HESS\,J1018$-$589 & $0.9 \pm 0.4$ & \nodata & PSR\,J1016$-$5857 & 2.6e+36 & 21 & 3 \\
HESS\,J1023$-$575 & $4.8 \pm 1.7$ & \nodata & PSR\,J1023$-$5746 & 1.1e+37 & 4.6 & 2.8 \\
HESS\,J1026$-$582 & $5.9 \pm 4.4$ & \nodata & PSR\,J1028$-$5819 & 8.4e+35 & 90 & $2.3 \pm 0.3$ \\
HESS\,J1119$-$614 & $2.3 \pm 1.2$ & \nodata & PSR\,J1119$-$6127 & 2.3e+36 & 1.6 & $8.4 \pm 0.4$ \\
HESS\,J1303$-$631 & $27 \pm 1$ & $0.16 \pm 0.03$ & PSR\,J1301$-$6305 & 1.7e+36 & 11 & $6.7_{-1.2}^{+1.1}$ \\
HESS\,J1356$-$645 & $6.7 \pm 3.7$ & $0.06 \pm 0.01$ & PSR\,J1357$-$6429 & 3.1e+36 & 7.3 & $2.5_{-0.4}^{+0.5}$ \\
HESS\,J1418$-$609 & $3.4 \pm 1.8$ & $3.1 \pm 0.1$ & PSR\,J1418$-$6058 & 4.9e+36 & 1 & $1.6 \pm 0.7$ \\
HESS\,J1420$-$607 & $15 \pm 3$ & $1.3 \pm 0.3$ & PSR\,J1420$-$6048 & 1.0e+37 & 13 & $5.6 \pm 0.9$ \\
HESS\,J1458$-$608 & $3.9 \pm 2.4$ & \nodata & PSR\,J1459$-$6053 & 9.1e+35 & 64.7 & 4 \\
HESS\,J1514$-$591 & $20 \pm 4$ & $29 \pm 6$ & PSR\,J1513$-$5906 & 1.7e+37 & 1.56 & $4.2 \pm 0.6$ \\
HESS\,J1554$-$550 & $1.6 \pm 0.5$ & $3.1 \pm 1.0$ & \nodata & \nodata & 18 & $7.8 \pm 1.3$ \\
HESS\,J1616$-$508 & $21 \pm 5$ & $4.2 \pm 0.8$ & PSR\,J1617$-$5055 & 1.6e+37 & 8.13 & $6.8 \pm 0.7$ \\
HESS\,J1632$-$478 & $15 \pm 5$ & $0.43 \pm 0.08$ & \nodata & 3.0e+36 & 20 & 3 \\
HESS\,J1640$-$465 & $5.5 \pm 1.2$ & $0.46 \pm 0.09$ & \nodata & 4.0e+36 & \nodata & \nodata \\
HESS\,J1646$-$458B & $5.0 \pm 2.0$ & \nodata & PSR\,J1648$-$4611 & 2.1e+35 & 110 & $5.0 \pm 0.7$ \\
HESS\,J1702$-$420 & $9.0 \pm 3.0$ & $0.01 \pm 0.00$ & PSR\,J1702$-$4128 & 3.4e+35 & 55 & $4.8 \pm 0.6$ \\
HESS\,J1708$-$443 & $23 \pm 7$ & \nodata & PSR\,J1709$-$4429 & 3.4e+36 & 17.5 & $2.3 \pm 0.3$ \\
HESS\,J1718$-$385 & $4.3 \pm 1.6$ & $0.14 \pm 0.03$ & PSR\,J1718$-$3825 & 1.3e+36 & 89.5 & $3.6 \pm 0.4$ \\
HESS\,J1804$-$216 & $12 \pm 2$ & $0.07 \pm 0.01$ & PSR\,J1803$-$2137 & 2.2e+36 & 16 & $3.8_{-0.5}^{+0.4}$ \\
HESS\,J1809$-$193 & $19 \pm 6$ & $0.23 \pm 0.05$ & PSR\,J1809$-$1917 & 1.8e+36 & 51.3 & $3.5 \pm 0.4$ \\
HESS\,J1813$-$178 & $5.0 \pm 0.6$ & \nodata & PSR\,J1813$-$1749 & 6.8e+37 & 5.4 & 4.7 \\
HESS\,J1818$-$154 & $1.3 \pm 0.9$ & \nodata & PSR\,J1818$-$1541 & 2.3e+33 & 9 & $7.8_{-1.4}^{+1.6}$ \\
HESS\,J1825$-$137 & $61 \pm 14$ & $0.44 \pm 0.09$ & PSR\,J1826$-$1334 & 2.8e+36 & 21 & $3.9 \pm 0.4$ \\
HESS\,J1831$-$098 & $5.1 \pm 0.6$ & \nodata & PSR\,J1831$-$0952 & 1.1e+36 & 128 & $4.0 \pm 0.4$ \\
HESS\,J1833$-$105 & $2.4 \pm 1.2$ & $40 \pm 0$ & PSR\,J1833$-$1034 & 3.4e+37 & 4.85 & $4.7 \pm 0.4$ \\
HESS\,J1837$-$069 & $23 \pm 9$ & $0.64 \pm 0.24$ & PSR\,J1836$-$0655 & 5.5e+36 & 2.23 & $6.6 \pm 0.9$ \\
HESS\,J1841$-$055 & $23 \pm 3$ & \nodata & PSR\,J1838$-$0537 & 5.9e+36 & 4.97 & 1.3 \\
HESS\,J1846$-$029 & $9.0 \pm 1.5$ & $29 \pm 1$ & PSR\,J1846$-$0258 & 8.1e+36 & 0.73 & 5.1 \\
HESS\,J1848$-$018 & $4.3 \pm 1.0$ & \nodata & \nodata & \nodata & \nodata & 6 \\
HESS\,J1849$-$000 & $2.1 \pm 0.4$ & $0.90 \pm 0.20$ & PSR\,J1849$-$001 & 9.8e+36 & 42.9 & 7 \\
HESS\,J1857+026 & $18 \pm 3$ & \nodata & PSR\,J1856+0245 & 4.6e+36 & 20.6 & $9.0 \pm 1.2$ \\
MGRO\,J1908+06 & $12 \pm 5$ & \nodata & PSR\,J1907+0602 & 2.8e+36 & 19.5 & $3.2 \pm 0.3$ \\
HESS\,J1912+101 & $7.3 \pm 3.7$ & \nodata & PSR\,J1913+1011 & 2.9e+36 & 169 & $4.8_{-0.7}^{+0.5}$ \\
VER\,J1930+188 & $2.3 \pm 1.3$ & $5.2 \pm 0.1$ & PSR\,J1930+1852 & 1.2e+37 & 2.89 & $9_{-2}^{+7}$ \\
VER\,J1959+208 & \nodata & \nodata & PSR\,J1959+2048 & 1.6e+35 & \nodata & $2.5 \pm 1.0$ \\
MGRO\,J2019+37 & \nodata & \nodata & PSR\,J2021+3651 & 3.4e+36 & 17.2 & $10_{-4}^{+2}$ \\
MGRO\,J2228+61 & \nodata & $0.88 \pm 0.02$ & PSR\,J2229+6114 & 2.2e+37 & 10.5 & $0.80 \pm 0.20$ \\
\enddata

\tablecomments{For the \ac{VHE} \ac{PWN} candidates, this table includes
the multiwavelenth properties of the \ac{PWN}.  This table includes the
X-ray flux in the $2\unitspace\tev$ to $30\unitspace\tev$ energy range
(\FluxPWNKeV) and the flux in \ac{VHE} flux in the $1\unitspace\tev$
to $30\unitspace\tev$ range (\FluxPWNTeV).  In addition, this table
includes the names of the associated pulsars and their spin-down energy,
age, and distance. For several sources, no associated pulsar has been
detected, but properties from an assumed pulsar can be estimated.
The references for all sources in this table for except 3C\,58 can be
found in \cite{acero_2013a_constraints-galactic}.  For 3C\,58, we took the
X-ray flux from \cite{torii_2000a_observations-crab-like}, the \ac{VHE}
flux upper limit from \cite{konopelko_2008a_observations-pulsar}, and the
pulsar properties from \citep{abdo_2013a_second-fermi}.  We note that
there is no error reported on the X-ray flux measurement.}

\end{deluxetable}


In \tabref{pwn_multiwavelenth_properties}, we compile the multiwavelength
properties of the \ac{VHE} sources studied in \chapref{tevcat}. In
particular, we include the spectrum observed at X-ray and \ac{VHE}
energies, the name of the associated pulsar, and the observed spin-down
power, age, and distance of the pulsar.

\begin{figure}[htbp]
  \centering
  \includegraphics{chapters/population_study/figures/pwn_luminosity_vs_edot.pdf}
  \caption{The observed $\gamma$-ray luminosity compared to the
  observed spin-down luminosity for the \ac{PWN} candidates presented
  in \tabref{pwn_multiwavelenth_properties}.}
  \figlabel{pwn_luminosity_vs_edot}
\end{figure}

In \figref{pwn_luminosity_vs_edot}, we compare the observed luminosity at
\gev energies to the spin-down power of the observed pulsar.  This plot
shows that all \ac{LAT}-detected \acp{PWN} emit a fraction $\lesssim 10\%$
of their spin-down energy goes into powering the $\gamma$-ray emission
from the pulsar wind.

\begin{figure}[htbp]
  \centering
  \includegraphics{chapters/population_study/figures/pwn_age_edot_vs_l_gev.pdf}
  \caption{The observed $\gamma$-ray luminosity and the \gev
  to \tev luminosity ratio as a function of the pulsar's age
  and spin-down energy for the \ac{PWN} candidates presented in
  \tabref{pwn_multiwavelenth_properties}.  The dotted line corresponds
  to the average luminosity ratio.  Because \hessj{1708} is classified
  as being a \PSRClass-type source in \chapref{tevcat}, we consider it's
  observed $\gamma$-ray luminosity to be an upper limit on the \ac{PWN}
  emission.  \figlabel{pwn_age_edot_vs_l_gev.pdf}.}
\end{figure}

Next, in \figref{pwn_age_edot_vs_l_gev.pdf} we compare compare \gev
luminosity and \gev to \tev luminosity ratio as a function of age and
spin-down energy.  These plots shows that there is no correlation between
the \gev luminosity and the age and spin-down energy of the associate
pulsar.  In addition, we calculated and overlay the mean between the
\gev and \tev luminosity ($\MeanLuminosityRatio=2.7_{-1.4}^{+2.7}$).

\begin{figure}[htbp]
  \centering
  \includegraphics{chapters/population_study/figures/pwn_age_edot_vs_l_xray.pdf}
  \caption{The observed X-ray flux and \gev to \tev luminosity ratio as
  a function of the pulsar's age and spin-down energy for the \ac{PWN}
  candidates presented in \tabref{pwn_multiwavelenth_properties}.
  The dotted line corresponds to the scaling relationships from
  \cite{mattana_2009_evolution-gamma-} for the \tev to X-ray luminosity
  scaled by the average \gev to \tev luminosity (\MeanLuminosityRatio).
  We caution that \threecfiftyeight is not have a X-ray luminosity error.}
  \figlabel{pwn_age_edot_vs_l_xray}
\end{figure}

Finally, in \figref{pwn_age_edot_vs_l_xray} we compare the distribution
of the X-ray luminosity and the \gev to X-ray luminosity ratio as a
function of the pulsar's age and spin-down energy.  This plot shows
that the X-ray luminosity decreases with pulsar age and increases with
spin-down energy. Similarly, the \gev to X-ray luminosity increases
with age and decreases with energy.  These correlations are consistent
simple model predicted in \cite{mattana_2009_evolution-gamma-} (See
\secref{pwn_emission}) and also with the observed \ac{VHE} relationships
from the same paper.
