
The field of $\gamma$-ray astrophysics has generally found
maximum-likelihood to be a dependable method to avoid the issues
discussed above.  The term likelihood was first introduced by
\cite{likelihood_fisher_1925}.  Maximum-likelihood was applied
to photon-counting experiments in the context of astrophysics by
\cite{likelihood_cash_1979}.  \cite{likelihood_egret_mattox_1996}
described the maximum-likelihood analysis framework devoped to analyze
EGRET data.

In the formulation, one relies upon primarily upon the likelihood
function.  The likelihood, denoted $\likelihood$, is quite simply the
probability of obtaining the observed data given an assumed model:
\begin{equation}
  \likelihood = P(\data|\model)
\end{equation}
\secref{defining_model} will provide describe the
components that go into a model of the data.

Generaly, a model of the sky depends upon a list
of parameters that we denote as $\modelparams$.
Therefore, the likelihood function itself becomes
a function of the parameters of the model:
\begin{equation}
  \likelihood = \likelihood(\modelparams)
\end{equation}
The term maximum-likelihood refers to the fact that
the best-fit parameters of a model can be estimated
by maximizing the likelihood function.

\todo[inline]{What are the benefits of maximum likelihood}

\todo[inline]{Describe Wilk's Therorem and it's application to parameter error estimation}
