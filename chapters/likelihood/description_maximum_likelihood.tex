
The field of $\gamma$-ray astrophysics has generally
found maximum-likelihood to be a dependable method to
avoid the issues discussed above.  The term likelihood was
first introduced by \cite{fisher_1925_statistical-methods}.
Maximum-likelihood was applied to astrophysical photon-counting
experiments by \cite{cash_1979_parameter-estimation}.
\cite{mattox_1996_likelihood-analysis} described the maximum-likelihood
analysis framework developed to analyze \ac{EGRET} data.

In the formulation, one relies upon primarily upon the likelihood
function.  The likelihood, denoted $\likelihood$, is quite simply the
probability of obtaining the observed data given an assumed model:
\begin{equation}
  \likelihood = P(\data|\model)
\end{equation}
\secref{defining_model} will provide describe the
components that go into a model of the data.

Generaly, a model of the sky depends upon a list
of parameters that we denote as $\modelparams$.
Therefore, the likelihood function itself becomes
a function of the parameters of the model:
\begin{equation}
  \likelihood = \likelihood(\modelparams)
\end{equation}
In a maximum-likelihood analysis, one typically
fits parametesr of a model by maximuzing the liklihood
by varying parameters of the model:

\todo[inline]{Finish discussion of why maximum likelihood analysis is so good}
The term maximum-likelihood refers to the fact that
the best-fit parameters of a model can be estimated
by maximizing the likelihood function.
\begin{equation}
\modelparams_\text{max} = \underset{}{\text{arg }}\underset{\modelparams}{\text{max}} \likelihood(\modelparams)
\end{equation}

The primary benefits of the likelihood function is that it allows
very data with varying sensitivities to be combined into
a single quantitity which sensitivly probes the structure one is looking for.

Typically, in a maximum likelihood analysis

\todo[inline]{Describe Wilk's Therorem and it's application to parameter error estimation}
