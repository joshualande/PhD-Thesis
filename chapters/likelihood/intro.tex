
In this chapter, we discuss maximum-likelihood analysis, the principle
analysis method used to perform spectral and spatial analysis of
LAT data.  In \secref{motivations_maximum_likelihood}, we discuss the
reasons necessary for employing this analysis procedure compared to
other simpler analysis methods.  In \secref{defining_model}, we discuss
the steps invovled in defining a complete model of the sky, a necessary
part of any likelihood analysis.

In \secref{binned_science_tools}, we discuss the standard implementation
of binned maximum likelihood in the LAT Science Tools and in particular
the tool \gtlike.

In \secref{pointlike_pacakge}, we then discuss the \pointlike pacakge,
an alterate package for maximum-likelihood analysis of LAT data. We
discuss the similarities and differences between \pointlike and \gtlike.

We not that much of the notation and formulation of likelihood
analysis in this chapter follows \cite{matthew_kerr_thesis}.


