
In this chapter, we discuss maximum-likelihood analysis, the principle
analysis method used to perform spectral and spatial analysis of
LAT data.  In \secref{motivations_maximum_likelihood}, we discuss the
reasons necessary for employing this analysis procedure compared to other
simpler analysis methods.  In \secref{description_maximum_likelihood},
we describe the benefits of a maximum-likelihood analysis.
In \secref{defining_model}, we discuss the steps invovled in defining
a complete model of the sky, a necessary part of any likelihood analysis.

In \secref{binned_science_tools}, we discuss the standard implementation
of binned maximum likelihood in the LAT Science Tools and in particular
the tool \gtlike.
In \secref{pointlike_pacakge}, we then discuss the \pointlike pacakge,
an alterate package for maximum-likelihood analysis of LAT data. We
discuss the similarities and differences between \pointlike and \gtlike.

In the next chapter (\chapref{extended_analysis}),
we will discuss the addition of capability into \pointlike
for studying spatially-extended sources and the
analysis method which will be used in this paper to study
spatially-extended sources.
We not that much of the notation and formulation of likelihood
analysis in this chapter follows \cite{kerr_2010a_likelihood-methods}.


