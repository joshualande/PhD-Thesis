
% Terminology from James Chiang:
%   http://www-glast.slac.stanford.edu/software/datachallenges/dc2/JuneWorkshop/Downloads/Likelihood_performance.pdf
%   https://confluence.slac.stanford.edu/download/attachments/28521/likelihood.pdf
%   https://confluence.slac.stanford.edu/download/attachments/3342362/binned.pdf
\begin{itemize}
  \item For a standard LAT analysis, we perform a binned maximum-likelihood analysis:
  \item In the standard science tools, the data is binned in position and energy.
    and integrated in energy.
  \item For time-serires analysis, typically a time-summed analysyis is performed successivly in
    multiple time bins.
  \item The likelihood comes from a sum over each bin
  \item The likelihood is defined as
    \begin{equation}
      \likelihood=\prod_j \frac{\theta_j^{n_j} e^{-\theta_j}}{n_j!}
    \end{equation}
    \begin{itemize}
      \item Here, $j$ is a sum over position/energy bins.
      \item $\theta_j$ is the counts predicted by the model, which
        is defiend followign the discussion in \secref{defining_model}.
      \item $n_j$ are the observed counts in the spatial/energy bin $j$
    \end{itemize}
  \item The model counts are computed by integrating the differential
    counts defined in \eqnref{eventrate} over the energy bin:
    \begin{equation}
      \theta_{ij} = \int_j \intspace \denergy \intspace 
      \dsolidangle \intspace \dtime \intspace 
      \eventrate(\energy,\solidangle,\time|\modelparams_i)
    \end{equation}
    Here, $j$ represents the integral over the $j$th position/energy bin,
    $i$ represents the $i$th source, and $\modelparams_i$ refers to the
    parmeters defining the $i$th source. The total model counts
    is computed by summing over all sources:
    \begin{equation}
      \theta_j = \sum_i \theta_{ij}
    \end{equation}
  \item In the standard \fermi science tools, 
    the binning of photons over position in the sky and energy to compute $n_j$ 
    is done with \gtbin.
  \item In the standard \fermi science tools, the 
    model counts $\theta_j$ are computed in several steps \ldots

  \item The instrument response is computed with a combination of \gtltcube,
    \gtexpcube.

  \item Convert a model of the sky into model predicted counts
  \item poisson likelihood
  \item Particular implemenation of maximum likelihood anlaysis
  \item Describe \gtbin, \gtselect, \gtlike
\end{itemize}

