\section{Binned Maximum-Likelihood of \Acstitle{LAT} Data with the Science Tools}
\seclabel{binned_science_tools}

We typically use binned maximum-likelihood analysis to analyze \ac{LAT}
data.  In this analysis, the data is binned in position and energy (and
sometimes also separately into front-entering and back-entering events).
The likelihood function comes from the Poisson nature of the observed
emission:
\begin{equation}\eqnlabel{poisson_likelihood_counts_model}
  \likelihood=\prod_j \frac{\theta_j^{k_j} e^{-\theta_j}}{k_j!}.
\end{equation}
Here, $j$ is a sum over position/energy bins,
$k_j$ are the counts observed in bin $j$, and 
$\theta_j$ are the model counts predicted in the same bin.


The model counts in bin $j$ are computed by integrating the differential
model counts over the energy bin:
  \begin{equation}
    \theta_{ij} = \int_j \intspace \denergy \intspace 
    \dsolidangle \intspace \dtime \intspace 
    \eventrate(\energy,\solidangle,\time|\modelparams_i)
  \end{equation}
Here, $j$ represents the integral over the $j$th position/energy bin,
$i$ represents the $i$th source, $\modelparams_i$ refers to the
parmeters defining the $i$th source,
and $\eventrate$ is defined in \eqnref{eventrate} 
. The total model counts
is computed by summing over all sources:
\begin{equation}
  \theta_j = \sum_i \theta_{ij}
\end{equation}

We mention that that in most situations, it is more conveneint to work
with the log of the likelihood, since that quantity varies more slowly.
In addition, typically an analysis requires either 
maximizing the likelihood or looking at a change in the likelihood.
so we
typically write
the likelihood as
\begin{equation}\eqnlabel{log_likelihood_sum_theta}
  \log\likelihood = -\sum_j\theta_j + \sum_j k_j\log\theta_j 
\end{equation}
where we have dropped the arbitrary additive constant $-\log k_j!$.

In the standard \fermi science tools, \gtbin can be used to perform
basic cuts on the $\gamma$-ray data.  the binning of photons over
position in the sky and energy is performed with \gtbin.  The tools
required to compute exposure are \gtltcube and \gtexpcubetwo. Finally,
the likelihood itself is computed with a combination of \gtsrcmaps and
\gtlike.  Essentially, \gtsrcmaps is used to perform two-dimensional
convolution integral in equation \eqnref{differential_model_counts} and
\gtlike is used to compute the likelihood function defined in equation
\eqnref{poisson_likelihood_counts_model}.

As was discussed in \secref{description_maximum_likelihood},
we typically use \acp{LRT} to test for significant features in
the $\gamma$-ray data.  For example, we compare a model with and
without a source of interest to test if that source is significant.
\cite{mattox_1996a_likelihood-analysis} shows that for \ac{EGRET} data,
assuming the position of the source was known and that the spectrum was
fixed, than the distribution of \ts values in the null hypothesis was
\begin{equation}
  \pdf(\ts) = \tfrac{1}{2} (\delta(\ts) + \chi^2_n(\ts))
\end{equation}
From this, one fins that $\ts^{1/2}$ can e used as a measure of the
statistical significance of the detection of a source.

We finaly mention that this formulation assumed that the source models
are time independent.  In principle, these formulas could be generalized
to be binned also in time. But this would almost never be useful because
it is rarely possible to have a simple paremterized model for the time
dependence of a source. Instead, the analyisis of a variable sources
is typically done by dividing the analysis into multiple time intervals
and performing the likelihood fits independently in each time range.


