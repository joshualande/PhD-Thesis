\section{Binned Maximum-Likelihood of \Acstitle{LAT} Data with the Science Tools}
\seclabel{binned_science_tools}

We typically use binned maximum-likelihood analysis to analyze \ac{LAT}
data.  In this analysis, the data is binned in position and energy (and
sometimes also separately into front-entering and back-entering events).
The likelihood function comes from the Poisson nature of the observed
emission:
\begin{equation}
  \likelihood=\prod_j \frac{\theta_j^{k_j} e^{-\theta_j}}{k_j!}.
\end{equation}
Here, $j$ is a sum over position/energy bins,
$k_j$ are the counts observed in bin $j$, and 
$\theta_j$ are the model counts predicted in the same bin.


The model counts in bin $j$ are computed by integrating the differential
model counts defined in \eqnref{eventrate} over the energy bin:
  \begin{equation}
    \theta_{ij} = \int_j \intspace \denergy \intspace 
    \dsolidangle \intspace \dtime \intspace 
    \eventrate(\energy,\solidangle,\time|\modelparams_i)
  \end{equation}
Here, $j$ represents the integral over the $j$th position/energy bin,
$i$ represents the $i$th source, and $\modelparams_i$ refers to the
parmeters defining the $i$th source. The total model counts
is computed by summing over all sources:
\begin{equation}
  \theta_j = \sum_i \theta_{ij}
\end{equation}


\begin{itemize}
  \item For time-serires analysis, typically a time-summed analysyis is performed successivly in
    multiple time bins.
    \begin{itemize}
  \item In the standard \fermi science tools, 
    the binning of photons over position in the sky and energy to compute $n_j$ 
    is done with \gtbin.
  \item In the standard \fermi science tools, the 
    model counts $\theta_j$ are computed in several steps \ldots

  \item The instrument response is computed with a combination of \gtltcube,
    \gtexpcube.

  \item Convert a model of the sky into model predicted counts
  \item poisson likelihood
  \item Particular implemenation of maximum likelihood anlaysis
  \item Describe \gtbin, \gtselect, \gtlike
\end{itemize}


\todo[inline]{put note about source significance testing}
