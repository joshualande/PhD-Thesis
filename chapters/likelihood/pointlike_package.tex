\section{The Alternate Maximum-Likelihood Package \pointlike}
\seclabel{pointlike_package}

\pointlike is an alternative maximum-likelihood framework developed
for analyzing \ac{LAT} data. In principle, both \pointlike and \gtlike
perform the same binned maximum-likelihood analysis described in
\secref{binned_science_tools}. \pointlike's major design difference
is that it was written with efficiency in mind, for certain analysis
procedures which which require multiple iterations, such as source
finding, position and extension fitting, and making large residual
\ts maps.

What makes maximum-likelihood of \ac{LAT} data difficult
is the strongly non-linear performance of the \ac{LAT} (see
\subsecref{performance_lat}). At low energy, one typically finds lots
of counts and each photon is not very important due to the poor angular
resolution. At these energies, a binned analysis with coarse bins is
perfectly adequate to study the sky.  But at high energy, there are
limited numbers of photons due to the limited source fluxes, but the
angular resolution is much improved.  At these energies, an unbinned
analysis which loops only over the photons is more appropriate.

The primary efficiency gain of \pointlike comes from scaling the bin
size with energy, so that the bin size is always comparable to the
\ac{PSF}.  To do this, \pointlike bins the sky into \healpix pixels
\citep{gorski_2005_healpix:-framework}, but only keeps bins with counts
in them.

At low energy, the bins are large and essentially every healpix bin
has counts in it.  But at high energy, bins are very small and rarely
have more than one count in them.  So \pointlike is essentially a binned
analysis at low energy , approximates an unbinned analysis at high energy,
and naturally interpolates between the two extremes.

There is one obvious trade-off for keeping only bins with counts in them.
From \eqnref{log_likelihood_sum_theta}, we note that that the evaluation
of the $\sum_j k_j\log\theta_j$ term can easily be evaluation if only
the counts and model counts are computed in bins with counts in them.
But the $\sum_j \theta_j$ term (the overall model predicted counts in
each bin) can no longer be easily computed since the model counts aren't
computed in every bin. To avoid this, \pointlike has to independently
compute this integral counts.

More details about the implementation of \pointlike can be found in
\cite{kerr_2010a_likelihood-methods}. We will discuss the implementation
of extended sources in \pointlike in \chapref{extended_analysis}.

