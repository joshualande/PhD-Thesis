
\begin{itemize}
  \item
% THis notation is roughly taken from page 23 of Matthew Kerr's Thesis.
% with some parts adapted from 
%   http://www-glast.slac.stanford.edu/software/datachallenges/dc2/JuneWorkshop/Downloads/Likelihood_performance.pdf
Each source can be characterized by its photon flux density 
  $\fluxdensity(\energy,\time,\vec\Omega|\modelparams)$.
This is the number of photons emitted per unit energy, time, into a unit solid angle $d\Omega$
at a given energy, time, and position $\vec\Omega$ in the sky.

\item Typically, spatial and spectral model's are indepndent of time
  with the spatial and spectral component decopuling:
  \begin{equation}
    \fluxdensity(\energy,\time,\vec\Omega|\modelparams) = A(E) B(\vec\Omega)
  \end{equation}
  Presumably, $A$ takes in some of the \modelparams parameters
  and $B$ takes in the other parameters.
\item In situations where there is a time dependence, likelihoood assuming constant
  source is performed in smaller time bins.
\item In situations where spatial and spectral components couple, typical to make
  multiple spatial templates, each with an indepdnet spectra (e.g. the Puppis A paper's
  fitting multiple hemispheres).
\item Discuss how diffuse background is more complciated.
\item Show some examples spectral models: point source, extended source.
\end{itemize}
