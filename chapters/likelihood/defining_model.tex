% This notation folows page 23 of Matthew Kerr's Thesis.
% with some parts adapted from 
%   http://www-glast.slac.stanford.edu/software/datachallenges/dc2/JuneWorkshop/Downloads/Likelihood_performance.pdf

In order to perform a maximum-likelihood analysis, one requires
a parameterized model of the sky. A model of the sky
is composed of a set of $\gamma$-ray sources,
each characterized by its photon flux density 
  $\fluxdensity(\energy,\time,\solidangle|\modelparams)$.
This represents
the number of photons emitted per unit energy, per unit
time, per units solid angle
at a given energy, time, and position in the sky.
In \ac{CGS}, it has units of \fluxdensityunits.

Often, the spatial and spectral part of the source model
are separable and independent of time. When that is the case,
we like to write the source model as
\begin{equation}
  \fluxdensity(\energy,\time,\solidangle|\modelparams) = \dnde \pdf(\solidangle).
\end{equation}
Here, \dnde is only a function of energy and \pdf(\solidangle) is only
a function of position (\solidangle).  In this formulation, some of the
model parameters \modelparams are taken by the $\dnde$ function and some
by the $\pdf(\solidangle)$ function.

The spectrum \dnde is typically modeled by simple geometric functions.
The most popular spectral model is a power law:
  \begin{equation}
    \dnde = \prefactor \left(\frac{E}{\energyscale}\right)^{-\spectralindex}
  \end{equation}
Here, \dnde is a function of energy and also fo the two model parameters
(the prefactor $\prefactor$ and the spectral index $\spectralindex$). The
parameter \energyscale is often called the energy scale or the pivot
energy and is not considered a model parameter.

\begin{itemize}

\item In \ac{CGS}, \dnde is in units of \prefunits.

\item The spatial model is traditionally normalized as though it is a probability
  distribution:
  \begin{equation}
    \int \dsolidangle \intspace \pdf(\solidangle)
  \end{equation}
\item Therefore, in \ac{CGS} \pdf has units of \pdfunits

\item For point-like source as position $\solidangle'$, the spatial model is:
  \begin{equation}
    \pdf(\solidangle) = \delta(\solidangle - \solidangle')
  \end{equation}
\item Example spatial models for spatially-extended sources will be presented
  in section XXXXX\todo{WHAT SECTION DESCRIBES EXTENDED SOURCE PDFs}


\item In situations where there is a time dependence, likelihoood assuming constant
  source is performed in smaller time bins.
\item In situations where spatial and spectral components couple, typical to make
  multiple spatial templates, each with an indepdnet spectra (e.g. the Puppis A paper's
  fitting multiple hemispheres).
\item Discuss how diffuse background is more complciated.
\item Show some examples spectral models: point source, extended source.
\end{itemize}
