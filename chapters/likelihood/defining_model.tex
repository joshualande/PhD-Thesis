\section{Defining a Model of the Sources in the Sky}
\seclabel{defining_model}

In order to perform a maximum-likelihood analysis, one requires a
parameterized model of the sky. A model of the sky is composed of a set
of $\gamma$-ray sources, each characterized by its photon flux density
$\fluxdensity(\energy,\time,\solidangle|\modelparams)$.  This represents
the number of photons emitted per unit energy, per unit time, per
units solid angle at a given energy, time, and position in the sky.
In \ac{CGS}, it has units of \fluxdensityunits.

Often, the spatial and spectral part of the source model are separable
and independent of time. When that is the case, we like to write the
source model as
\begin{equation}
  \fluxdensity(\energy,\time,\solidangle|\modelparams) = \dnde \times \pdf(\solidangle).
\end{equation}
Here, \dndeinline is a function of energy and \pdf(\solidangle) is 
a function of position (\solidangle).  In this formulation, some of
the model parameters \modelparams are taken by the $\dndeinline$ function and
some by the $\pdf(\solidangle)$ function.  In \ac{CGS}, \dndeinline has units
of \prefunits.

The spectrum \dndeinline is typically modeled by simple geometric functions.
The most popular spectral model is a \ac{PL}:
\begin{equation}
  \dnde = \prefactor \left(\frac{E}{\Escale}\right)^{-\spectralindex}
\end{equation}
Here, \dndeinline is a function of energy and also of the two model parameters
(the prefactor $\prefactor$ and the spectral index $\spectralindex$). The
parameter \Escale is often called the energy scale or the pivot energy
and is not fit to the data (sinde it is degenerate with \prefactor).

Another common spectral model is the \ac{BPL}:
\begin{equation}
  \dnde = \prefactor \times
    \begin{cases}
      (E/\Ebreak)^{-\spectralindex_1} &\text{ if }E<\Ebreak \\
      (E/\Ebreak)^{-\spectralindex_2} &\text{ if }E\ge\Ebreak.
    \end{cases}
\end{equation}
This model represents a \ac{PL} with an index of $\spectralindex_1$ which
breaks at energy \Ebreak to having an index of $\spectralindex_2$.

Finally, the \ac{ECPL} spectral model is often used to model the
$\gamma$-ray emission from pulsars:
\begin{equation}
  \dnde = \prefactor \left(\frac{E}{\Escale}\right)^{-\spectralindex}
  \exp\left(-\frac{E}{\Ecutoff}\right).
\end{equation}
For energies much below \Ecutoff, the \ac{ECPL} is a \ac{PL} with
spectral index \spectralindex.  For energies much larger than \Ecutoff,
the \ac{ECPL} spectrum exponentially decreases.

\pdf represents the spatial distribution of the emission.  It is
traditionally normalized as though it was a probability:
\begin{equation}
  \int \dsolidangle \intspace \pdf(\solidangle).
\end{equation}
Therefore, \pdf has units of \pdfunits For a point-like source at a
position $\solidangle'$, the spatial model is:
\begin{equation}
  \pdf(\solidangle) = \delta(\solidangle - \solidangle')
\end{equation}
and is a function of the position of the source ($\solidangle'$).
Example spatial models for spatially-extended sources will be presented
in \subsecref{extension_fitting}.

In some situations, the spatial and spectral part of a source do not
nicely decouple.  An example of this could be a spatially-extended
\acs{SNR} or \acp{PWN} which show a spectral variation
across the source, or equivalently show an energy-dependent
morphology.  \cite{katsuta_2012_fermi-lat-observation} and
\cite{hewitt_2012_fermi-lat-observations} have avoided this issue by
dividing the extended source into multiple non-overlapping extended
source templates which are each allowed to have a different spectra.
