% This notation folows page 23 of Matthew Kerr's Thesis.
% with some parts adapted from 
%   http://www-glast.slac.stanford.edu/software/datachallenges/dc2/JuneWorkshop/Downloads/Likelihood_performance.pdf

In order to perform a maximum-likelihood analysis, one requires
a parameters model of the sky.

Each source can be characterized by its photon flux density 
  $\fluxdensity(\energy,\time,\solidangle|\modelparams)$.
This is the number of photons emitted per unit energy, time, into a unit solid angle
at a given energy, time, and position in the sky.
In \ac{CGS}, it has units of \fluxdensityunits.


\begin{itemize}
\item Often, spatial and spectral model's are indepndent of time
  with the spatial and spectral component of a source model decopuling:
  \begin{equation}
    \fluxdensity(\energy,\time,\solidangle|\modelparams) = \dnde \pdf(\solidangle)
  \end{equation}
\item Here, \dnde is a function of energy and \pdf(\solidangle) is a
function of position (\solidangle).
\item In this formulation $\dnde$ takes in some of the \modelparams parameters
  and $\pdf(\solidangle)$ takes in the other parameters.
\item
  We discuss a few example \dnde functions.
  A power-law spectral model
  \begin{equation}
    \dnde = N_0 \left(\frac{E}{E_0}\right)^{-\gamma}
  \end{equation}
  Here, \dnde is a function of energy and also fo the two model parameters (the prefactor $N_0$ and
  the spectral index $\gamma$.

\item In \ac{CGS}, \dnde is in units of \prefunits.

\item The spatial model is traditionally normalized as though it is a probability
  distribution:
  \begin{equation}
    \int \dsolidangle \intspace \pdf(\solidangle)
  \end{equation}
\item Therefore, in \ac{CGS} \pdf has units of \pdfunits

\item For point-like source as position $\solidangle'$, the spatial model is:
  \begin{equation}
    \pdf(\solidangle) = \delta(\solidangle - \solidangle')
  \end{equation}
\item Example spatial models for spatially-extended sources will be presented
  in section XXXXX\todo{WHAT SECTION DESCRIBES EXTENDED SOURCE PDFs}


\item In situations where there is a time dependence, likelihoood assuming constant
  source is performed in smaller time bins.
\item In situations where spatial and spectral components couple, typical to make
  multiple spatial templates, each with an indepdnet spectra (e.g. the Puppis A paper's
  fitting multiple hemispheres).
\item Discuss how diffuse background is more complciated.
\item Show some examples spectral models: point source, extended source.
\end{itemize}
