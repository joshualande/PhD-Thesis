

The performance of the LAT is composted of two effects.
The efficiency of the LAT referes to its ability to to
reconstruct a photon which comes into the detect.
The disperson of the LAT refers to the probability of
misreconstructing an event. 

The efficiency is written as \efficiency and is measured in units of area ($\cm^2$).

The dispersion is the probability of a photon with true energy \energy
and incoming direction $\solidangle$ at time \time being reconstructed to 
to have an energy $\energy'$, an incomming direction $\solidangle'$ at a time $\time'$.
The dispersion is written as $\dispersion(\energy',\time',\solidangle'|\energy,\time,\solidangle)$.
It represents a probability and is therefore normalized such that
\begin{equation}
  \int \int \int \denergy \dsolidangle \dtime \dispersion(\energy',\time',\solidangle'|\energy,\time,\solidangle) = 1
\end{equation}
Therefore, $\dispersion(\energy',\time',\solidangle'|\energy,\time,\solidangle)$ 
has units of 1/energy/solid angle/time

%We further break the diseprsion into two components, once associated with the
%spatial 

%\todo[inline]{What about temporal dispersion}

We assume these two factors to decouple and write the LAT's instrument response as
\begin{equation}
  \response(\energy',\solidangle',\time'|\energy,\solidangle,\time) = \efficiency \dispersion(\energy',\time',\solidangle'|\energy,\time,\solidangle)
\end{equation}
Therefore, the instrument response has units of area/energy/solid angle/time

The convolution of the flux of a model with the instrument response 
produces the expected counts per unit energy/time/solid angle
begin reconstructed to have 
an energy $\energy'$ 
at a position $\solidangle'$ and at a time $\time'$:
\begin{equation}
  \eqnlabel{eventrate}
  \eventrate(\energy',\solidangle',\time'|\modelparams)
  = \int \int \int \denergy \, \dsolidangle \, \dtime \,
  \fluxdensity(\energy,\time,\vec\Omega|\modelparams) \response(\energy',\solidangle',\time'|\energy,\solidangle,\time)
\end{equation}
Here, this integral is performed over all true energies, solid angles, and times
for which the source model has support.

For LAT analysis, we conventionally make the simplifying assumption that
the energy , spatial , and time dispersion decouple:
\begin{equation}
  \dispersion(\energy',\time',\solidangle'|\energy,\time,\solidangle) = 
  \psf(\solidangle'|E,\solidangle) \times \edisp(\energy'|\energy) \times \tdisp(\time'|\time)
\end{equation}

Here, \psf is the point-spread function and represents 

\edisp is the energy dispersion. \todo{Why do we discard energy dispersion. Find reference.}

\tdisp is the time dispersion. \todo{Why discard time dispersion}

