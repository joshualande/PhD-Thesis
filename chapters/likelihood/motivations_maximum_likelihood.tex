
Traditonally, spectral and spatial analysis of astrophysical data
relies on a process known as aperture photometry.  In this process,
a source in the data is analyzed by directly measuring the number of
photons coming from the object. This process is done by measuring the
counts within a given radius of the source and subtracting from it a
background level estimated from a nearby region.  Often, the source's
flux is calibrated by measurements of nearby objects with known fluxes.
Otherwise, the flux can be obtained from by dividing the number of
counts from the source by the telesocpe's size, the observation time,
and the telescope's conversion efficiency.

Similarly, for faint sources the statistical significance of the
detection can be obtained from the Poission nature of the data. For
TeV experiments such as H.E.S.S., this analysis method is described in
\cite{background_estimation_li_ma_1983}.

Unfortunately, this simpler analysis method is inadequate for dealing
with the complexities introduced in analyzing LAT data.  

Most importantly, aperture photometry assumed that the
background is istropic so that the background level below the source
can be estimated from nearby regions.
As was discussed in \secref{modeling_background},
the Galactic diffuse emission is highly anistropic,
rendering this assumption invalid.


% Note, number of square degrees in whole sky is 
%  * solid angle = 4*pi(180/pi)^2 = 129600/pi ~ 41,253 deg^2
%  * sqrt(41253 deg^2/1873) = 4.69 deg ~ 

% Command to find number of sources with LAT<=5degrees
% >>> import pyfits as pf,numpy as np
% >>> x=pf.open("gll_psc_v07.fit")[1].data;
% >>> lon,lat=x["GLON"],x["GLAT"]
% >>> cut=(np.abs(lat)<=0.5)&((lon<=0+45)|(lon>=360-45))
% >>> print np.sum(cut),len(cut)
% ... 73 1873

% Approximate Solid angle of plane ~ 90deg * 1deg = 90 deg^2

In addition,
this method is not optimal due to the high density of sources
detected in the Gamma-ray sky.  \ac{2FGL} reported on the detection
of 1873 sources, which corresponds to an average source spacing of
$\sim5\degree$.  But within the inner $45\degree$ of the galactic plane
in longitude and $0.5\degree$ of the galactic plane in latitidue, there
are 73 sources, corresponding to a source density of $\sim 1$ source per
square degree.  The aperature photometry method is unable to effecitvly
fit multiple sources when the tails of the PSF overlap and furthermore
make background estimation problematic.

Finally, this method is suboptimal due to the large energy range of
LAT observations.  A typical spectral analysis studies a source from
an energy of 100 \mev to energies above 100 \gev.  Similarly, as was
shown in \todo{what section discusses energy dependent psf?}, the PSF
of the LAT is rather broad ($\gtrsim 1\degree$) at low energy and much
narrower ($\sim 0.1\degree$) at higher energies. Therefore, there is a
much higher sensitivty to the higher energy photons coming from a source.
But simple aperture photometry method would ignore this improvement by
weighting each photon equally.

