\chapter{Maximum-likelihood analysis of LAT data}

\chaplabel{maximum_likelihood_analysis}

\begin{itemize}
  \item The notation and terminology follows the convetion in 
\end{itemize}

\section{Motivations for Maximum-Likelihood Analysis of Gamma-ray Data}

\begin{itemize}
  \item Traditonal astrophysical analysis involves an on minus of background estimation.
  \item Analysis of LAT data more complciated due to:
    \begin{itemize}
      \item Anisotrpic background. See \secref{modeling_background}.
      \item Energy-dependent PSF
      \item High source density, espeically in the Galactic plane.
    \end{itemize}
  \item To avoid issues assocaited with this, we perform a maximum likelihood analysis
  \item Define a model of the sky.
  \item likelihood $L$ is defiend as $L=P(data|model)$, where $L=L(model parameters)$.
  \item Benefits: XXX
\end{itemize}

\section{Defining a Model of the Sources in the Sky}

\begin{itemize}
  \item Sky model must predict the emisson 
  \item Issues with maximum 
\end{itemize}

% THis notation is roughly taken from page 23 of Matthew Kerr's Thesis.
% with some parts adapted from 
%   http://www-glast.slac.stanford.edu/software/datachallenges/dc2/JuneWorkshop/Downloads/Likelihood_performance.pdf
Each source can be characterized by its photon flux density (number
of photons emitted per unit energy, time, into a unit solid angle $d\Omega$)
at a given energy, time, and position $\vec\Omega$ in the sky:
\begin{equation}
  \mathcal{F}(E,t,\vec\Omega)
\end{equation}

\section{The LAT Instrument Response Functions}

\begin{itemize}
  \item The instrument response of the LAT can be factored
  \item The exposure
  \item
\end{itemize}

\section{Application of Binned Maximum-Likelihood to LAT Data with the Science Tools}

\begin{itemize}
  \item In the standard science tools, the data is binned in position and energy.
    The counts in each bin are
  \item In the standard science tools, 
    \begin{itemize}
      \item the binning is done with \gtbin.
      \item The instrument response is computed with a combination of \gtltcube,
        \gtexpcube
      \item The integral over 
    \end{itemize}
  \item Bin the LAT data
  \item Convert a model of the sky into model predicted counts
  \item poisson likelihood
  \item Particular implemenation of maximum likelihood anlaysis
  \item Describe \gtbin, \gtselect, \gtlike
\end{itemize}




\section{The Alternate Maximum-Likelihood Pacakge \pointlike}

\begin{itemize}
\item Developed for Speed
\item Sparce Matricies, 
\end{itemize}

\section{Extended Source Analysis in \pointlike}
