\section{The \Acstitle{LAT} Instrument Response Functions}
\seclabel{lat_irfs}


The performance of the LAT is quantified by its effective area
and its dispersion. The effecive area represents the collection area of the
\ac{LAT} and the dispersion represents the probaility of misreconstructing
the true parameters of the incident $\gamma$-ray.

The
effective area $\effectivearea(\energy,\time,\solidangle)$
is a function of energy, time, and \ac{SA} and is measured in units of 
$\cm^2$. 

The dispersion is the probability of a photon with true energy
\energy and incoming direction $\solidangle$ at time \time being
reconstructed to have an energy $\energy'$, an incomming direction
$\solidangle'$ at a time $\time'$.  The dispersion is written as
$\dispersion(\energy',\time',\solidangle'|\energy,\time,\solidangle)$.
It represents a probability and is therefore normalized such that
\begin{equation}
  \int \int \int \denergy \dsolidangle \dtime 
  \dispersion(\energy',\time',\solidangle'|\energy,\time,\solidangle) = 1
\end{equation}
Therefore,
$\dispersion(\energy',\time',\solidangle'|\energy,\time,\solidangle)$
has units of 1/energy/\acl{SA}/time

The convolution of the flux of a model with the instrument response 
produces the expected counts per unit energy/time/\acl{SA}
begin reconstructed to have 
an energy $\energy'$ 
at a position $\solidangle'$ and at a time $\time'$:
\begin{equation}
  \eqnlabel{eventrate}
  \eventrate(\energy',\solidangle',\time'|\modelparams)
  = \int \int \int \denergy \, \dsolidangle \, \dtime \,
  \fluxdensity(\energy,\time,\vec\Omega|\modelparams) 
  \effectivearea(\energy,\time,\solidangle) \dispersion(\energy',\time',\solidangle'|\energy,\time,\solidangle)
\end{equation}
Here, this integral is performed over all energies, \acp{SA}, and times
for which the source model has support.

For LAT analysis, we conventionally make the simplifying assumption that
the energy, spatial, and temporal dispersion decouple:
\begin{equation}
  \dispersion(\energy',\time',\solidangle'|\energy,\time,\solidangle) = 
  \psf(\solidangle'|E,\solidangle) \edisp(\energy'|\energy) \tdisp(\time'|\time)
\end{equation}

\edisp represents the energy dispersion of the LAT.  The energy dispersion
of the LAT is a function of both the incident energy and incident angle
of the photon. It varies from $\sim$ 5\% to 20\%, degrading at lower
energies due to energy losses in the tracker and at higher energy due
to electromagnetic shower losses outside the calorimiter. Similarly,
it improves for photons with higher incident angles that are allowed a
longer path through the calorimieter \citep{ackermann_2012a_fermi-large}.

Here, $\psf(\solidangle'|E,\solidangle)$ is the probability of
reconstructing a $\gamma$-ray to have a position $\solidangle'$ if
the true position of the $\gamma$-ray has a position $\solidangle$.
\subsecref{performance_lat} includes a plot of the \psf of the \ac{LAT}.


Finally, we note that in principle, there is a finite timing resolution
of $\gamma$-rays measured by the \ac{LAT}. But the timing accuracy is
$<10\unitspace\microsecond$ \citep{atwood_2009a_large-telescope}. Since
this is much less than the smallest timing signal which is expected to
be observed by the \ac{LAT} (millisceond pulsars), issues with timing
accuracy are typically ignored.  \subsecref{performance_lat} includes
a plot of the \edisp of the \ac{LAT}.

For a typical analysis of \ac{LAT} data, we also ingore the inherent
energy dipsersion of the \ac{LAT}.  \cite{ackermann_2012a_fermi-large}
performed a monte carlo simulation to show that for power-law point-like
sources, the bias introduced by ignoring energy dispersion was on the
level of a few perect.  
Therefore, the instrument response is typically approximated as
\begin{equation}
  \response(\energy',\solidangle',\time'|\energy,\solidangle,) = 
  \effectivearea(\energy,\time',\solidangle) \psf(\solidangle'|E,\solidangle)
\end{equation}
We cauation that for analysis of sources extended
to energies below $100\unitspace\mev$, the effects of energy dispersion
could be more severe.

The expected count rate is then typically integrated over time
to compute the total counts. Assuming that the source model
is time indepdendent, we get: 
\begin{equation}\eqnlabel{differential_model_counts}
  \eventrate(\energy',\solidangle'|\modelparams)
  = \int \dsolidangle \,
  \fluxdensity(\energy',\vec\Omega|\modelparams) 
\left(
\int \dtime \intspace \effectivearea(\energy',\time,\solidangle) 
\right)
\psf(\solidangle'|E,\solidangle)
\end{equation}
This equation says that the counts expected by the LAT
from a given model
is the product of the source's flux with the effective area and then
convolved with the point-spread function.

Finally, we note that the \psf and effective area is also a function of the conversion type
of the $\gamma$-ray (front-entering or back-entering event), and the azimuthal
angle of the $\gamma$-ray. These formulas can be readily generalized to include these effects.
