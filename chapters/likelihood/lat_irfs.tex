

The performance of the LAT is composted of two effects.
The efficiency of the LAT referes to its ability to to
reconstruct a photon which comes into the detect.
The disperson of the LAT refers to the probability of
misreconstructing an event. 


The efficiency is typically called the effective area.
We write it as $\effectivearea(\energy,\time,\solidangle)$.
It is a function of energy, time, and \gls{SA}.
It is measured in units of area ($\cm^2$).

\todo[inline]{LINK TO arXiv:1206.1896 for MORE THOUROUGH
DISCUSSION OF EFFECTIVE AREA}

\todo[inline]{DISCUSS HOW EFFECTIVE AREA IS A FUNCTION OF DIFFERENT THINGS}

The dispersion is the probability of a photon with true energy \energy
and incoming direction $\solidangle$ at time \time being reconstructed to 
to have an energy $\energy'$, an incomming direction $\solidangle'$ at a time $\time'$.
The dispersion is written as $\dispersion(\energy',\time',\solidangle'|\energy,\time,\solidangle)$.
It represents a probability and is therefore normalized such that
\begin{equation}
  \int \int \int \denergy \dsolidangle \dtime \dispersion(\energy',\time',\solidangle'|\energy,\time,\solidangle) = 1
\end{equation}
\todo[inline]{What is the range of the integrals}
Therefore, $\dispersion(\energy',\time',\solidangle'|\energy,\time,\solidangle)$ 
has units of 1/energy/\glstext{SA}/time

%We further break the diseprsion into two components, once associated with the
%spatial 

%\todo[inline]{What about temporal dispersion}

We assume these two factors to decouple and write the LAT's instrument response as
\begin{equation}
  \response(\energy',\solidangle',\time'|\energy,\solidangle,\time) = 
\effectivearea(\energy,\time,\solidangle) \dispersion(\energy',\time',\solidangle'|\energy,\time,\solidangle)
\end{equation}
Therefore, the instrument response has units of area/energy/\glstext{SA}/time

The convolution of the flux of a model with the instrument response 
produces the expected counts per unit energy/time/\glstext{SA}
begin reconstructed to have 
an energy $\energy'$ 
at a position $\solidangle'$ and at a time $\time'$:
\begin{equation}
  \eqnlabel{eventrate}
  \eventrate(\energy',\solidangle',\time'|\modelparams)
  = \int \int \int \denergy \, \dsolidangle \, \dtime \,
  \fluxdensity(\energy,\time,\vec\Omega|\modelparams) \response(\energy',\solidangle',\time'|\energy,\solidangle,\time)
\end{equation}
Here, this integral is performed over all true energies, \glsplural{SA}, and times
for which the source model has support.

For LAT analysis, we conventionally make the simplifying assumption that
the energy , spatial , and time dispersion decouple:
\begin{equation}
  \dispersion(\energy',\time',\solidangle'|\energy,\time,\solidangle) = 
  \psf(\solidangle'|E,\solidangle) \times \edisp(\energy'|\energy) \times \tdisp(\time'|\time)
\end{equation}

Here, \psf is the point-spread function and represents \ldots
\todo[inline]{BETTER DISCUSSION OF PSF OF THE LAT, WHAT ITS SCALE IS\ldots}

% Information from section 7.1.3 of Eric Charles' calibration paper (arXiv:1206.1896)
\edisp represents the energy dispersion of the LAT.
The energy dispersion of the LAT is a function of both the
incident energy and incident angle of the photon. It varies
from $\sim$ 5\% to 20\%, degrading at lower energies due to energy
losses in the tracker and at higher energy due to electromagnetic
shower losses outside the calorimiter. Similarly, it improves
for photons with higher incident angles that are 
allowed a longer path through the calorimieter \citep{ackermann_2012a_fermi-large}.

% This information comes from seciton 7.4 of Eric Charles' calibration paper (arXiv:1206.1896)
For sources with smoothly-varying spectra, the effects of ignoring
the inherent energy dispersion of the LAT are typically small.
\cite{ackermann_2012a_fermi-large} performed a monte carlo simulation to show
that for power-law point-like sources, the bias introduced by ignoring
energy dispersion was on the level of a few perect.  Therefore, energy
dispersion is typicially ignored for standard likelihood analysis:
\begin{equation}
\edisp = \delta(\energy-\energy')
\end{equation}
We cauation that for analysis of sources extended to energies below 100
MeV and for sources expected to have spectra that do not smoothly vary,
the effects of energy dispersion could be more severe.

\begin{itemize}
  \item \tdisp is the time dispersion. 
  \item \todo{Why discard time dispersion}
  \item The timing dispersion is $<10$ \microsecond\cite{atwood_2009a_large-telescope}
\item \todo[inline]{WRITE ENERGY DISPERSION AS A DELTA FUNCTION}
\end{itemize}

\todo{FINISH}
Therefore, the instrument response is typically approximated as
\begin{equation}
  \response(\energy',\solidangle',\time'|\energy,\solidangle,) = 
  \effectivearea(\energy,\time',\solidangle) \psf(\solidangle'|E,\solidangle)
\end{equation}
The expected count rate is then typically integrated over time
to compute the total counts. Assuming that the source model
is time indepdendent, we get: 
\begin{equation}
  \eventrate(\energy',\solidangle'|\modelparams)
  = \int \dsolidangle \,
  \fluxdensity(\energy,\vec\Omega|\modelparams) 
\left(
\int \dtime \intspace \effectivearea(\energy,\time,\solidangle) 
\right)
\psf(\solidangle'|E,\solidangle)
\end{equation}
This equation essentially says that the counts expected by the LAT
for the particular model
is the product of the source's flux with the effective area and then
convolved with the point-spread function.

\todo[inline]{Figure out how the $\theta$ depedence of the IRFs factors into this calcualtion}


